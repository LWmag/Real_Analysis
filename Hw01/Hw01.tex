\documentclass{article}

\usepackage{fancyhdr}
\usepackage{extramarks}
\usepackage{amsmath}
\usepackage{amsthm}
\newtheorem{lemma}{Lemma}
\usepackage{amsfonts}
\usepackage{tikz}
\usepackage[plain]{algorithm}
\usepackage{algpseudocode}

\usetikzlibrary{automata,positioning}

%
% Basic Document Settings
%

\topmargin=-0.45in
\evensidemargin=0in
\oddsidemargin=0in
\textwidth=6.5in
\textheight=9.0in
\headsep=0.25in

\linespread{1.1}

\pagestyle{fancy}
\lhead{\hmwkAuthorName}
\chead{\hmwkClass\ (\hmwkClassInstructor\ \hmwkClassTime): \hmwkTitle}
\rhead{\firstxmark}
\lfoot{\lastxmark}
\cfoot{\thepage}

\renewcommand\headrulewidth{0.4pt}
\renewcommand\footrulewidth{0.4pt}

\setlength\parindent{0pt}

%
% Create Problem Sections
%

\newcommand{\enterProblemHeader}[1]{
    \nobreak\extramarks{}{Problem \arabic{#1} continued on next page\ldots}\nobreak{}
    \nobreak\extramarks{Problem \arabic{#1} (continued)}{Problem \arabic{#1} continued on next page\ldots}\nobreak{}
}

\newcommand{\exitProblemHeader}[1]{
    \nobreak\extramarks{Problem \arabic{#1} (continued)}{Problem \arabic{#1} continued on next page\ldots}\nobreak{}
    \stepcounter{#1}
    \nobreak\extramarks{Problem \arabic{#1}}{}\nobreak{}
}

\setcounter{secnumdepth}{0}
\newcounter{partCounter}
\newcounter{homeworkProblemCounter}
\setcounter{homeworkProblemCounter}{1}
\nobreak\extramarks{Problem \arabic{homeworkProblemCounter}}{}\nobreak{}

%
% Homework Problem Environment
%
% This environment takes an optional argument. When given, it will adjust the
% problem counter. This is useful for when the problems given for your
% assignment aren't sequential. See the last 3 problems of this template for an
% example.
%
\newenvironment{homeworkProblem}[1][-1]{
    \ifnum#1>0
        \setcounter{homeworkProblemCounter}{#1}
    \fi
    \section{Problem \arabic{homeworkProblemCounter}}
    \setcounter{partCounter}{1}
    \enterProblemHeader{homeworkProblemCounter}
}{
    \exitProblemHeader{homeworkProblemCounter}
}

%
% Homework Details
%   - Title
%   - Due date
%   - Class
%   - Section/Time
%   - Instructor
%   - Author
%

\newcommand{\hmwkTitle}{Homework\ \#1}
\newcommand{\hmwkDueDate}{Mar 4, 2024}
\newcommand{\hmwkClass}{Real Analysis}
\newcommand{\hmwkClassTime}{Monday}
\newcommand{\hmwkClassInstructor}{Professor Yakun Xi}
\newcommand{\hmwkAuthorName}{\textbf{Shuang Hu}}

%
% Title Page
%

\title{
    \vspace{2in}
    \textmd{\textbf{\hmwkClass:\ \hmwkTitle}}\\
    \normalsize\vspace{0.1in}\small{Due\ on\ \hmwkDueDate\ at 10:00am}\\
    \vspace{0.1in}\large{\textit{\hmwkClassInstructor\ \hmwkClassTime}}
    \vspace{3in}
}

\author{\hmwkAuthorName}
\date{}

\renewcommand{\part}[1]{\textbf{\large Part \Alph{partCounter}}\stepcounter{partCounter}\\}

%
% Various Helper Commands
%

% Useful for algorithms
\newcommand{\alg}[1]{\textsc{\bfseries \footnotesize #1}}

% For derivatives
\newcommand{\deriv}[1]{\frac{\mathrm{d}}{\mathrm{d}x} (#1)}

% For partial derivatives
\newcommand{\pderiv}[2]{\frac{\partial}{\partial #1} (#2)}

% Integral dx
\newcommand{\dx}{\mathrm{d}x}

% Alias for the Solution section header
\newcommand{\solution}{\textbf{\large Solution}}
\newcommand{\norm}[1]{\|#1\|}
% Probability commands: Expectation, Variance, Covariance, Bias
\newcommand{\E}{\mathrm{E}}
\newcommand{\Var}{\mathrm{Var}}
\newcommand{\Cov}{\mathrm{Cov}}
\newcommand{\Bias}{\mathrm{Bias}}
\newcommand{\supp}{\text{supp}}
\newcommand{\Rn}{\mathbb{R}^{n}}
\newcommand{\dif}{\mathrm{d}}
\newcommand{\avg}[1]{\left\langle #1 \right\rangle}
\newcommand{\difFrac}[2]{\frac{\dif #1}{\dif #2}}
\newcommand{\pdfFrac}[2]{\frac{\partial #1}{\partial #2}}
\newcommand{\OFL}{\mathrm{OFL}}
\newcommand{\UFL}{\mathrm{UFL}}
\newcommand{\fl}{\mathrm{fl}}
\newcommand{\op}{\odot}
\newcommand{\cp}{\cdot}
\newcommand{\Eabs}{E_{\mathrm{abs}}}
\newcommand{\Erel}{E_{\mathrm{rel}}}
\newcommand{\DR}{\mathcal{D}_{\widetilde{LN}}^{n}}
\newcommand{\add}[2]{\sum_{#1=1}^{#2}}
\newcommand{\innerprod}[2]{\left<#1,#2\right>}
\newcommand\tbbint{{-\mkern -16mu\int}}
\newcommand\tbint{{\mathchar '26\mkern -14mu\int}}
\newcommand\dbbint{{-\mkern -19mu\int}}
\newcommand\dbint{{\mathchar '26\mkern -18mu\int}}
\newcommand\bint{
{\mathchoice{\dbint}{\tbint}{\tbint}{\tbint}}
}
\newcommand\bbint{
{\mathchoice{\dbbint}{\tbbint}{\tbbint}{\tbbint}}
}
\begin{document}
\maketitle
\pagebreak
\begin{homeworkProblem}{Complete the proof of Proposition 1.2.}
\begin{proof}
    \begin{itemize}
        \item choose $N_{0}\in\mathbb{N}$ such that 
        $\frac{1}{N_{0}}<\frac{1}{b-a}$, 
        \begin{displaymath}
            (a,b)=\cup_{n=N_{0}}^{\infty}\left(a,b-\frac{1}{n}\right].
        \end{displaymath}
        \item choose $N_{0}\in\mathbb{N}$ such that 
        $\frac{1}{N_{0}}<\frac{1}{b-a}$, 
        \begin{displaymath}
            (a,b)=\cup_{n=N_{0}}^{\infty}\left[a+\frac{1}{n},b\right),
        \end{displaymath}
        \item \begin{displaymath}
            (a,b)=(a,\infty)\cap
            \left(\cap_{n=1}^{\infty}
            \left(b-\frac{1}{n},\infty\right)\right)^{c},
        \end{displaymath}
        \item \begin{displaymath}
            (a,b)=(-\infty,b)\cap
            \left(\cap_{n=1}^{\infty}
            \left(-\infty,a+\frac{1}{n}\right)\right)^{c},
        \end{displaymath}
        \item \begin{displaymath}
            (a,b)=[b,\infty)^{c}
            \cap\left(\cup_{n=1}^{\infty}
            \left[a+\frac{1}{n},\infty\right)\right),
        \end{displaymath}
        \item \begin{displaymath}
            (a,b)=(-\infty,a]^{c}
            \cap\left(\cup_{n=1}^{\infty}
            \left(-\infty,b-\frac{1}{n}\right]\right).
        \end{displaymath}
    \end{itemize}   
    By the definition of 
    generated $\sigma$-algebra, 
    it means $(a,b)\in\mathcal{M}(\mathcal{E}_{i})$, $\forall i=3,4,5,6,7,8$.  
    By Lemma 1.1, 
    $\mathcal{B}_{\mathbb{R}}\subset\mathcal{M}(\mathcal{E}_{i})$. 
    Then we complete the proof of Proposition 1.2.
    \qedhere
\end{proof}
\end{homeworkProblem}
\begin{homeworkProblem}
    Let $\mathcal{M}$ be an infinite 
    $\sigma$-algebra
    \begin{enumerate}
        \item $\mathcal{M}$ contains an infinite sequence 
        of disjoint sets.
        \item $\text{card}(\mathcal{M})\ge\aleph$.
    \end{enumerate}
\end{homeworkProblem}
\begin{proof}
    $(1)$. $\mathcal{M}$ be an infinite $\sigma$-algebra 
    means that there exists at least $\aleph_{0}$ 
    distinct non-empty sets 
    $\{\mathcal{S}_{i}\}_{i=1}^{\infty}\subset\mathcal{M}$.
    Unfortunately, $\{\mathcal{S}_{i}\}_{i=1}^{\infty}$ 
    isn't a sequence of disjoint sets in general.
    
    So, we should check the following lemma:
    \begin{lemma}
        $\forall n\in\mathbb{N}$, 
        if $\{\mathcal{S}_{i}\}_{i=1}^{n}\subset\mathcal{M}$ 
        are distinct non-empty sets,
        there exists a sequence of disjoint non-empty sets 
        $\{\mathcal{T}_{i}\}_{i=1}^{n}$ 
        such that $\mathcal{T}_{i}\in\mathcal{M}$, and 
        $\cup_{i=1}^{n}\mathcal{T}_{i}=\cup_{i=1}^{n}\mathcal{S}_{i}$.
    \end{lemma}
    \begin{proof}
        For $n=1$, 
        just set $\mathcal{T}_{1}=\mathcal{S}_{1}$.
        Assume the lemma holds for $n=k$, 
        consider $\{\mathcal{S}_{i}\}_{i=1}^{k+1}$. 
        By the assumption, 
        there exists a sequence of disjoint non-empty sets 
        $\{\tilde{\mathcal{S}}_{i}\}_{i=1}^{k}\subset\mathcal{M}$ 
        such that $\cup_{i=1}^{k}\mathcal{S}_{i}
        =\cup_{i=1}^{k}\tilde{\mathcal{S}}_{i}$. 
        For $\mathcal{S}_{k+1}$, there are three distinct cases:
        \begin{itemize}
            \item $\exists i\le k$, 
            $\mathcal{S}_{k+1}\subset\tilde{\mathcal{S}}_{i}$.
            \item $\forall i\le k$, 
            $\mathcal{S}_{k+1}\cap\tilde{\mathcal{S}}_{i}=\emptyset$.
            \item Other cases.
        \end{itemize}
        For case 1, we choose 
        $\mathcal{T}_{j}=\tilde{\mathcal{S}}_{j}$ 
        for $j\le k$ and $j\neq i$, 
        $\mathcal{T}_{i}=\tilde{\mathcal{S}}_{i}
        \setminus\mathcal{S}_{k+1}$, 
        $\mathcal{T}_{k+1}=\mathcal{S}_{k+1}$. 
        For case 2, we choose 
        $\mathcal{T}_{i}=\tilde{\mathcal{S}}_{i}$ for $i\le k$, 
        $\mathcal{T}_{k+1}=\mathcal{S}_{k+1}$. 
        For case 3, we choose 
        $\mathcal{T}_{i}=\tilde{\mathcal{S}}_{i}$ for $i\le k$, 
        $\mathcal{T}_{k+1}=\mathcal{S}_{k+1}\setminus
        \left(\cup_{i=1}^{n}\tilde{\mathcal{S}}_{i}\right)$.
        Then the sequence $\{\mathcal{T}_{i}\}_{i=1}^{n}$ 
        satisfies the condition in Lemma 1. 
        By induction, we complete the proof.\qedhere
    \end{proof}
    The lemma means that $\forall n$, 
    $\exists$ a sequence of distinct non-empty disjoint sets 
    $\{\mathcal{T}_{i}\}_{i=1}^{n}$ 
    such that $\mathcal{T}_{i}\in\mathcal{M}$. 
    Then, $\{\mathcal{T}_{i}\}_{i=1}^{\infty}$ is just 
    the sequence of disjoint sets.

    (2) Choose the sequence 
    $\{\mathcal{T}_{i}\}_{i=1}^{\infty}$ in (1), 
    as $\{\mathcal{T}_{i}\}$ are non-empty disjoint sets, 
    $\forall \mathcal{S}_{1},\mathcal{S}_{2}\subset\mathbb{N}$ 
    and $\mathcal{S}_{1}\neq\mathcal{S}_{2}$, 
    \begin{displaymath}
        \left(\cup_{i\in\mathcal{S}_{1}}\mathcal{T}_{i}\right)
        \neq
        \left(\cup_{j\in\mathcal{S}_{2}}\mathcal{T}_{j}\right).
    \end{displaymath}
    And by the definition of $\sigma$-algebra, 
    $\forall\mathcal{S}\subset\mathbb{N}$, 
    $\cup_{i\in\mathcal{S}}\mathcal{T}_{i}\in\mathcal{M}$.
    It means that:
    \begin{displaymath}
        \text{card}(\mathcal{M})\ge\text{card}
        \left(2^{\mathbb{N}}\right)
        =\aleph.
    \end{displaymath}
\end{proof}
\begin{homeworkProblem}
    An algebra $\mathcal{A}$ is a $\sigma$-algebra iff 
    $\mathcal{A}$ is closed under countable increasing unions.
\end{homeworkProblem}
\begin{proof}
    $"\Rightarrow"$: Just by the definition of $\sigma$-algebra.

    $"\Leftarrow"$:
    $\mathcal{A}$ is an algebra, 
    so we just need to check its countable unions. 

    Choose a series of non-empty 
    sets $\{\mathcal{S}_{i}\}_{i=1}^{\infty}\subset{\mathcal{M}}$, 
    mark $\mathcal{F}_{i}:=\cup_{j=1}^{i}\mathcal{S}_{j}$, 
    it's clear that $\mathcal{F}_{1}\subset\mathcal{F}_{2}\subset\cdots$.
    And by the definition of algebra, $\mathcal{F}_{i}\in\mathcal{A}$.
    As $\mathcal{A}$ is closed under countable increasing unions, 
    \begin{displaymath}
        \cup_{i=1}^{\infty}\mathcal{S}_{i}=
        \cup_{i=1}^{\infty}\mathcal{F}_{i}\in\mathcal{A}. 
    \end{displaymath}
    So $\mathcal{A}$ is closed under countable unions, 
    i.e. $\mathcal{A}$ is a $\sigma$-algebra.
\end{proof}
\begin{homeworkProblem}
    If $\mathcal{M}$ is the $\sigma$-algebra generated by $\mathcal{E}$, 
    then $\mathcal{M}$ is the union of the 
    $\sigma$-algebras generated by $\mathcal{F}$ 
    as $\mathcal{F}$ ranges over all 
    countable subsets of $\mathcal{E}$.
\end{homeworkProblem}
\begin{proof}
    We complete this proof by two steps. Mark
    \begin{displaymath}
        \tilde{\mathcal{M}}:=     
        \cup_{\mathcal{F}\subset\mathcal{E},|\mathcal{F}|=\aleph_{0}}
        \mathcal{M}(\mathcal{F}).
    \end{displaymath}

    First, we show that 
    \begin{equation}
        \label{eq:Left}
        \mathcal{M}\subset\tilde{\mathcal{M}}.
    \end{equation}
    If $\mathcal{S}\in\mathcal{M}$, 
    i.e. $\exists\{\mathcal{S}_{i}\}_{i=1}^{\infty}\subset\mathcal{E}$ 
    s.t. $\mathcal{S}=\cup_{i=1}^{\infty}\mathcal{S}_{i}$, 
    choose $\mathcal{F}=\{\mathcal{S}_{i}\}_{i=1}^{\infty}$, 
    we can see $\mathcal{S}\in\mathcal{M}(\mathcal{F})$.
    So \eqref{eq:Left} is true.

    Second, we show $\tilde{\mathcal{M}}$ 
    is a $\sigma$-algebra. 
    If so, as $\forall\mathcal{F}\subset\mathcal{E}$, 
    $\mathcal{M}(\mathcal{F})\subset\mathcal{M}$, 
    \begin{displaymath}
        \tilde{\mathcal{M}}\subset\mathcal{M}.        
    \end{displaymath}
    For the countable unions, 
    we choose 
    $\{\mathcal{S}_{i}\}_{i=1}^{\infty}\subset\tilde{\mathcal{M}}$, 
    i.e. $\exists$ countable sets $\mathcal{F}_{i}\subset\mathcal{E}$  
    and $\{\tilde{\mathcal{S}}_{ij}\}_{j=1}^{\infty}
    \subset\mathcal{F}_{i}$ such that 
    $\mathcal{S}_{i}=\cup_{j=1}^{\infty}\tilde{\mathcal{S}}_{ij}$.
    It means 
    \begin{displaymath}
        \cup_{i=1}^{\infty}\mathcal{S}_{i}
        =\cup_{i=1}^{\infty}\cup_{j=1}^{\infty}
        \tilde{\mathcal{S}}_{ij}.
    \end{displaymath}
    As $\tilde{\mathcal{S}}_{ij}\in\mathcal{E}$, 
    $\cup_{i=1}^{\infty}\mathcal{S}_{i}$ 
    is a countable union of elements in $\mathcal{E}$. 
    So $\cup_{i=1}^{\infty}\mathcal{S}_{i}\in\tilde{\mathcal{M}}$.

    On the other hand, 
    $\tilde{\mathcal{M}}$ is closed under complement 
    as $\forall\mathcal{F}\subset\mathcal{E}$, 
    $\mathcal{M}(\mathcal{F})$ is closed under complement.
    So $\mathcal{M}=\tilde{\mathcal{M}}$.\qedhere
\end{proof}
\begin{homeworkProblem}
    If $\mu_{1},\ldots,\mu_{n}$ are measures on $(X,\mathcal{M})$ 
    and $a_{1},\ldots,a_{n}\in[0,\infty)$, 
    then $\sum_{1}^{n}a_{j}\mu_{j}$ is a measure on $(X,\mathcal{M})$.
\end{homeworkProblem}
\begin{proof}
    First, choose $\mathcal{S}\in\mathcal{M}$, 
    as $a_{j}\ge 0$ and $\mu_{j}(\mathcal{S})\ge 0$, 
    $\sum_{1}^{n}a_{j}\mu_{j}(\mathcal{S})\ge 0$, 
    i.e. $\sum_{1}^{n}a_{j}\mu_{j}$ is a map from $\mathcal{M}$ to 
    $[0,\infty]$.

    Then, as $\forall j$, $\mu_{j}(\emptyset)=0$, 
    $\sum_{1}^{n}a_{j}\mu_{j}(\emptyset)=0$.

    Finally, choose a sequence of disjoint sets 
    $\{E_{j}\}_{j=1}^{\infty}\subset\mathcal{M}$, 
    \begin{equation}
        \label{eq:propMeasure}
        \sum_{1}^{n}a_{j}\mu_{j}(\cup_{k=1}^{\infty}E_{k})
        =\sum_{j=1}^{n}\sum_{k=1}^{\infty}a_{j}\mu_{j}(E_{k}).
    \end{equation}
    We claim that 
    \begin{equation}
        \label{eq:propCommute}
        \sum_{j=1}^{n}\sum_{k=1}^{\infty}a_{j}\mu_{j}(E_{k})
        =\sum_{k=1}^{\infty}\sum_{j=1}^{n}a_{j}\mu_{j}(E_{k}).
    \end{equation}
    If $\text{LHS}=\infty$, i.e. $\exists j_{0}\le n$, 
    $\sum_{k=1}^{\infty}a_{j_0}\mu_{j_0}(E_{k})=\infty$, 
    \begin{displaymath}
        \text{RHS}\ge\sum_{k=1}^{\infty}a_{j_{0}}\mu_{j_{0}}(E_{k})
        =\infty.
    \end{displaymath}
    If $\text{LHS}<\infty$, 
    i.e. $\forall j\le n$, $\sum_{k=1}^{\infty}a_{j}\mu_{j}(E_{k})$ 
    is convergent, 
    as $n<\infty$, $\exists M>0$ 
    such that $\forall j\le n$, 
    $\sum_{k=1}^{n}a_{j}\mu_{j}(E_{k})\le M$. 
    By dominant convergent theorem(DCT), \eqref{eq:propCommute} is true. 
    From \eqref{eq:propCommute}, \eqref{eq:propMeasure} and the first two steps, 
    $\sum_{1}^{n}a_{j}\mu_{j}$ is a measure.
\end{proof}
\end{document}