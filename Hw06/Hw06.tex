\documentclass{article}

\usepackage{fancyhdr}
\usepackage{extramarks}
\usepackage{amsmath}
\usepackage{amsthm}
\newtheorem{lemma}{Lemma}
\usepackage{amsfonts}
\usepackage{tikz}
\usepackage[plain]{algorithm}
\usepackage{algpseudocode}

\usetikzlibrary{automata,positioning}

%
% Basic Document Settings
%

\topmargin=-0.45in
\evensidemargin=0in
\oddsidemargin=0in
\textwidth=6.5in
\textheight=9.0in
\headsep=0.25in

\linespread{1.1}

\pagestyle{fancy}
\lhead{\hmwkAuthorName}
\chead{\hmwkClass\ (\hmwkClassInstructor\ \hmwkClassTime): \hmwkTitle}
\rhead{\firstxmark}
\lfoot{\lastxmark}
\cfoot{\thepage}

\renewcommand\headrulewidth{0.4pt}
\renewcommand\footrulewidth{0.4pt}

\setlength\parindent{0pt}

%
% Create Problem Sections
%

\newcommand{\enterProblemHeader}[1]{
    \nobreak\extramarks{}{Problem \arabic{#1} continued on next page\ldots}\nobreak{}
    \nobreak\extramarks{Problem \arabic{#1} (continued)}{Problem \arabic{#1} continued on next page\ldots}\nobreak{}
}

\newcommand{\exitProblemHeader}[1]{
    \nobreak\extramarks{Problem \arabic{#1} (continued)}{Problem \arabic{#1} continued on next page\ldots}\nobreak{}
    \stepcounter{#1}
    \nobreak\extramarks{Problem \arabic{#1}}{}\nobreak{}
}

\setcounter{secnumdepth}{0}
\newcounter{partCounter}
\newcounter{homeworkProblemCounter}
\setcounter{homeworkProblemCounter}{1}
\nobreak\extramarks{Problem \arabic{homeworkProblemCounter}}{}\nobreak{}

%
% Homework Problem Environment
%
% This environment takes an optional argument. When given, it will adjust the
% problem counter. This is useful for when the problems given for your
% assignment aren't sequential. See the last 3 problems of this template for an
% example.
%
\newenvironment{homeworkProblem}[1][-1]{
    \ifnum#1>0
        \setcounter{homeworkProblemCounter}{#1}
    \fi
    \section{Problem \arabic{homeworkProblemCounter}}
    \setcounter{partCounter}{1}
    \enterProblemHeader{homeworkProblemCounter}
}{
    \exitProblemHeader{homeworkProblemCounter}
}

%
% Homework Details
%   - Title
%   - Due date
%   - Class
%   - Section/Time
%   - Instructor
%   - Author
%

\newcommand{\hmwkTitle}{Homework\ \#6}
\newcommand{\hmwkDueDate}{Apr 9, 2024}
\newcommand{\hmwkClass}{Real Analysis}
\newcommand{\hmwkClassTime}{Tuesday}
\newcommand{\hmwkClassInstructor}{Professor Yakun Xi}
\newcommand{\hmwkAuthorName}{\textbf{Shuang Hu}}

%
% Title Page
%

\title{
    \vspace{2in}
    \textmd{\textbf{\hmwkClass:\ \hmwkTitle}}\\
    \normalsize\vspace{0.1in}\small{Due\ on\ \hmwkDueDate\ at 10:00am}\\
    \vspace{0.1in}\large{\textit{\hmwkClassInstructor\ \hmwkClassTime}}
    \vspace{3in}
}

\author{\hmwkAuthorName}
\date{}

\renewcommand{\part}[1]{\textbf{\large Part \Alph{partCounter}}\stepcounter{partCounter}\\}

%
% Various Helper Commands
%

% Useful for algorithms
\newcommand{\alg}[1]{\textsc{\bfseries \footnotesize #1}}

% For derivatives
\newcommand{\deriv}[1]{\frac{\mathrm{d}}{\mathrm{d}x} (#1)}

% For partial derivatives
\newcommand{\pderiv}[2]{\frac{\partial}{\partial #1} (#2)}

% Integral dx
\newcommand{\dx}{\mathrm{d}x}

% Alias for the Solution section header
\newcommand{\solution}{\textbf{\large Solution}}
\newcommand{\norm}[1]{\|#1\|}
% Probability commands: Expectation, Variance, Covariance, Bias
\newcommand{\Var}{\mathrm{Var}}
\newcommand{\Cov}{\mathrm{Cov}}
\newcommand{\Bias}{\mathrm{Bias}}
\newcommand{\supp}{\text{supp}}
\newcommand{\Rn}{\mathbb{R}^{n}}
\newcommand{\dif}{\mathrm{d}}
\newcommand{\avg}[1]{\left\langle #1 \right\rangle}
\newcommand{\difFrac}[2]{\frac{\dif #1}{\dif #2}}
\newcommand{\pdfFrac}[2]{\frac{\partial #1}{\partial #2}}
\newcommand{\OFL}{\mathrm{OFL}}
\newcommand{\UFL}{\mathrm{UFL}}
\newcommand{\fl}{\mathrm{fl}}
\newcommand{\Eabs}{E_{\mathrm{abs}}}
\newcommand{\Erel}{E_{\mathrm{rel}}}
\newcommand{\DR}{\mathcal{D}_{\widetilde{LN}}^{n}}
\newcommand{\add}[2]{\sum_{#1=1}^{#2}}
\newcommand{\innerprod}[2]{\left<#1,#2\right>}
\newcommand{\Sc}{\mathcal{S}}
\newcommand{\F}{\mathcal{F}}
\newcommand{\E}{\mathcal{E}}
\newcommand{\A}{\mathcal{A}}
\newcommand{\cp}[2]{\cup_{#1=1}^{#2}}
\newcommand{\sm}[2]{\sum_{#1=1}^{#2}}
\newcommand{\M}{\mathcal{M}}
\newcommand{\Lc}{\mathcal{L}}
\newcommand\tbbint{{-\mkern -16mu\int}}
\newcommand\tbint{{\mathchar '26\mkern -14mu\int}}
\newcommand\dbbint{{-\mkern -19mu\int}}
\newcommand\dbint{{\mathchar '26\mkern -18mu\int}}
\newcommand\bint{
{\mathchoice{\dbint}{\tbint}{\tbint}{\tbint}}
}
\newcommand\bbint{
{\mathchoice{\dbbint}{\tbbint}{\tbbint}{\tbbint}}
}
\begin{document}
\maketitle
\pagebreak
\begin{homeworkProblem}
    Suppose $\{f_{n}\}\subset L^1(\mu)$ and $f_{n}\rightarrow f$ 
    uniformly.
    \begin{enumerate}
        \item If $\mu(X)<\infty$, then $f\in L^{1}(\mu)$ and 
        $\int f_n\rightarrow\int f$.
        \item If $\mu(X)=\infty$, the conclusions of $(1)$ can fail.
    \end{enumerate}
\end{homeworkProblem}
\begin{proof}
    $(1)$ By Cauchy Convergence Theorem, $f_{n}\rightarrow f$ 
    uniformly means
    \begin{displaymath}
        \forall\epsilon>0,\;\exists N_{0},\;\forall n,m>N_{0},\;
        \forall x\in X,|f_{n}(x)-f_{m}(x)|<\epsilon.
    \end{displaymath}
    Choose $g(x):=f_{N_0}(x)+\epsilon\text{sgn}(f_{N_0}(x))$, 
    then $\forall n\ge N_0$, $|f_{n}(x)|\le |g(x)|$, and 
    \begin{displaymath}
        \int_{X}|g|\le\int_{X}|f_{N_0}|+\epsilon\mu(X)
        <\infty.
    \end{displaymath}
    By DCT, 
    $$
    \lim_{n\rightarrow\infty}\int |f_{n}|=
    \int\lim_{n\rightarrow\infty}|f_{n}|
    =\int|f|<\infty,
    $$
    so $f\in L^{1}(\mu)$ and $\int f_{n}\rightarrow\int f$.

    $(2)$ Choose 
    \begin{displaymath}
        f_{n}(x)=\left\{
            \begin{aligned}
                &1 &0\le x\le 1,\\
                &\frac{1}{x^{1+\frac{1}{n}}}&x>1,
            \end{aligned}
            \right.
        g(x)=\left\{
            \begin{aligned}
                &1 &0\le x\le 1,\\
                &\frac{1}{x}&x>1,
            \end{aligned}
            \right.
    \end{displaymath}
    then $\forall n$, $f_{n}\in L^{1}(\mu)$, 
    and $f_{n}\rightarrow g$ uniformly, but $g\notin L^{1}(\mu)$.
\end{proof}
\begin{homeworkProblem}
    (Generalized DCT) If $f_{n},g_{n},f,g\in L^1$, 
    $f_{n}\rightarrow f$ and 
    $g_n\rightarrow g$ a.e., $|f_n|\le g_n$, and 
    $\int g_{n}\rightarrow\int g$, then $\int f_n\rightarrow\int f$.
\end{homeworkProblem}
\begin{proof}
    $|f_{n}|\le g_{n}$ means $f_{n}+g_n\ge 0$. On one hand, 
    \begin{displaymath}
        \int g+\int f=\int (g+f)=\int \lim_{n\rightarrow\infty}
        (g_n+f_n)\le\liminf_{n\rightarrow\infty}(\int g_n+\int f_n)
        =\int g+\liminf_{n\rightarrow\infty}\int f_{n},
    \end{displaymath}
    on the other hand:
    \begin{displaymath}
        \int g-\int f=\int (g-f)=\int\lim_{n\rightarrow\infty}(g_n-f_n)
        \le\liminf_{n\rightarrow\infty}(\int g_n-\int f_n)
        =\int g-\limsup_{n\rightarrow\infty}\int f_n.
    \end{displaymath}
    Then 
    \begin{displaymath}
        \limsup_{n\rightarrow\infty}f_n\le\int f\le
        \liminf_{n\rightarrow\infty}f_{n}
        \Rightarrow\lim_{n\rightarrow\infty}\int f_{n}=\int f.
    \end{displaymath}
\end{proof}
\begin{homeworkProblem}
    Suppose $f_{n},f\in L^{1}$ and $f_n\rightarrow f$ a.e. 
    Then $\int|f_n-f|\rightarrow 0$ iff $\int|f_n|\rightarrow
    \int |f|$.
\end{homeworkProblem}
\begin{proof}
    $"\Rightarrow"$: By triangular inequality, 
    $|f|-|f_n|\le g_{n}:=|f-f_{n}|$. Since $g_{n}\rightarrow 0$ a.e. 
    and $\lim_{n}\rightarrow\infty g_{n}=0$, 
    by Exercise 20, 
    \begin{displaymath}
        \lim_{n\rightarrow\infty}\int(|f|-|f_n|)
        =\int\lim_{n\rightarrow\infty}(|f|-|f_n|)
        =0.
    \end{displaymath}
    So $\int|f_{n}|\rightarrow\int |f|$.

    $"\Leftarrow"$: Mark $g_{n}:=|f_{n}|+|f|$, since 
    $g_{n}\rightarrow 2|f|$ a.e., $\lim_{n\rightarrow\infty}
    \int g_{n}=2\int |f|$ 
    and $|f_n-f|\le g_{n}$, 
    by Exercise 20, 
    \begin{displaymath}
        \lim_{n\rightarrow\infty}\int(|f-f_n|)
        =\int\lim_{n\rightarrow\infty}(|f-f_n|)
        =0.
    \end{displaymath}
\end{proof}
\end{document}