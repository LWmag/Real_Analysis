\documentclass{article}

\usepackage{fancyhdr}
\usepackage{extramarks}
\usepackage{amsmath}
\usepackage{amsthm}
\newtheorem{lemma}{Lemma}
\usepackage{amsfonts}
\usepackage{tikz}
\usepackage[plain]{algorithm}
\usepackage{algpseudocode}

\usetikzlibrary{automata,positioning}

%
% Basic Document Settings
%

\topmargin=-0.45in
\evensidemargin=0in
\oddsidemargin=0in
\textwidth=6.5in
\textheight=9.0in
\headsep=0.25in

\linespread{1.1}

\pagestyle{fancy}
\lhead{\hmwkAuthorName}
\chead{\hmwkClass\ (\hmwkClassInstructor\ \hmwkClassTime): \hmwkTitle}
\rhead{\firstxmark}
\lfoot{\lastxmark}
\cfoot{\thepage}

\renewcommand\headrulewidth{0.4pt}
\renewcommand\footrulewidth{0.4pt}

\setlength\parindent{0pt}

%
% Create Problem Sections
%

\newcommand{\enterProblemHeader}[1]{
    \nobreak\extramarks{}{Problem \arabic{#1} continued on next page\ldots}\nobreak{}
    \nobreak\extramarks{Problem \arabic{#1} (continued)}{Problem \arabic{#1} continued on next page\ldots}\nobreak{}
}

\newcommand{\exitProblemHeader}[1]{
    \nobreak\extramarks{Problem \arabic{#1} (continued)}{Problem \arabic{#1} continued on next page\ldots}\nobreak{}
    \stepcounter{#1}
    \nobreak\extramarks{Problem \arabic{#1}}{}\nobreak{}
}

\setcounter{secnumdepth}{0}
\newcounter{partCounter}
\newcounter{homeworkProblemCounter}
\setcounter{homeworkProblemCounter}{1}
\nobreak\extramarks{Problem \arabic{homeworkProblemCounter}}{}\nobreak{}

%
% Homework Problem Environment
%
% This environment takes an optional argument. When given, it will adjust the
% problem counter. This is useful for when the problems given for your
% assignment aren't sequential. See the last 3 problems of this template for an
% example.
%
\newenvironment{homeworkProblem}[1][-1]{
    \ifnum#1>0
        \setcounter{homeworkProblemCounter}{#1}
    \fi
    \section{Problem \arabic{homeworkProblemCounter}}
    \setcounter{partCounter}{1}
    \enterProblemHeader{homeworkProblemCounter}
}{
    \exitProblemHeader{homeworkProblemCounter}
}

%
% Homework Details
%   - Title
%   - Due date
%   - Class
%   - Section/Time
%   - Instructor
%   - Author
%

\newcommand{\hmwkTitle}{Homework\ \#5}
\newcommand{\hmwkDueDate}{Apr 2, 2024}
\newcommand{\hmwkClass}{Real Analysis}
\newcommand{\hmwkClassTime}{Tuesday}
\newcommand{\hmwkClassInstructor}{Professor Yakun Xi}
\newcommand{\hmwkAuthorName}{\textbf{Shuang Hu}}

%
% Title Page
%

\title{
    \vspace{2in}
    \textmd{\textbf{\hmwkClass:\ \hmwkTitle}}\\
    \normalsize\vspace{0.1in}\small{Due\ on\ \hmwkDueDate\ at 10:00am}\\
    \vspace{0.1in}\large{\textit{\hmwkClassInstructor\ \hmwkClassTime}}
    \vspace{3in}
}

\author{\hmwkAuthorName}
\date{}

\renewcommand{\part}[1]{\textbf{\large Part \Alph{partCounter}}\stepcounter{partCounter}\\}

%
% Various Helper Commands
%

% Useful for algorithms
\newcommand{\alg}[1]{\textsc{\bfseries \footnotesize #1}}

% For derivatives
\newcommand{\deriv}[1]{\frac{\mathrm{d}}{\mathrm{d}x} (#1)}

% For partial derivatives
\newcommand{\pderiv}[2]{\frac{\partial}{\partial #1} (#2)}

% Integral dx
\newcommand{\dx}{\mathrm{d}x}

% Alias for the Solution section header
\newcommand{\solution}{\textbf{\large Solution}}
\newcommand{\norm}[1]{\|#1\|}
% Probability commands: Expectation, Variance, Covariance, Bias
\newcommand{\Var}{\mathrm{Var}}
\newcommand{\Cov}{\mathrm{Cov}}
\newcommand{\Bias}{\mathrm{Bias}}
\newcommand{\supp}{\text{supp}}
\newcommand{\Rn}{\mathbb{R}^{n}}
\newcommand{\dif}{\mathrm{d}}
\newcommand{\avg}[1]{\left\langle #1 \right\rangle}
\newcommand{\difFrac}[2]{\frac{\dif #1}{\dif #2}}
\newcommand{\pdfFrac}[2]{\frac{\partial #1}{\partial #2}}
\newcommand{\OFL}{\mathrm{OFL}}
\newcommand{\UFL}{\mathrm{UFL}}
\newcommand{\fl}{\mathrm{fl}}
\newcommand{\Eabs}{E_{\mathrm{abs}}}
\newcommand{\Erel}{E_{\mathrm{rel}}}
\newcommand{\DR}{\mathcal{D}_{\widetilde{LN}}^{n}}
\newcommand{\add}[2]{\sum_{#1=1}^{#2}}
\newcommand{\innerprod}[2]{\left<#1,#2\right>}
\newcommand{\Sc}{\mathcal{S}}
\newcommand{\F}{\mathcal{F}}
\newcommand{\E}{\mathcal{E}}
\newcommand{\A}{\mathcal{A}}
\newcommand{\cp}[2]{\cup_{#1=1}^{#2}}
\newcommand{\sm}[2]{\sum_{#1=1}^{#2}}
\newcommand{\M}{\mathcal{M}}
\newcommand{\Lc}{\mathcal{L}}
\newcommand\tbbint{{-\mkern -16mu\int}}
\newcommand\tbint{{\mathchar '26\mkern -14mu\int}}
\newcommand\dbbint{{-\mkern -19mu\int}}
\newcommand\dbint{{\mathchar '26\mkern -18mu\int}}
\newcommand\bint{
{\mathchoice{\dbint}{\tbint}{\tbint}{\tbint}}
}
\newcommand\bbint{
{\mathchoice{\dbbint}{\tbbint}{\tbbint}{\tbbint}}
}
\begin{document}
\maketitle
\pagebreak
\begin{homeworkProblem}
    Prove Proposition 2.11.
\end{homeworkProblem}
\begin{proof}
    WLOG, assume $f:X\rightarrow\mathbb{C}$.

    $(a)$ $"\Rightarrow"$:
    $f$ measurable means 
    $\forall$ Borel set $B\subset\mathbb{C}$, 
    $f^{-1}(B)$ is measurable. 
    Assume $E=\{x:f(x)\neq g(x)\}$, as $f=g$ a.e., $\mu(E)=0$. 
    Choose $h:=f-g$, then $\forall$ Borel set $B$, 
    $h^{-1}(B)\subset E$.

    As $\mu$ is complete, $\mu(h^{-1}(B))=0$. 
    So $h$ is measurable. It means that 
    $g=f-h$ is measurable.

    $"\Leftarrow"$: Choose $E$ with $\mu(E)=0$, 
    and set $\Sc\subset E$, it suffices to show $\mu(\Sc)=0$. 
    Set $f=\chi_{E}$, $g=2\chi_{\Sc}$, since $\{f\neq g\}=E$ and 
    $\mu(E)=0$, we can see $f=g$ a.e. 
    Then $g$ is measurable, so $\mu(\Sc)=0$. 
    Now we can see $\mu$ is complete.

    $(b)$ $"\Rightarrow":$ 
    Choose Borel set $B$, assume $f_{n}\rightarrow f$ on 
    $\M\setminus E$ with $\mu(E)=0$, since $\mu$ is complete, 
    $\mu(E\cap f^{-1}(B))=0$. What's more:
    \begin{displaymath}
        E^{c}\cap f^{-1}(B)=(\cap_{n=1}^{\infty}\cup_{k=n}^{\infty}
        f_{k}^{-1}(B))\cap E^{c}
        =\cap_{n=1}^{\infty}\cup_{k=n}^{\infty}
        (f_{k}^{-1}(B)\cap E)
    \end{displaymath} 
    is measurable. 
    So $f^{-1}(B)=(f^{-1}(B)\cap E)\cup(f^{-1}(B)\cap E^{c})$ 
    is measurable.

    $"\Leftarrow":$ Set $\mu(E)=0$, $N\subset E$. 
    Choose $f(x)=\chi_{N}$, $f_{n}(x)=2\chi_{E}$, 
    it's clear that $f_{n}\rightarrow f$ $\mu$ a.e. 
    So $f$ is measurable, which means $N$ is measurable.
\end{proof}
\begin{homeworkProblem}
    Prove Proposition 2.20.
\end{homeworkProblem}
\begin{proof}
    Choose $f_{n}(x):=\min\{n,f(x)\}$, 
    then $f_{n}(x)$ is measurable, and 
    \begin{displaymath}
        \int f\ge\int f_{n}\ge n\mu(\{x:f(x)\ge n\}).
    \end{displaymath}
    Since $\int f<\infty$, 
    \begin{displaymath}
        \limsup_{n\rightarrow\infty}n\mu(\{x:f(x)\ge n\})
        \le\int f<\infty
        \Rightarrow \lim_{n\rightarrow\infty}\mu(\{x:f(x)\ge n\})=0
        \Rightarrow \mu(\{x:f(x)=\infty\})=0.
    \end{displaymath}
    Set $E_{n}:=\{x:n<f(x)\le n+1\}$, 
    $F_{n}:=\{x:\frac{1}{n+1}<f(x)\le\frac{1}{n}\}$, 
    since $f\in L^{+}$, $E_{n},F_{n}$ are measurable, and 
    \begin{displaymath}
        \{x:f(x)>0\}=\cup_{n=1}^{\infty}(E_{n}\cup F_n).
    \end{displaymath}
    It suffices to show $\mu(E_n)<\infty$, $\mu(F_n)<\infty$.
    On one hand:
    \begin{displaymath}
        n\mu(E_{n})<\int_{E_n}f\le\int f<\infty
        \Rightarrow \mu(E_n)<\frac{\int f}{n}<\infty,
    \end{displaymath}
    on the other hand:
    \begin{displaymath}
        \frac{1}{n+1}\mu(F_n)<\int_{F_n} f
        \le\int f<\infty\Rightarrow \mu(F_{n})<(n+1)\int f<\infty. 
    \end{displaymath}
    So $\{x:f(x)>0\}$ is $\sigma$-finite.
\end{proof}
\begin{homeworkProblem}
    Suppose $\{f_{n}\}\subset L^{+}$, 
    $f_{n}\rightarrow f$ pointwise, and 
    $\int f=\lim\int f_{n}<\infty.$ Then, 
    $\int_{E}f=\lim\int_{E}f_{n}$ for all 
    $E\in\M$. However this need not be true if 
    $\int f=\lim\int f_n=\infty$.
\end{homeworkProblem}
\begin{proof}
    Since $f_{n}\rightarrow f$ pointwise, by Fatou's Lemma:
    \begin{equation}
        \label{eq:Fatou1}
        \int_{E}f=\int_{E}\liminf_{n\rightarrow\infty}f_{n}
        \le\liminf_{n\rightarrow\infty}\int_{E}f_n,
    \end{equation}
    since $\int f<\infty$, by Fatou's Lemma:
    \begin{equation}
        \label{eq:Fatou2}
        \int_{E^{c}}f=\int_{E^c}\liminf_{n\rightarrow\infty}f_n
        \le\liminf_{n\rightarrow\infty}\int_{E^c}f_n
        \Rightarrow \int f-\int_{E}f\le\liminf_{n\rightarrow\infty}
        \int_{E^c}f_n.
    \end{equation}
    By \eqref{eq:Fatou1}, \eqref{eq:Fatou2} 
    and $\int f=\lim\int f_{n}$, 
    \begin{displaymath}
        \begin{array}{rl}
            \int f-\int_{E}f&\le\liminf_{n\rightarrow\infty}
            (\int f_{n}-\int_{E}f_{n})\\
            &=\int f-\limsup_{n\rightarrow\infty}\int_{E}f_{n}\\
            &\Rightarrow \limsup_{n\rightarrow\infty}\int_{E}f_{n}
            \le\int_{E}f\le\liminf_{n\rightarrow\infty}\int_{E}f_{n}.
        \end{array}
    \end{displaymath}
    So $\lim_{n\rightarrow\infty}\int_{E}f_{n}=\int_{E}f$.

    If $\int f=\infty$, choose 
    $f_{n}(x)=\chi_{(-\infty,0)}+(n-n^{2}x)\chi_{[0,\frac{1}{n})}$,
    then $f_{n}(x)\rightarrow f:=\chi_{(\infty,0)}+\infty\chi_{\{0\}}$ 
    pointwise, and $\lim_{n\rightarrow\infty}\int f_{n}(x)=\infty$.

    But if we choose $E=[0,1]$, $\int_{E}f_{n}(x)\rightarrow 0.5$ 
    while $\int_{E}f=0$, contradict.
\end{proof}
\begin{homeworkProblem}
    If $f\in L^{+}$, let $\lambda(E)=\int_{E}f\dif \mu$ 
    for $E\in\M$. Then $\lambda$ is a measure on $\M$, 
    and for any $g\in L^{+}$, $\int g\dif \lambda=\int fg\dif\mu$.
\end{homeworkProblem}
\begin{proof}
    Choose disjoint sets $\{E_{i}\}_{i=1}^{\infty}\subset\M$, 
    mark $E:=\cp{n}{\infty}E_{n}$, $f_{n}:=f\chi_{E_{n}}$, 
    $f_{0}:=f\chi_{E}$, then $f_{0}=\sm{n}{\infty}f_{n}$.
    By Theorem 2.15, 
    \begin{displaymath}
        \lambda(E)=\int_{E}f=\int \sm{n}{\infty}f_n
        =\sm{n}{\infty}\int f_n=\sm{n}{\infty}\lambda(E_n).
    \end{displaymath}
    It means that $\lambda$ is a measure. 
    For a simple function $g=\sum a_i\chi_{F_i}$, 
    \begin{displaymath}
        \begin{array}{l}
            \int g\dif\lambda=\sum a_i\lambda(F_i)=\sum a_i\int_{F_i}f\dif \mu\\
            \int fg\dif\mu=\sum_{i}\int_{F_i}a_i f\dif\mu=\sum a_{i}\int_{F_{i}}f\dif\mu.
        \end{array}
    \end{displaymath}
    So $\int g\dif\lambda=\int fg\dif\mu$ holds for any simple functions $g$.
    If $g\in L^{+}$, 
    $\exists$ a sequence of simple functions $\{\phi_{n}\}$ satisfies:
    \begin{itemize}
        \item $\phi_{n}\le g$,
        \item $\lim_{n\rightarrow\infty}\int\phi_{n}\dif\lambda=\int g\dif\lambda$.
    \end{itemize}
    It means
    \begin{displaymath} 
    \int g\dif\lambda=\lim_{n\rightarrow\infty}\int \phi_{n}\dif\lambda
    =\lim_{n\rightarrow\infty}\int f\phi_{n}\dif\mu.
    \end{displaymath}
    By MCT:
    \begin{displaymath}
        \lim_{n\rightarrow\infty}\int f\phi_{n}\dif\mu
        =\int\lim_{n\rightarrow\infty}f\phi_{n}\dif\mu
        =\int fg\dif\mu.
    \end{displaymath}
    So $\int g\dif\lambda=\int fg\dif\mu$.
\end{proof}
\begin{homeworkProblem}
    If $f\in L^{+}$ and $\int f<\infty$, for every $\epsilon>0$ 
    there exists $E\in\M$ such that $\mu(E)<\infty$ and $\int_{E}f>(\int f)-\epsilon$.
\end{homeworkProblem}
\begin{proof}
    Choose $E_{n}:=\{x:f(x)\ge\frac{1}{n}\}$, since $\int f<\infty$, 
    it means 
    $\forall n$, $\frac{1}{n}\mu(E_{n})<\infty$, 
    so $\forall n\in\mathbb{N}$, $\mu(E_{n})<\infty$.
    
    What's more, $f_{n}:=\chi_{E_{n}}$ satisfies:
    \begin{itemize}
        \item $f_{n}$ increasing with $n$,
        \item $\lim_{n\rightarrow\infty}f_{n}=f$.
    \end{itemize}
    Then by MCT, $\lim_{n\rightarrow\infty}\int_{E_n}f=\int f$. 
    It means: 
    \begin{displaymath}
        \forall\epsilon>0,\;\exists N,\;\forall n>N\quad\int_{E_n}f>(\int f) -\epsilon.   
    \end{displaymath}
    Just choose $n_{0}>N$, $E_{n_{0}}\in\M$ is the set.
\end{proof}
\begin{homeworkProblem}
    Assume Fatou's Lemma and deduce the MCT from it.
\end{homeworkProblem}
\begin{proof}
    Assume $\{f_{n}\}$ is increasing and converges to $f$ pointwise, 
    then $\forall n\in\mathbb{N}$, $\int f_{n}\le\int f$. 
    Moreover, by Fatou's Lemma:
    \begin{displaymath}
        \int f=\int\liminf_{n\rightarrow\infty}f_n
        \le\liminf_{n\rightarrow\infty}\int f_{n}
        =\lim_{n\rightarrow\infty}\int f_{n},
    \end{displaymath}
    the final step holds because $f_{n}$ increasing, 
    via $\int f_{n}$ increasing, so $\int f_{n}$ converges.

    From the above two inequalities, $\int f=\lim_{n\rightarrow\infty}\int f_n$. 
    So MCT holds.
\end{proof}
\end{document}