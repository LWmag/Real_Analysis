\documentclass{article}

\usepackage{fancyhdr}
\usepackage{extramarks}
\usepackage{amsmath}
\usepackage{amsthm}
\newtheorem{lemma}{Lemma}
\usepackage{amsfonts}
\usepackage{tikz}
\usepackage[plain]{algorithm}
\usepackage{algpseudocode}

\usetikzlibrary{automata,positioning}

%
% Basic Document Settings
%

\topmargin=-0.45in
\evensidemargin=0in
\oddsidemargin=0in
\textwidth=6.5in
\textheight=9.0in
\headsep=0.25in

\linespread{1.1}

\pagestyle{fancy}
\lhead{\hmwkAuthorName}
\chead{\hmwkClass\ (\hmwkClassInstructor\ \hmwkClassTime): \hmwkTitle}
\rhead{\firstxmark}
\lfoot{\lastxmark}
\cfoot{\thepage}

\renewcommand\headrulewidth{0.4pt}
\renewcommand\footrulewidth{0.4pt}

\setlength\parindent{0pt}

%
% Create Problem Sections
%

\newcommand{\enterProblemHeader}[1]{
    \nobreak\extramarks{}{Problem \arabic{#1} continued on next page\ldots}\nobreak{}
    \nobreak\extramarks{Problem \arabic{#1} (continued)}{Problem \arabic{#1} continued on next page\ldots}\nobreak{}
}

\newcommand{\exitProblemHeader}[1]{
    \nobreak\extramarks{Problem \arabic{#1} (continued)}{Problem \arabic{#1} continued on next page\ldots}\nobreak{}
    \stepcounter{#1}
    \nobreak\extramarks{Problem \arabic{#1}}{}\nobreak{}
}

\setcounter{secnumdepth}{0}
\newcounter{partCounter}
\newcounter{homeworkProblemCounter}
\setcounter{homeworkProblemCounter}{1}
\nobreak\extramarks{Problem \arabic{homeworkProblemCounter}}{}\nobreak{}

%
% Homework Problem Environment
%
% This environment takes an optional argument. When given, it will adjust the
% problem counter. This is useful for when the problems given for your
% assignment aren't sequential. See the last 3 problems of this template for an
% example.
%
\newenvironment{homeworkProblem}[1][-1]{
    \ifnum#1>0
        \setcounter{homeworkProblemCounter}{#1}
    \fi
    \section{Problem \arabic{homeworkProblemCounter}}
    \setcounter{partCounter}{1}
    \enterProblemHeader{homeworkProblemCounter}
}{
    \exitProblemHeader{homeworkProblemCounter}
}

%
% Homework Details
%   - Title
%   - Due date
%   - Class
%   - Section/Time
%   - Instructor
%   - Author
%

\newcommand{\hmwkTitle}{Homework\ \#7}
\newcommand{\hmwkDueDate}{Apr 16, 2024}
\newcommand{\hmwkClass}{Real Analysis}
\newcommand{\hmwkClassTime}{Tuesday}
\newcommand{\hmwkClassInstructor}{Professor Yakun Xi}
\newcommand{\hmwkAuthorName}{\textbf{Shuang Hu}}

%
% Title Page
%

\title{
    \vspace{2in}
    \textmd{\textbf{\hmwkClass:\ \hmwkTitle}}\\
    \normalsize\vspace{0.1in}\small{Due\ on\ \hmwkDueDate\ at 10:00am}\\
    \vspace{0.1in}\large{\textit{\hmwkClassInstructor\ \hmwkClassTime}}
    \vspace{3in}
}

\author{\hmwkAuthorName}
\date{}

\renewcommand{\part}[1]{\textbf{\large Part \Alph{partCounter}}\stepcounter{partCounter}\\}

%
% Various Helper Commands
%

% Useful for algorithms
\newcommand{\alg}[1]{\textsc{\bfseries \footnotesize #1}}

% For derivatives
\newcommand{\deriv}[1]{\frac{\mathrm{d}}{\mathrm{d}x} (#1)}

% For partial derivatives
\newcommand{\pderiv}[2]{\frac{\partial}{\partial #1} (#2)}

% Integral dx
\newcommand{\dx}{\mathrm{d}x}

% Alias for the Solution section header
\newcommand{\solution}{\textbf{\large Solution}}
\newcommand{\norm}[1]{\|#1\|}
% Probability commands: Expectation, Variance, Covariance, Bias
\newcommand{\Var}{\mathrm{Var}}
\newcommand{\Cov}{\mathrm{Cov}}
\newcommand{\Bias}{\mathrm{Bias}}
\newcommand{\supp}{\text{supp}}
\newcommand{\Rn}{\mathbb{R}^{n}}
\newcommand{\dif}{\mathrm{d}}
\newcommand{\avg}[1]{\left\langle #1 \right\rangle}
\newcommand{\difFrac}[2]{\frac{\dif #1}{\dif #2}}
\newcommand{\pdfFrac}[2]{\frac{\partial #1}{\partial #2}}
\newcommand{\OFL}{\mathrm{OFL}}
\newcommand{\UFL}{\mathrm{UFL}}
\newcommand{\fl}{\mathrm{fl}}
\newcommand{\Eabs}{E_{\mathrm{abs}}}
\newcommand{\Erel}{E_{\mathrm{rel}}}
\newcommand{\DR}{\mathcal{D}_{\widetilde{LN}}^{n}}
\newcommand{\add}[2]{\sum_{#1=1}^{#2}}
\newcommand{\innerprod}[2]{\left<#1,#2\right>}
\newcommand{\Sc}{\mathcal{S}}
\newcommand{\F}{\mathcal{F}}
\newcommand{\E}{\mathcal{E}}
\newcommand{\A}{\mathcal{A}}
\newcommand{\cp}[2]{\cup_{#1=1}^{#2}}
\newcommand{\sm}[2]{\sum_{#1=1}^{#2}}
\newcommand{\M}{\mathcal{M}}
\newcommand{\Lc}{\mathcal{L}}
\newcommand\tbbint{{-\mkern -16mu\int}}
\newcommand\tbint{{\mathchar '26\mkern -14mu\int}}
\newcommand\dbbint{{-\mkern -19mu\int}}
\newcommand\dbint{{\mathchar '26\mkern -18mu\int}}
\newcommand\bint{
{\mathchoice{\dbint}{\tbint}{\tbint}{\tbint}}
}
\newcommand\bbint{
{\mathchoice{\dbbint}{\tbbint}{\tbbint}{\tbbint}}
}
\begin{document}
\maketitle
\pagebreak
\begin{homeworkProblem}
    If $f_n\ge 0$ and $f_n\rightarrow f$ in measure, then $\int f\le\liminf \int f_{n}$.
\end{homeworkProblem}
\begin{proof}
    By the definition of limes inferior, exists a subsequence $f_{n_{k}}$ 
    s.t. $\int f_{n_{k}}\rightarrow\liminf\int f_n$. Since $f_{n}\rightarrow f$ 
    in measure, $f_{n_k}\rightarrow f$ in measure as well. 
    By Riesz's Theorem, $\exists$ a subsequence of $f_{n_{k}}$, marked as 
    $f_{m}$, s.t. $f_{m}\rightarrow f$ a.e. 

    Finally, by Fatou's Lemma:
    \begin{displaymath}
        \liminf\int f_n=\lim\int f_{n_k}=\lim\int f_{m}
        \ge\int\liminf f_{m}=\int f.
    \end{displaymath}
\end{proof}
\begin{homeworkProblem}
    Suppose $|f_{n}|\le g\in L^1$ and $f_n\rightarrow f$ in measure.
    \begin{enumerate}
        \item $\int f=\lim\int f_n$.
        \item $f_n\rightarrow f$ in $L^1$.
    \end{enumerate} 
\end{homeworkProblem}
\begin{proof}
    If $(2)$ holds, since 
    \begin{displaymath}
        \left|\int(f-f_n)\right|\le\int|f-f_n|,
    \end{displaymath}
    it's clear that 
    \begin{displaymath}
        \lim\left|\int(f-f_n)\right|
        \le\lim\int|f-f_n|=0,
    \end{displaymath}
    so $(2)$ implies $(1)$. We only need to show $(2)$ holds.

    Assume $f_n$ doesn't converge to $f$ in $L^1$, i.e. 
    $\exists \epsilon_0>0$ and a subsequence $\{f_{n_j}\}$ of $\{f_n\}$, such that 
    \begin{displaymath}
        \int|f_{n_j}-f|\ge\epsilon_{0}.
    \end{displaymath}
    Because $f_n\rightarrow f$ in measure, $f_{n_j}\rightarrow f$ in measure as well. 
    By Riesz's Theorem, there exists a subsequence of $\{f_{n_{j}}\}$, 
    marked as $\{f_{m}\}$, such that 
    $f_m\rightarrow f$ a.e. 
    Since $|f_m-f|\le|f_m|+|f|\le 2|g|$ a.e., 
    by DCT, 
    \begin{displaymath}
        \lim\int|f_m-f|=\int\lim|f_m-f|=0.
    \end{displaymath}
    Contradict! So $f_n\rightarrow f$ in $L^1$, $(2)$ holds.
\end{proof}
\begin{homeworkProblem}
    Suppose $f_{n}\rightarrow f$ in measure and $g_{n}\rightarrow g$ 
    in measure.
    \begin{enumerate}
        \item $f_n+g_n\rightarrow f+g$ in measure.
        \item $f_n g_n\rightarrow fg$ in measure if $\mu(X)<\infty$, but 
        not necessarily if $\mu(X)=\infty$.
    \end{enumerate}
\end{homeworkProblem}
\begin{proof}
    $(1)$ $f_{n}\rightarrow f$ in measure means $\lim\mu(\{|f_n-f|>\epsilon\})=0$, 
    $g_{n}\rightarrow g$ in measure means $\lim\mu(\{|g_n-g|>\epsilon\})=0$. 
    Since 
    \begin{displaymath}
        \{|f_{n}+g_n-f-g|> \epsilon\}\subset
        \{|f_{n}-f|>\frac{\epsilon}{2}\}\cup\{|g_n-g|>\frac{\epsilon}{2}\},
    \end{displaymath}
    it's clear that 
    \begin{displaymath}
        \mu(\{|f_n+g_n-f-g|>\epsilon\})
        \le\mu(\{|f_n-f|>\frac{\epsilon}{2}\})
        +\mu(\{|g_n-g|>\frac{\epsilon}{2}\}).
    \end{displaymath}
    Set $n\rightarrow\infty$, $f_{n}+g_n\rightarrow f+g$ in measure.

    $(2)$ If $\mu(X)<\infty$, $f_{n}\rightarrow f$ in measure 
    yields $f<\infty$ a.e. 
    So $\forall\epsilon>0$, $\exists A>0$ such that 
    \begin{displaymath}
        \mu(\{|f|>A\})+\mu(\{|g|>A\})<\epsilon.
    \end{displaymath}
    By triangular inequality:
    \begin{displaymath}
        |f_{n}g_n-fg|\le|f_n-f||g_n|+|f||g_n-g|,
    \end{displaymath}
    so:
    \begin{displaymath}
        \mu(\{|f_ng_n-fg|>2\epsilon\})\le\mu(\{|f_n-f||g_n|>\epsilon\})
        +\mu(\{|f||g_n-g|>\epsilon\}).
    \end{displaymath}
    Choose $B:=\{|f|<A\}\cap\{|g|<A\}$, then $\mu(B^c)<\epsilon$, it means:
    \begin{displaymath}
        \mu(\{|f_ng_n-fg|>2\epsilon\})\le
        \mu(B^{c})+\mu(\{|f_{n}-f||g_n|>\epsilon\}\cap B)
        +\mu(\{|f||g_n-g|>\epsilon\}\cap B).
    \end{displaymath}
    On the one hand, if $|f|<A$, 
    \begin{displaymath}
        \mu(\{|f||g_n-g|>\epsilon\}\cap B)<\mu(\{|g_n-g|
        >\frac{\epsilon}{A}\})
        \rightarrow 0,
    \end{displaymath}
    on the other hand, when $|g|<A$, 
    since $g_n\rightarrow g$ in measure, 
    $\lim\mu(\{|g_n|>2A\})=0$. So 
    \begin{displaymath}
        \mu(\{|f_n-f||g_n|>\epsilon\}\cap B)
        \le\mu(\{|g_n|>2A\})
        +\mu(\{|f_n-f|>\frac{\epsilon}{2A}\})\rightarrow 0.
    \end{displaymath}
    It means 
    \begin{displaymath}
        \lim\mu(\{|f_ng_n-fg|>2\epsilon\})<\epsilon.
    \end{displaymath}
    By Exercise 35, $f_ng_n\rightarrow fg$ in measure.

    If $\mu(X)=\infty$, choose $f_{n}(x)=\frac{1}{n}\chi_{[0,n]}$, 
    $g_{n}(x)\equiv x$, $f_n\rightarrow 0$ in measure and $g_n\rightarrow x$ in measure. 
    But $f_ng_n=\frac{x}{n}\chi_{[0,n]}$ doesn't converge to $0$ by measure. 
    So in this case, $f_ng_n\rightarrow fg$ in measure isn't necessarily true.
\end{proof}
\begin{homeworkProblem}
    In Egoroff's Theorem, the hypothesis $\mu(X)<\infty$ 
    can be replaced by $|f_n|\le g$ for all $n$ where $g\in L^{1}(\mu)$. 
\end{homeworkProblem}
\begin{proof}
    Set $G_{m}:=\{x:g(x)\ge\frac{1}{m}\}$, 
    $g\in L^{1}(\mu)$ means $\forall m\in\mathbb{N}$, $\mu(G_{m})<\infty$. 
    First, choose $Z:=\{x:g(x)=0\}$, it means 
    $\forall x\in Z$, $n\in\mathbb{N}$, $|f_n(x)|=0$, i.e. $f_{n}\rightarrow g$ uniformly on $Z$.
    Then, $\forall\epsilon>0$, by Egoroff's Theorem, $\exists F_{m}\subset G_{m}$ with $\mu(F_m)<\frac{\epsilon}{2^m}$ such that $f_n\rightarrow f$ 
    uniformly on $G_m\setminus F_m$.
    
    Choose $F:=\cp{m}{\infty}F_m$, it's clear that 
    $\mu(F)<\epsilon$. Then we show that $\exists N_{0}$ such that $\forall n\ge N_0$, $|f_n-f|<\epsilon$ on $X\setminus F$. 

    First, choose $M$ such that $\frac{1}{M}<\frac{\epsilon}{2}$, then for $n\ge M$, on $G_{n}^{c}$:
    \begin{displaymath}
        |f_n-f|\le|f_n|+|f|<\epsilon.
    \end{displaymath}
    By the discussions above, on $G_{M}\setminus F$, 
    $f_{n}\rightarrow f$ uniformly, i.e. $\exists N$ such that 
    $\forall x\in G_{M}\setminus F$, 
    $|f_{N}-f|<\epsilon$. 
    Choose $N_{0}=\max\{M,N\}$, 
    $\forall n>N_{0}$, $|f_n-f|<\epsilon$ on $X\setminus E$. 

    So, $f_{n}\rightarrow f$ uniformly on $X\setminus F$ with $\mu(F)<\epsilon$, 
    which completes the proof.
\end{proof}
\begin{homeworkProblem}
    Let $\mu$ be counting measure on $\mathbb{N}$, 
    then $f_n\rightarrow f$ in measure iff $f_n\rightarrow f$ uniformly.
\end{homeworkProblem}
\begin{proof}
    Since $\forall\Sc\in\mathcal{P}(X)$, 
    $\mu_{N}(\Sc)\in\mathbb{N}$, 
    in counting measure, $f_n\rightarrow f$ in measure equivalent to :
    \begin{displaymath}
        \forall \epsilon>0,\;\exists M,\;\forall m>M,\;n\in\mathbb{N},\;
        |f_{m}(n)-f(n)|<\epsilon.
    \end{displaymath}
    It's just the definition of $f_{n}\rightarrow f$ uniformly.
\end{proof}
\begin{homeworkProblem}
    Let $X=Y=\mathbb{N}$, $\M=\mathcal{N}=\mathcal{P}(\mathbb{B})$, 
    $\mu$, $\nu$ are counting measures. 
    Define $f(m,n)=1$ if $m=n$, 
    $f(m,n)=-1$ if $m=n+1$ and $f(m,n)=0$ otherwise, 
    then $\int|f|\dif(\mu\times\nu)=\infty$, and 
    $\iint f\dif\mu\dif\nu$ and 
    $\iint f\dif\nu\dif\mu$ exist and are unequal.
\end{homeworkProblem}
\begin{proof}
    $f(m,n)$ marks an infinite matrix 
    with diagonal elements equal $1$ and subdiagonal elements equal $-1$. 
    Now:
    \begin{displaymath}
        \int|f|\dif(\mu\times\nu)
        =\sum_{m,n=1}^{\infty}|f(m,n)|
        =2\aleph_{0}=\infty.
    \end{displaymath}
    Since 
    \begin{displaymath}
        \int f(x,y)\dif\mu(x)
        =\sum_{n=1}^{\infty}f(n,y)
        =0,
    \end{displaymath}
    $\iint f\dif\mu\dif\nu=0$.

    Since 
    \begin{displaymath}
        \int f(x,y)\dif\nu(y)
        =\sum_{n=1}^{\infty}f(x,n)
        =\left\{
            \begin{aligned}
                1&\quad x=0\\
                0&\quad x\neq 0\\
            \end{aligned}
        \right.,
    \end{displaymath}
    $\iint f\dif\nu\dif\mu=1$. 
    So $\iint f\dif\mu\dif\nu\neq\iint f\dif\nu\dif\mu$.
\end{proof}
\begin{homeworkProblem}
    The Fubini-Tonelli theorem is valid when $(X,\M,\mu)$ is an arbitrary measure space 
    and $Y$ is a countable set, $\mathcal{N}=\mathcal{P}(Y)$, 
    and $\nu$ is counting measure on $Y$.
\end{homeworkProblem}
\begin{proof}
    It suffices to show $\sm{n}{\infty}\int f_n=\int\sm{n}{\infty}f_n$ with the condition:
    \begin{itemize}
        \item Tonelli: $\{f_n\}$ are positive functions (It's Theorem 2.15).
        \item Fubini: $\sm{n}{\infty}\int|f_n|<\infty$ 
        (It's Theorem 2.25).
    \end{itemize}
\end{proof}
\end{document}