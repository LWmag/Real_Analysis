\documentclass{article}

\usepackage{fancyhdr}
\usepackage{extramarks}
\usepackage{amsmath}
\usepackage{amsthm}
\newtheorem{lemma}{Lemma}
\usepackage{amsfonts}
\usepackage{tikz}
\usepackage[plain]{algorithm}
\usepackage{algpseudocode}

\usetikzlibrary{automata,positioning}

%
% Basic Document Settings
%

\topmargin=-0.45in
\evensidemargin=0in
\oddsidemargin=0in
\textwidth=6.5in
\textheight=9.0in
\headsep=0.25in

\linespread{1.1}

\pagestyle{fancy}
\lhead{\hmwkAuthorName}
\chead{\hmwkClass\ (\hmwkClassInstructor\ \hmwkClassTime): \hmwkTitle}
\rhead{\firstxmark}
\lfoot{\lastxmark}
\cfoot{\thepage}

\renewcommand\headrulewidth{0.4pt}
\renewcommand\footrulewidth{0.4pt}

\setlength\parindent{0pt}

%
% Create Problem Sections
%

\newcommand{\enterProblemHeader}[1]{
    \nobreak\extramarks{}{Problem \arabic{#1} continued on next page\ldots}\nobreak{}
    \nobreak\extramarks{Problem \arabic{#1} (continued)}{Problem \arabic{#1} continued on next page\ldots}\nobreak{}
}

\newcommand{\exitProblemHeader}[1]{
    \nobreak\extramarks{Problem \arabic{#1} (continued)}{Problem \arabic{#1} continued on next page\ldots}\nobreak{}
    \stepcounter{#1}
    \nobreak\extramarks{Problem \arabic{#1}}{}\nobreak{}
}

\setcounter{secnumdepth}{0}
\newcounter{partCounter}
\newcounter{homeworkProblemCounter}
\setcounter{homeworkProblemCounter}{1}
\nobreak\extramarks{Problem \arabic{homeworkProblemCounter}}{}\nobreak{}

%
% Homework Problem Environment
%
% This environment takes an optional argument. When given, it will adjust the
% problem counter. This is useful for when the problems given for your
% assignment aren't sequential. See the last 3 problems of this template for an
% example.
%
\newenvironment{homeworkProblem}[1][-1]{
    \ifnum#1>0
        \setcounter{homeworkProblemCounter}{#1}
    \fi
    \section{Problem \arabic{homeworkProblemCounter}}
    \setcounter{partCounter}{1}
    \enterProblemHeader{homeworkProblemCounter}
}{
    \exitProblemHeader{homeworkProblemCounter}
}

%
% Homework Details
%   - Title
%   - Due date
%   - Class
%   - Section/Time
%   - Instructor
%   - Author
%

\newcommand{\hmwkTitle}{Homework\ \#8}
\newcommand{\hmwkDueDate}{Apr 30, 2024}
\newcommand{\hmwkClass}{Real Analysis}
\newcommand{\hmwkClassTime}{Tuesday}
\newcommand{\hmwkClassInstructor}{Professor Yakun Xi}
\newcommand{\hmwkAuthorName}{\textbf{Shuang Hu}}

%
% Title Page
%

\title{
    \vspace{2in}
    \textmd{\textbf{\hmwkClass:\ \hmwkTitle}}\\
    \normalsize\vspace{0.1in}\small{Due\ on\ \hmwkDueDate\ at 10:00am}\\
    \vspace{0.1in}\large{\textit{\hmwkClassInstructor\ \hmwkClassTime}}
    \vspace{3in}
}

\author{\hmwkAuthorName}
\date{}

\renewcommand{\part}[1]{\textbf{\large Part \Alph{partCounter}}\stepcounter{partCounter}\\}

%
% Various Helper Commands
%

% Useful for algorithms
\newcommand{\alg}[1]{\textsc{\bfseries \footnotesize #1}}

% For derivatives
\newcommand{\deriv}[1]{\frac{\mathrm{d}}{\mathrm{d}x} (#1)}

% For partial derivatives
\newcommand{\pderiv}[2]{\frac{\partial}{\partial #1} (#2)}

% Integral dx
\newcommand{\dx}{\mathrm{d}x}

% Alias for the Solution section header
\newcommand{\solution}{\textbf{\large Solution}}
\newcommand{\norm}[1]{\|#1\|}
% Probability commands: Expectation, Variance, Covariance, Bias
\newcommand{\Var}{\mathrm{Var}}
\newcommand{\Cov}{\mathrm{Cov}}
\newcommand{\Bias}{\mathrm{Bias}}
\newcommand{\supp}{\text{supp}}
\newcommand{\Rn}{\mathbb{R}^{n}}
\newcommand{\dif}{\mathrm{d}}
\newcommand{\avg}[1]{\left\langle #1 \right\rangle}
\newcommand{\difFrac}[2]{\frac{\dif #1}{\dif #2}}
\newcommand{\pdfFrac}[2]{\frac{\partial #1}{\partial #2}}
\newcommand{\OFL}{\mathrm{OFL}}
\newcommand{\UFL}{\mathrm{UFL}}
\newcommand{\fl}{\mathrm{fl}}
\newcommand{\Eabs}{E_{\mathrm{abs}}}
\newcommand{\Erel}{E_{\mathrm{rel}}}
\newcommand{\DR}{\mathcal{D}_{\widetilde{LN}}^{n}}
\newcommand{\add}[2]{\sum_{#1=1}^{#2}}
\newcommand{\innerprod}[2]{\left<#1,#2\right>}
\newcommand{\Sc}{\mathcal{S}}
\newcommand{\F}{\mathcal{F}}
\newcommand{\E}{\mathcal{E}}
\newcommand{\A}{\mathcal{A}}
\newcommand{\cp}[2]{\cup_{#1=1}^{#2}}
\newcommand{\sm}[2]{\sum_{#1=1}^{#2}}
\newcommand{\M}{\mathcal{M}}
\newcommand{\Lc}{\mathcal{L}}
\newcommand\tbbint{{-\mkern -16mu\int}}
\newcommand\tbint{{\mathchar '26\mkern -14mu\int}}
\newcommand\dbbint{{-\mkern -19mu\int}}
\newcommand\dbint{{\mathchar '26\mkern -18mu\int}}
\newcommand\bint{
{\mathchoice{\dbint}{\tbint}{\tbint}{\tbint}}
}
\newcommand\bbint{
{\mathchoice{\dbbint}{\tbbint}{\tbbint}{\tbbint}}
}
\begin{document}
\maketitle
\pagebreak
\begin{homeworkProblem}
    Prove Proposition 3.1
\end{homeworkProblem}
\begin{proof}
    Set $F_n=\cp{j}{n}E_j$, $F_0=\emptyset$, 
    $S_n:=F_n\setminus F_{n-1}$, it means that 
    \begin{displaymath}
        F_1\subset F_{2}\subset\ldots\subset F_n\subset\ldots,
    \end{displaymath}
    and since $\{E_i\}$ increasing, it means $F_n=E_n$, 
    $\{\Sc_{i}\}$ disjoint and 
    \begin{displaymath}
        \cp{i}{\infty}\Sc_{i}=\cp{i}{\infty}E_i.
    \end{displaymath}
    Then:
    \begin{displaymath}
        \nu(\cp{i}{\infty}E_i)=\nu(\cp{i}{\infty}\Sc_i)
        =\sm{i}{\infty}\nu(\Sc_{i})=\sm{i}{\infty}[\nu(F_i)-\nu(F_{i-1})]
    \end{displaymath}
    If $\nu(\cp{i}{\infty}E_i)<\infty$, 
    by the definition, $\sm{i}{\infty}\nu(\Sc_{i})$ converges absolutely, 
    i.e. 
    \begin{displaymath}
        \sm{i}{\infty}(\nu(F_i)-\nu(F_{i-1}))
        =\lim_{n\rightarrow\infty}\nu(F_n)
        =\lim_{n\rightarrow\infty}\nu(E_n).
    \end{displaymath}
    If $\nu(\cp{i}{\infty}E_{i})=\infty$, i.e. $\sm{i}{\infty}(\nu(F_{i})-\nu(F_{i-1}))=\infty$, 
    it means that $\lim_{n\rightarrow\infty}\nu(E_n)
    =\lim_{n\rightarrow\infty}\nu(F_n)=\infty$.

    The proof of continuous from above 
    is totally the same as Theorem 1.8.
\end{proof}
\begin{homeworkProblem}
    If $\nu$ is a signed measure, $E$ is $\nu$-null 
    iff $|\nu|(E)=0$. Also, if $\nu$ and $\mu$ are signed measures, 
    $\nu\perp\mu$ iff $|\nu|\perp\mu$ iff $\nu^{+}\perp\mu$ and $\nu^{-}\perp\mu$.
\end{homeworkProblem}
\begin{proof}
    $"\Leftarrow":$ $\forall F\subset E$ and $F\in\M$, 
    since $|\nu|$ is a positive measure on $\M$, 
    $|\nu|(E)=0$ means $|\nu|(F)=0$, i.e. 
    $\nu^{+}(F)+\nu^{-}(F)=0$. Since $\nu^{+}$ and $\nu^{-}$ are both non-negative, $\nu^{+}(F)=\nu^{-}(F)=0$, then 
    $\nu(F)=\nu^{+}(F)-\nu^{-}(F)=0$. 
    Which means $E$ is a null set.

    $"\Rightarrow":$ If $\nu(E)=0$ and $\nu^{+}(E)=\nu^{-}(E)=a>0$, by Jordan Decomposition Theorem, 
    $\nu^{+}\perp\nu^{-}$, 
    so there exists $F,G$ such that 
    $F\cap G=\emptyset$, $F\cup G=X$, $F$ is null for $\nu^{+}$ and $G$ is null for $\nu^{-}$.
    Choose $\tilde{F}:=F\cap E$, $\tilde{G}:=G\cap E$, we have 
    \begin{displaymath}
        a=\nu^{+}(E)=\nu^{+}(\tilde{F})
        +\nu^{+}(\tilde{G})=\nu^{+}(\tilde{G})
    \end{displaymath}
    Since $\nu^{-}(\tilde{G})=0$, we can see $\nu(\tilde{G})=a\neq 0$, 
    so $E$ isn't $\nu$-null, contradict!

    By the result above, $\nu$ is $F$-null iff $|\nu|$ is $F$-null 
    iff $\nu^{+}$ and $\nu^{-}$ are $F$-null, 
    so $\nu\perp\mu$ iff $|\nu|\perp\mu$ iff 
    $\nu^{+}\perp\mu$ and $\nu^{-}\perp\mu$.
\end{proof}
\begin{homeworkProblem}
    If $\nu_1$, $\nu_2$ are signed measures that both omit the value $+\infty$ or $-\infty$, 
    then $|\nu_1+\nu_2|\le|\nu_1|+|\nu_2|$.
\end{homeworkProblem}
\begin{proof}  
    It suffices to show:  
    \begin{equation}  
        \label{eq:eqtoshow}  
        |\nu_1+\nu_2|(E)=(\nu_1+\nu_2)^{+}(E)+(\nu_1+\nu_2)^{-}(E)  
        \le\nu_{1}^{+}(E)+\nu_{1}^{-}(E)+\nu_{2}^{+}(E)+\nu_{2}^{-}(E).  
    \end{equation}  
    Since   
    \begin{displaymath}  
        \nu_1+\nu_2=(\nu_{1}^{+}+\nu_{2}^{+})-(\nu_{1}^{-}+\nu_{2}^{-}),  
    \end{displaymath}  
    by Exercise 4, $\nu_{1}^{+}+\nu_{2}^{+}\ge(\nu_1+\nu_2)^{+}$ and   
    $\nu_{1}^{-}+\nu_{2}^{-}\ge(\nu_{1}+\nu_{2})^{-}$.   
    It means \eqref{eq:eqtoshow} is true.  
\end{proof}  
\begin{homeworkProblem}
    $\nu\ll\mu$ iff $|\nu|\ll\mu$ iff $\nu^{+}\ll\mu$ and $\nu^{-}\ll\mu$.
\end{homeworkProblem}
\begin{proof}
    $\nu\ll\mu$ means $\forall E$ s.t. 
    $\mu(E)=0$, 
    $\nu(E)=0$. 
    Since $\mu$ is a positive measure, 
    $\forall F\subset E$ and $F\in\M$, 
    $\mu(F)=0$, then $\nu(F)=0$.
    So it is equivalent to the fact that 
    $\nu$ is null on $E$. 
    By Exercise 2, $\nu\ll\mu$ iff 
    $|\nu|\ll\mu$ iff $\nu^{+}\ll\mu$ and 
    $\nu^{-}\ll\mu$.
\end{proof}
\begin{homeworkProblem}
    Suppose $\{\nu_j\}$ is a sequence of positive measures. 
    If $\nu_{j}\perp\mu$ for all $j$, 
    then $\sm{j}{\infty}\nu_{j}\perp \mu$; 
    and if $\nu_{j}\ll\mu$ for all $j$, 
    then $\sm{j}{\infty}\nu_{j}\ll\mu$.
\end{homeworkProblem}
\begin{proof}
    If $\nu_{j}\perp\mu$, then there 
    exists $E_{j},F_{j}\subset X$ such that 
    $E_j\cap F_j=\emptyset$, $E_j\cup F_j=X$, $E_j$ is null on $\mu$ and $F_j$ is null on $\nu_{j}$.

    Set $F:=\cap_{j=1}^{\infty}F_{j}$, since $\nu_{j}$ positive, 
    \begin{displaymath}
        (\sm{j}{\infty}\nu_j)(F)
        \le\sm{j}{\infty}\nu_{j}(F_j)
        =0.
    \end{displaymath}
    Then set $E=F^{c}=\cp{j}{\infty}F_{j}^{c}$, 
    since $\mu$ positive, 
    \begin{displaymath}
        \mu(E)=\mu(F^{c})\le\sm{j}{\infty}\mu(F_{j}^{c})=\sm{j}{\infty}\mu(E_j)=0.
    \end{displaymath}
    So $\sm{j}{\infty}\nu_{j}\perp\mu$.

    If $\nu_j\ll\mu$, then $\mu(E)=0$ means $\nu_j(E)=0$. $\forall E\in\M$ such that $\mu(E)=0$, 
    we can see $\forall j$, $\nu_j(E)=0$. 
    Since $\nu_{j}$ positive, $(\sm{j}{\infty}\nu_{j})(E)=0$. 
    Then $\sm{j}{\infty}\nu_{j}\ll\mu$.
\end{proof}
\begin{homeworkProblem}
    Theorem 3.5 may fail when $\nu$ is not finite.
\end{homeworkProblem}
\begin{proof}
    For $\dif\nu=\frac{\dif x}{x}$ and $\dif\mu=\dif x$ on $(0,1)$, 
    if $\mu(E)=0$, it means $\int_{E}\dif x=0$, i.e. $m(E)=0$, where 
    $m$ means the Lebesgue measure on $(0,1)$. 
    Then $\nu(E)=\int_{E}\dif\nu=\int_{E}\frac{\dif x}{x}$, since 
    $E\subset(0,1)$ and $m(E)=0$, we can see $\nu(E)=0$. So $\nu\ll\mu$.

    However, since $\int_{0}^{1}\frac{1}{x}\dif x=\infty$, $\forall\delta>0$, 
    $\exists\mu(E_{\delta})<\delta$ s.t. $\int_{E_{\delta}}\frac{1}{x}\dif x>1$, 
    then Theorem 3.5 fails.
\end{proof}
\begin{homeworkProblem}
    If $\nu$ is an arbitrary signed measure and $\mu$ is a $\sigma$-finite 
    measure on $(X,\M)$ such that $\nu\ll\mu$, there exists an extended 
    $\mu$-integrable function $f:X\rightarrow[-\infty,\infty]$ such that 
    $\dif\nu=f\dif\mu$.
\end{homeworkProblem}
\begin{proof}
    First, assume $\mu$ is finite and $\nu$ is positive. 

    Consider the set $\Sc:=\{A:\text{ }A\text{ is }\sigma\text{-finite related 
    to }\mu\}$, then since $\mu$ is finite, 
    $\sup_{E\in\Sc}\mu(E)<\infty$. 
    Choose a sequence $\{E_n\}$ in $\Sc$ such that $\mu(E_n)\rightarrow\sup_{E\in\Sc}\mu(E)$, 
    i.e. 
    \begin{displaymath}
    \mu(\cap_{n=1}^{\infty}\cup_{j=n}^{\infty}E_j)=\sup_{E\in\Sc}\mu(E),
    \end{displaymath}
    Choose $E=\cap_{n=1}^{\infty}\cup_{j=n}^{\infty}E_{j}$, 
    $\{E_n\}$ is $\sigma$-finite related to $\nu$ 
    means $E$ is $\sigma$-finite related to $\nu$. 
    And by the definition of superior, 
    $\forall F\in\Sc$, $\mu(F)\le\mu(E)$.

    By the above discussion, if $F\cap E=\emptyset$, 
    if $\mu(F)=0$, by $\nu\ll\mu$, $\nu(F)=0$. 
    If $\mu(F)>0$, since $F\notin\Sc$, $|\nu(F)|=\infty$.

    Now, apply R-N Theorem on $E$, $\exists$ $\sigma$-finite signed measures $\lambda$, $\rho$ such that 
    \begin{itemize}
        \item $\lambda\perp\mu$,
        \item $\rho\ll\mu$,
        \item $\nu=\lambda+\rho$,
        \item $\dif\rho=f\dif\mu$.
    \end{itemize}
    Since $\lambda\perp\mu$, $\dif\nu=f\dif\mu$. On $E^{c}$, we set $f=\infty$. 
    Then: $\forall S\subset X$, 
    \begin{displaymath}
        \int_{S}\dif\nu=\int_{S\cap E}\dif\nu
        +\int_{S\cap E^{c}}\dif\nu
        =\int_{S\cap E}f\dif\mu+\int_{S\cap E^{c}}\dif\nu.
    \end{displaymath}
    If $\mu(S\cap E^{c})=0$, it means 
    $\nu(S\cap E^{c})=0$, then 
    $\int_{S\cap E^{c}}\dif\nu=\int_{S\cap E^{c}}f\dif\mu=0$. 
    If $\mu(S\cap E^{c})>0$, then $\int_{S\cap E^c}\dif\nu=
    \int_{S\cap E^c}f\dif\mu=\infty$. 
    So:
    \begin{displaymath}
        \forall S\subset X,\;\int_{S}\dif\nu=\int_{S}f\dif\mu.
    \end{displaymath}
    i.e. $\dif\nu=f\dif\mu$. 

    Now consider the general case. 
    Choose disjoint sets $\{E_{i}\}$ such that $\cp{i}{\infty}E_i=X$, 
    $\mu(E_{i})<\infty$, and by 
    Jordan Decomposition Theorem, 
    $\nu=\nu^{+}-\nu^{-}$.

    By exercise 8, $\nu\ll\mu\Leftrightarrow \nu^{+},\nu^{-}\ll\mu$, it means that 
    $\forall E_{i}$, $\exists f_{i+},f_{i-}:E_i\rightarrow[-\infty,\infty]$ 
    such that 
    \begin{displaymath}
        \begin{aligned}
            \int_{E_i}\dif\nu^{+}&=\int_{E_i}f_{i+}\dif\mu\\
            \int_{E_i}\dif\nu^{-}&=\int_{E_i}f_{i-}\dif\mu.
        \end{aligned}
    \end{displaymath}
    So $\int_{E_i}(f_{i+}-f_{i-})\dif\mu=\int_{E_i}\dif\nu$, it means:
    \begin{displaymath}
        \dif\nu=\left(\sum(f_{i+}-f_{i-})\chi_{E_i}\right)\dif\mu.
    \end{displaymath}
\end{proof}
\begin{homeworkProblem}
    Suppose that $\mu,\nu$ are $\sigma$-finite measures on $(X,\M)$ 
    with $\nu\ll\mu$, and let $\lambda=\mu+\nu$. 
    If $f=\difFrac{\nu}{\lambda}$, 
    then $0\le f<1$ a.e. and $\difFrac{\nu}{\mu}=\frac{f}{1-f}$.
\end{homeworkProblem}
\begin{proof}
    The condition means that $\dif\nu=f\dif\lambda$. 
    Choose $E\subset\M$, then $\int_{E}\dif\nu=\nu(E)=\int_{E}f\dif\lambda\ge 0$. 
    Since $\mu,\nu$ positive, $\lambda$ is positive, so $f\ge0$ $\lambda$ a.e. 
    which means $f\ge 0$ $\mu$ a.e. 

    Since $\lambda(E)\ge\nu(E)$, we get $f\le 1$ a.e. Mark $F:=\{x:f(x)=1\}$, then:
    \begin{displaymath}
        \int_{F}\dif\nu=\int_{F}f\dif\lambda
        \Rightarrow\nu(F)=\lambda(F)
        \Rightarrow\mu(F)=0.
    \end{displaymath}
    Since $\nu\ll\mu$, we can see $\nu(F)=0$, so $\lambda(F)=0$. 
    It shows that $0\le f<1$ $\lambda$-a.e. 

    Then, $\forall E\in\M$, we have:
    \begin{displaymath}
        \int_{E}\dif\nu=\int_{E}f\dif\lambda
        =\int_{E}f\dif(\mu+\nu)
        =\int_{E}f\dif\mu+\int_{E}f\dif\nu.
    \end{displaymath}
    So $\int_{E}(1-f)\dif\nu=\int_{E}f\dif\mu$, it equals:
    \begin{displaymath}
        \forall E\in\M,\;
        ((1-f)\dif\nu-f\dif\mu)(E)=0.
    \end{displaymath}
    So 
    \begin{displaymath}
        (1-f)\dif\nu-f\dif\mu=0\dif\mu=0\dif\nu,
    \end{displaymath}
    which means $(1-f)\dif\nu=f\dif\mu$, i.e. 
    $\dif\nu=\frac{f}{1-f}\dif\mu$.
\end{proof}
\end{document}