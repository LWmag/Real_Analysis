\documentclass{article}

\usepackage{fancyhdr}
\usepackage{extramarks}
\usepackage{amsmath}
\usepackage{amsthm}
\newtheorem{lemma}{Lemma}
\usepackage{amsfonts}
\usepackage{tikz}
\usepackage[plain]{algorithm}
\usepackage{algpseudocode}

\usetikzlibrary{automata,positioning}

%
% Basic Document Settings
%

\topmargin=-0.45in
\evensidemargin=0in
\oddsidemargin=0in
\textwidth=6.5in
\textheight=9.0in
\headsep=0.25in

\linespread{1.1}

\pagestyle{fancy}
\lhead{\hmwkAuthorName}
\chead{\hmwkClass\ (\hmwkClassInstructor\ \hmwkClassTime): \hmwkTitle}
\rhead{\firstxmark}
\lfoot{\lastxmark}
\cfoot{\thepage}

\renewcommand\headrulewidth{0.4pt}
\renewcommand\footrulewidth{0.4pt}

\setlength\parindent{0pt}

%
% Create Problem Sections
%

\newcommand{\enterProblemHeader}[1]{
    \nobreak\extramarks{}{Problem \arabic{#1} continued on next page\ldots}\nobreak{}
    \nobreak\extramarks{Problem \arabic{#1} (continued)}{Problem \arabic{#1} continued on next page\ldots}\nobreak{}
}

\newcommand{\exitProblemHeader}[1]{
    \nobreak\extramarks{Problem \arabic{#1} (continued)}{Problem \arabic{#1} continued on next page\ldots}\nobreak{}
    \stepcounter{#1}
    \nobreak\extramarks{Problem \arabic{#1}}{}\nobreak{}
}

\setcounter{secnumdepth}{0}
\newcounter{partCounter}
\newcounter{homeworkProblemCounter}
\setcounter{homeworkProblemCounter}{1}
\nobreak\extramarks{Problem \arabic{homeworkProblemCounter}}{}\nobreak{}

%
% Homework Problem Environment
%
% This environment takes an optional argument. When given, it will adjust the
% problem counter. This is useful for when the problems given for your
% assignment aren't sequential. See the last 3 problems of this template for an
% example.
%
\newenvironment{homeworkProblem}[1][-1]{
    \ifnum#1>0
        \setcounter{homeworkProblemCounter}{#1}
    \fi
    \section{Problem \arabic{homeworkProblemCounter}}
    \setcounter{partCounter}{1}
    \enterProblemHeader{homeworkProblemCounter}
}{
    \exitProblemHeader{homeworkProblemCounter}
}

%
% Homework Details
%   - Title
%   - Due date
%   - Class
%   - Section/Time
%   - Instructor
%   - Author
%

\newcommand{\hmwkTitle}{Homework\ \#9}
\newcommand{\hmwkDueDate}{May 7, 2024}
\newcommand{\hmwkClass}{Real Analysis}
\newcommand{\hmwkClassTime}{Tuesday}
\newcommand{\hmwkClassInstructor}{Professor Yakun Xi}
\newcommand{\hmwkAuthorName}{\textbf{Shuang Hu}}

%
% Title Page
%

\title{
    \vspace{2in}
    \textmd{\textbf{\hmwkClass:\ \hmwkTitle}}\\
    \normalsize\vspace{0.1in}\small{Due\ on\ \hmwkDueDate\ at 10:00am}\\
    \vspace{0.1in}\large{\textit{\hmwkClassInstructor\ \hmwkClassTime}}
    \vspace{3in}
}

\author{\hmwkAuthorName}
\date{}

\renewcommand{\part}[1]{\textbf{\large Part \Alph{partCounter}}\stepcounter{partCounter}\\}

%
% Various Helper Commands
%

% Useful for algorithms
\newcommand{\alg}[1]{\textsc{\bfseries \footnotesize #1}}

% For derivatives
\newcommand{\deriv}[1]{\frac{\mathrm{d}}{\mathrm{d}x} (#1)}

% For partial derivatives
\newcommand{\pderiv}[2]{\frac{\partial}{\partial #1} (#2)}

% Integral dx
\newcommand{\dx}{\mathrm{d}x}

% Alias for the Solution section header
\newcommand{\solution}{\textbf{\large Solution}}
\newcommand{\norm}[1]{\|#1\|}
% Probability commands: Expectation, Variance, Covariance, Bias
\newcommand{\Var}{\mathrm{Var}}
\newcommand{\Cov}{\mathrm{Cov}}
\newcommand{\Bias}{\mathrm{Bias}}
\newcommand{\supp}{\text{supp}}
\newcommand{\Rn}{\mathbb{R}^{n}}
\newcommand{\dif}{\mathrm{d}}
\newcommand{\avg}[1]{\left\langle #1 \right\rangle}
\newcommand{\difFrac}[2]{\frac{\dif #1}{\dif #2}}
\newcommand{\pdfFrac}[2]{\frac{\partial #1}{\partial #2}}
\newcommand{\OFL}{\mathrm{OFL}}
\newcommand{\UFL}{\mathrm{UFL}}
\newcommand{\fl}{\mathrm{fl}}
\newcommand{\Eabs}{E_{\mathrm{abs}}}
\newcommand{\Erel}{E_{\mathrm{rel}}}
\newcommand{\DR}{\mathcal{D}_{\widetilde{LN}}^{n}}
\newcommand{\add}[2]{\sum_{#1=1}^{#2}}
\newcommand{\innerprod}[2]{\left<#1,#2\right>}
\newcommand{\Sc}{\mathcal{S}}
\newcommand{\F}{\mathcal{F}}
\newcommand{\E}{\mathcal{E}}
\newcommand{\A}{\mathcal{A}}
\newcommand{\cp}[2]{\cup_{#1=1}^{#2}}
\newcommand{\sm}[2]{\sum_{#1=1}^{#2}}
\newcommand{\M}{\mathcal{M}}
\newcommand{\Lc}{\mathcal{L}}
\newcommand\tbbint{{-\mkern -16mu\int}}
\newcommand\tbint{{\mathchar '26\mkern -14mu\int}}
\newcommand\dbbint{{-\mkern -19mu\int}}
\newcommand\dbint{{\mathchar '26\mkern -18mu\int}}
\newcommand\bint{
{\mathchoice{\dbint}{\tbint}{\tbint}{\tbint}}
}
\newcommand\bbint{
{\mathchoice{\dbbint}{\tbbint}{\tbbint}{\tbbint}}
}
\begin{document}
\maketitle
\pagebreak
\begin{homeworkProblem}
    Prove Proposition 3.13c
\end{homeworkProblem}
\begin{proof}
    Suppose $\dif\nu=h\dif\mu$, then $\dif|\nu|=|h|\dif\mu$, assume 
    $h=h_r+ih_{i}$, then:
    \begin{displaymath}
        g\in L^{1}(\nu)\Leftrightarrow |g|h_{r},|g|h_i\in L^{1}(\mu)
        \Leftrightarrow |g||h|\in L^{1}(\mu)\Leftrightarrow g\in L^{1}(|\nu|).
    \end{displaymath}
    The inequality:
    \begin{displaymath}
        \left|\int f\dif\nu\right|=\left|\int fh\dif\mu\right|\le
        \int\left|fh\right|\dif\mu
        \le\int|f||h|\dif\mu=\int|f|\dif|\nu|.
    \end{displaymath}
\end{proof}
\begin{homeworkProblem}
    If $\nu$, $\mu$ are complex measures and $\lambda$ is a positive measure, 
    then $\nu\perp\mu$ iff $|\nu|\perp|\mu|$, and 
    $\nu\ll\lambda$ iff $|\nu|\ll\lambda$.
\end{homeworkProblem}
\begin{proof}
    We introduce the following lemma: 
    \begin{lemma}
    If $E$ is null related to $\nu$, 
    then $E$ is null related to $|\nu|$.
    \end{lemma}
    \begin{proof}
        E is null related to $\nu$ means $E$ is null related to $\nu_{i}$ and $\nu_{r}$, 
        assume $\dif\nu=f\dif\mu$, it means:
        \begin{displaymath}
            \forall F\in \mathcal{P}(E)\cap\M, \int_{F}(f_r+if_{i})\dif\mu=0.
        \end{displaymath}
        It means $f_r=f_i=0$ on $E$ a.e., 
        so $\int_{F}|f|\dif\mu=0$, which means $E$ is null related to $|\nu|$. 
        
        Since $|\nu(E)|\le|\nu|(E)$, if $E$ is null related to $|\nu|$, 
        then $E$ is null related to $\nu$.
    \end{proof}
    The problem is just a simple corollary of the lemma.
\end{proof}
\begin{homeworkProblem}
    If $\nu$ is a complex measure on $(X,\M)$ and $\nu(X)=|\nu|(X)$, then $\nu=|\nu|$.
\end{homeworkProblem}
\begin{proof}
    Assume $\dif\nu=f\dif\mu$ and $f=f_r+if_i$, 
    then $\nu(X)=|\nu|(X)$ means:
    \begin{displaymath}
        \int(|f|-f)\dif\mu=0.
    \end{displaymath}
    It means:
    \begin{displaymath}
        \begin{array}{rl}
            &\int f_i\dif\mu=0,\\
            &\int(|f_r|-f_r+|f_i|)\dif\mu=0.
        \end{array}
    \end{displaymath}
    Since $|f_{r}|-f_r\ge 0$, $|f_i|\ge 0$, it means 
    $|f_r|-f_r+|f_i|\ge 0$. 
    So $|f_r|-f_r+|f_{i}|=0$ a.e., which means:
    \begin{itemize}
        \item $|f_r|=f_{r}$ a.e.
        \item $|f_i|=0$ a.e.
    \end{itemize}
    Then $|f|=f$ a.e., which means $\nu=|\nu|$.
\end{proof}
\begin{homeworkProblem}
    If $f\in L^1(\mathbb{R}^n)$, $f\neq 0$, there exist $C,R>0$ such that 
    $Hf(x)\ge C|x|^{-n}$ for $|x|>R$. Hence 
    $m(\{x:Hf(x)>\alpha\})\ge\frac{C'}{\alpha}$ when $\alpha$ is small, 
    so the estimate in the maximal thoerem is essentially sharp.
\end{homeworkProblem}
\begin{proof}
    Since $Hf(x)=\sup_{r>0}A_{r}|f|(x)$, WLOG, we assume $f\ge 0$ a.e. 
    Then:
    \begin{displaymath}
        \begin{array}{rl}
            Hf(x)<C|x|^{-n}&\Leftrightarrow
            \forall r>0, A_{r}f(x)<C|x|^{-n}\\
            &\Leftrightarrow \forall r>0,\frac{\int_{B(r,x)}f(\tau)\dif\tau}
            {m(B(r,x))}<C|x|^{-n}\\
            &\Leftrightarrow \forall r>0,\frac{|x|^{n}\int_{B(r,x)}f(\tau)\dif\tau}
            {m(B(r,x))}<C.
        \end{array}
    \end{displaymath}
    If $\forall C,R>0$, $\exists |x|>R$ such that $\forall r>0$, $A_{r}f(x)<C|x|^{-n}$, 
    we choosem $C(m):=\frac{1}{m}$, then:
    $\forall R>0$, $\exists |x|>R$ s.t. 
    \begin{displaymath}
        \frac{|x|^{n}\int_{B(r,x)}f(\tau)\dif\tau}{m(B(r,x))}<\frac{1}{m}.
    \end{displaymath}
    Choose $m\rightarrow\infty$, we can see:
    \begin{displaymath}
        \forall R,r>0,\exists |x|>R\text{ s.t. }\frac{\int_{B(r,x)}f(\tau)\dif\tau}{m(B(r,x))}=0.
    \end{displaymath}
    Which means $f\equiv 0$ a.e., contradict! So $\exists C,R$ s.t. 
    $\forall |x|>R$, $Hf(x)\ge C|x|^{-n}$. 

    Since $C|x|^{-n}\ge\alpha$ means 
    $|x|\le\left(\frac{C}{\alpha}\right)^{n}$, 
    it's clear that 
    \begin{displaymath}
        m(\{x:Hf(x)\ge\alpha\})\ge 
        m(B((\frac{C}{\alpha})^{n},0))-m(B(R,0))
        \ge\frac{C'}{\alpha}.
    \end{displaymath}
\end{proof}
\begin{homeworkProblem}
    A useful variant of the Hardy-Littlewood maximal function is 
    \begin{displaymath}
        H^{*}f(x)=\sup\left\{\frac{1}{m(B)}\int_{B}|f(y)|\dif y\text{: }
        B\text{ is a ball and }x\in B\right\},
    \end{displaymath}
    show that $Hf\le H^{*}f\le 2^{n}Hf$.
\end{homeworkProblem}
\begin{proof}
    $\forall x\in\mathbb{R}^{n}$, 
    since $B(r,x)$ is a ball contains $x$, it's clear that 
    $Hf(x)\le H^{*}f(x)$. It suffices to show that $H^{*}f(x)\le 2^{n}Hf(x)$. 
    
    Choose a ball $B$ such that $x\in B$ arbitrarily and mark 
    \begin{displaymath}
        H^{*}(B,f):=\frac{1}{m(B)}\int_{B}|f(y)|\dif y,
    \end{displaymath}
    assume $d:=\text{diam}(B)$, 
    $\tilde{d}:=\text{dist}(x,\partial B)$, then:
    \begin{displaymath}
        B(d,x)\supset B(d-\tilde{d},x)\supset B.
    \end{displaymath}
    So:
    \begin{displaymath}
        H^{*}(B,f)\le\frac{\int_{B(d,x)}|f(y)|\dif y}{m(B)}
        =\frac{\int_{B(d,x)}|f(y)|\dif y}{m(B(d,x))}\frac{m(B(d,x))}{m(B)}
        =2^{n}\frac{\int_{B(d,x)}|f(y)|\dif y}{m(B(d,x))}.
    \end{displaymath}
    Get the supremum related to $B$, we get 
    \begin{displaymath}
        H^{*}f(x)\le 2^{n}Hf(x).
    \end{displaymath}
\end{proof}
\end{document}