\documentclass{article}

\usepackage{fancyhdr}
\usepackage{extramarks}
\usepackage{amsmath}
\usepackage{amsthm}
\newtheorem{lemma}{Lemma}
\usepackage{amsfonts}
\usepackage{tikz}
\usepackage[plain]{algorithm}
\usepackage{algpseudocode}

\usetikzlibrary{automata,positioning}

%
% Basic Document Settings
%

\topmargin=-0.45in
\evensidemargin=0in
\oddsidemargin=0in
\textwidth=6.5in
\textheight=9.0in
\headsep=0.25in

\linespread{1.1}

\pagestyle{fancy}
\lhead{\hmwkAuthorName}
\chead{\hmwkClass\ (\hmwkClassInstructor\ \hmwkClassTime): \hmwkTitle}
\rhead{\firstxmark}
\lfoot{\lastxmark}
\cfoot{\thepage}

\renewcommand\headrulewidth{0.4pt}
\renewcommand\footrulewidth{0.4pt}

\setlength\parindent{0pt}

%
% Create Problem Sections
%

\newcommand{\enterProblemHeader}[1]{
    \nobreak\extramarks{}{Problem \arabic{#1} continued on next page\ldots}\nobreak{}
    \nobreak\extramarks{Problem \arabic{#1} (continued)}{Problem \arabic{#1} continued on next page\ldots}\nobreak{}
}

\newcommand{\exitProblemHeader}[1]{
    \nobreak\extramarks{Problem \arabic{#1} (continued)}{Problem \arabic{#1} continued on next page\ldots}\nobreak{}
    \stepcounter{#1}
    \nobreak\extramarks{Problem \arabic{#1}}{}\nobreak{}
}

\setcounter{secnumdepth}{0}
\newcounter{partCounter}
\newcounter{homeworkProblemCounter}
\setcounter{homeworkProblemCounter}{1}
\nobreak\extramarks{Problem \arabic{homeworkProblemCounter}}{}\nobreak{}

%
% Homework Problem Environment
%
% This environment takes an optional argument. When given, it will adjust the
% problem counter. This is useful for when the problems given for your
% assignment aren't sequential. See the last 3 problems of this template for an
% example.
%
\newenvironment{homeworkProblem}[1][-1]{
    \ifnum#1>0
        \setcounter{homeworkProblemCounter}{#1}
    \fi
    \section{Problem \arabic{homeworkProblemCounter}}
    \setcounter{partCounter}{1}
    \enterProblemHeader{homeworkProblemCounter}
}{
    \exitProblemHeader{homeworkProblemCounter}
}

%
% Homework Details
%   - Title
%   - Due date
%   - Class
%   - Section/Time
%   - Instructor
%   - Author
%

\newcommand{\hmwkTitle}{Homework\ \#10}
\newcommand{\hmwkDueDate}{May 14, 2024}
\newcommand{\hmwkClass}{Real Analysis}
\newcommand{\hmwkClassTime}{Tuesday}
\newcommand{\hmwkClassInstructor}{Professor Yakun Xi}
\newcommand{\hmwkAuthorName}{\textbf{Shuang Hu}}

%
% Title Page
%

\title{
    \vspace{2in}
    \textmd{\textbf{\hmwkClass:\ \hmwkTitle}}\\
    \normalsize\vspace{0.1in}\small{Due\ on\ \hmwkDueDate\ at 10:00am}\\
    \vspace{0.1in}\large{\textit{\hmwkClassInstructor\ \hmwkClassTime}}
    \vspace{3in}
}

\author{\hmwkAuthorName}
\date{}

\renewcommand{\part}[1]{\textbf{\large Part \Alph{partCounter}}\stepcounter{partCounter}\\}

%
% Various Helper Commands
%

% Useful for algorithms
\newcommand{\alg}[1]{\textsc{\bfseries \footnotesize #1}}

% For derivatives
\newcommand{\deriv}[1]{\frac{\mathrm{d}}{\mathrm{d}x} (#1)}

% For partial derivatives
\newcommand{\pderiv}[2]{\frac{\partial}{\partial #1} (#2)}

% Integral dx
\newcommand{\dx}{\mathrm{d}x}

% Alias for the Solution section header
\newcommand{\solution}{\textbf{\large Solution}}
\newcommand{\norm}[1]{\|#1\|}
% Probability commands: Expectation, Variance, Covariance, Bias
\newcommand{\Var}{\mathrm{Var}}
\newcommand{\Cov}{\mathrm{Cov}}
\newcommand{\Bias}{\mathrm{Bias}}
\newcommand{\supp}{\text{supp}}
\newcommand{\Rn}{\mathbb{R}^{n}}
\newcommand{\dif}{\mathrm{d}}
\newcommand{\avg}[1]{\left\langle #1 \right\rangle}
\newcommand{\difFrac}[2]{\frac{\dif #1}{\dif #2}}
\newcommand{\pdfFrac}[2]{\frac{\partial #1}{\partial #2}}
\newcommand{\OFL}{\mathrm{OFL}}
\newcommand{\UFL}{\mathrm{UFL}}
\newcommand{\fl}{\mathrm{fl}}
\newcommand{\Eabs}{E_{\mathrm{abs}}}
\newcommand{\Erel}{E_{\mathrm{rel}}}
\newcommand{\DR}{\mathcal{D}_{\widetilde{LN}}^{n}}
\newcommand{\add}[2]{\sum_{#1=1}^{#2}}
\newcommand{\innerprod}[2]{\left<#1,#2\right>}
\newcommand{\Sc}{\mathcal{S}}
\newcommand{\F}{\mathcal{F}}
\newcommand{\E}{\mathcal{E}}
\newcommand{\A}{\mathcal{A}}
\newcommand{\cp}[2]{\cup_{#1=1}^{#2}}
\newcommand{\sm}[2]{\sum_{#1=1}^{#2}}
\newcommand{\M}{\mathcal{M}}
\newcommand{\Lc}{\mathcal{L}}
\newcommand\tbbint{{-\mkern -16mu\int}}
\newcommand\tbint{{\mathchar '26\mkern -14mu\int}}
\newcommand\dbbint{{-\mkern -19mu\int}}
\newcommand\dbint{{\mathchar '26\mkern -18mu\int}}
\newcommand\bint{
{\mathchoice{\dbint}{\tbint}{\tbint}{\tbint}}
}
\newcommand\bbint{
{\mathchoice{\dbbint}{\tbbint}{\tbbint}{\tbbint}}
}
\begin{document}
\maketitle
\pagebreak
\begin{homeworkProblem}
    Let $\nu$ be a complex measure on $(X,\M)$. If $E\in\M$, define:
    \begin{displaymath}
        \begin{array}{rl}
            \mu_{1}(E)&=\sup\{\sum_{1}^{n}|\nu(E_i)|:n\in\mathbb{N}, 
            E_{1},\ldots, E_{n}\text{ disjoint, }E=\cp{i}{n}E_{i}\},\\
            \mu_2(E)&=\sup\{\sm{i}{\infty}|\nu(E_i)|:E_1,\ldots,E_{n}\text{ disjoint, }
            E=\cp{1}{\infty}E_i\},\\
            \mu_3(E)&=\sup\left\{\left|\int_{E}f\dif\nu\right|:|f|\le 1\right\}.
        \end{array}
    \end{displaymath}
    Then $\mu_1=\mu_2=\mu_3=\nu$.
\end{homeworkProblem}
\begin{proof}
    First, we show that $\mu_1\le\mu_2\le\mu_3$. 

    If $E=\cp{j}{n}E_j$, $E_j$ disjoint, 
    set $\tilde{E}_j:=E_{j}$ for $j\le n$ and 
    $\tilde{E}_j:=\emptyset$ for $j>n$, then $\tilde{E}_j$ disjoint and 
    \begin{displaymath}
        \sm{j}{n}|\nu(E_j)|=\sm{j}{n}|\nu(\tilde{E}_j)|.
    \end{displaymath}
    Get supermum for $\sm{j}{n}|\nu(E_j)|$, it's clear that $\mu_1(E)\le\mu_2(E)$, 
    so $\mu_1\le\mu_2$.

    Assume the positive measure $\tilde{\mu}$ satisfies $\nu\ll\tilde{\mu}$, 
    and $\dif\nu=h\dif\tilde{\mu}$, then $\forall\epsilon>0$, 
    $\exists$ disjoint sets $E_{j}$ satisfies:
    \begin{displaymath}
        \mu_{2}(E)-\epsilon=\sm{j}{\infty}\left|\int_{E_j}h\dif\tilde{\mu}\right|
        \le\sm{j}{\infty}\int_{E_j}|h|\dif\tilde{\mu}=\int_{E}|h|\dif\tilde{\mu}.
    \end{displaymath}
    and 
    \begin{displaymath}
        \mu_{3}(E)=\sup_{|f|\le 1}\left\{\left|\int_{E}f\dif\nu\right|\right\}
        =\sup_{|f|\le 1}\left\{\left|\int_{E}fh\dif\tilde{\mu}\right|\right\}
        =\int_{E}|h|\dif\tilde{\mu}.
    \end{displaymath}
    So $\forall\epsilon>0$, $\mu_{2}(E)-\epsilon\le\mu_{3}(E)$, i.e. 
    $\mu_{2}\le\mu_{3}$. Then we show that $\mu_{3}=|\nu|$.

    On one hand, 
    \begin{displaymath}
        \left|\int_{E}f\dif\nu\right|\le\int_{E}|f|\dif|\nu|
        \le\int_{E}\dif|\nu|=|\nu|(E),
    \end{displaymath}
    i.e. $\mu_{3}\le|\nu|$. On the other hand, let $f=\overline{\frac{\dif\nu}{\dif|\nu|}}$, 
    then 
    \begin{displaymath}
        \mu_{3}(E)\ge\left|\int_{E}\overline{\frac{\dif\nu}{\dif|\nu|}}\dif\nu\right|
        =\left|\int_{E}\dif|\nu|\right|=|\nu|(E),
    \end{displaymath}
    i.e. $|\nu|\le\mu_{3}$, so $\mu_{3}=|\nu|$.

    Finally, we show $\mu_{3}\le\mu_{1}$. By the definition of integral, 
    $\exists$ a sequence of simple functions $\chi_{nf}$ such that 
    \begin{displaymath}
        \int_{E}\chi_{nf}\rightarrow\int_{E}f.
    \end{displaymath}
    What's more, $\forall$ simple function $\chi$ on $E$ with $|\chi|\le 1$, 
    $\chi:=\sm{i}{n}a_{i}\chi_{E_i}$, 
    \begin{displaymath}
        \left|\int_{E}\chi\dif\nu\right|=\sm{i}{n}|a_i||\nu(E_i)|
        \le\sm{i}{n}|\nu(E_i)|=\mu_{1}(E).
    \end{displaymath}
    So $\mu_{3}(E)\le\mu_{1}(E)$.
\end{proof}
\begin{homeworkProblem}
    If $f\in L_{\text{loc}}^{1}$ and $f$ is continuous at $x$, then $x$ is in the 
    Lebesgue set of $f$.
\end{homeworkProblem}
\begin{proof}
    $f$ is continuous at $x$ means $\forall\epsilon>0$, $\exists\delta$ such that 
    $\forall|y-x|<\delta_{\epsilon}$, $|f(y)-f(x)|<\epsilon$. Then:
    \begin{displaymath}
        \forall r<\delta_{\epsilon}, 
        \frac{1}{m(B(r,x))}\int_{B(r,x)}|f(y)-f(x)|\dif y<\epsilon.
    \end{displaymath}
    So:
    \begin{displaymath}
        \forall\epsilon>0, \limsup_{r\rightarrow 0}\frac{1}{m(B(r,x))}
        \int_{B(r,x)}|f(y)-f(x)|\dif y\le \epsilon.
    \end{displaymath}
    It means that $x\in L_{f}$.        
\end{proof}
\begin{homeworkProblem}
    If $E$ is a Borel set in $\mathbb{R}^{n}$, the \textbf{density} $D_{E}(x)$ of $E$ 
    at $x$ is defined as 
    \begin{displaymath}
        D_{E}(x):=\lim_{r\rightarrow 0}\frac{m(E\cap B(r,x))}{m(B(r,x))},
    \end{displaymath}
    whenever the limit exists.
    \begin{enumerate}
        \item Show that $D_{E}(x)=1$ for a.e. $x\in E$ and $D_{E}(x)=0$ for a.e. $x\in E^{c}$.
        \item Find examples of $E$ and $x$ such that $D_{E}(x)$ is a given number $\alpha\in(0,1)$, 
        or such that $D_{E}(x)$ doesn't exist.
    \end{enumerate}
\end{homeworkProblem}
\begin{proof}
    $(1)$ Choose $E_{r}:=B(r,x)$, then $E_{r}$ shrinks nicely to $x$. 
    Mark $f:=\chi_{E}$, by Lebesgue Differentiation Theorem, 
    \begin{displaymath}
        \lim_{r\rightarrow 0}\frac{\int_{B(r,x)}\chi_{E}}{m(B(r,x))}
        =\lim_{r\rightarrow 0}\frac{m(E\cap B(r,x))}{m(B(r,x))}=\chi_{E}(x)\;a.e.
    \end{displaymath}
    Then: for a.e. $x\in E$, $D_{E}(x)=\chi_{E}(x)=1$. 
    For a.e. $x\in E^{c}$, $D_{E}(x)=\chi_{E}(x)=0$.\qed

    $(2)$ $D_{E}(0)=\alpha$: If $n\ge 2$, 
    just choose $x=0$ and 
    \begin{displaymath}
        E:=\left\{x\in\mathbb{R}^{n}:(x,e_1)\ge\cos(\alpha\pi)\right\}.
    \end{displaymath}
    Then $D_{E}(0)=\alpha$. 

    (The following example comes from Math Stack Exchange)
    If $n=1$, put $U_{n,\alpha}:=\left(\frac{1}{n+1},\frac{1}{n+1}+\alpha(\frac{1}{n}-\frac{1}{n+1})\right)$, 
    set $E_{n}:=U_{n,\alpha}\cup(-U_{n,\alpha})$, $E:=\cup_{n\ge 1}E_{n}$, 
    then $D_{E}(0)=\alpha$.

    $D_{E}(0)$ doesn't exist: choose $F$ be a fat-Cantor set with $m(F)=\sigma>0$, and $E:=F^{n}$, then $D_{E}(0)$ doesn't exist. 
\end{proof}
\begin{homeworkProblem}
    If $\lambda,\mu$ are positive, mutually singular Borel measures on $\mathbb{R}^{n}$ and $\lambda+\mu$ is regular, then so are $\lambda$ and $\mu$.
\end{homeworkProblem}
\begin{proof}
    Since $\lambda,\mu$ are both positive and Borel measures, 
    $\forall$ compact set $K\subset\mathbb{R}^{n}$, it satisfies:
    \begin{itemize}
        \item $K\in\M(\lambda)\cap\M(\mu)$.
        \item $(\lambda+\mu)(K)<\infty$.
    \end{itemize}
    So $\lambda(K)\le(\lambda+\mu)(K)<\infty$ and $\mu(K)\le(\lambda+\mu)(K)<\infty$. 

    Let $E$ be a Borel set, it suffices to show:
    \begin{displaymath}
        \lambda(E)=\inf\left\{\lambda(U):\;U\;\text{open, }U\supset E\right\}.
    \end{displaymath}
    Set $G_{k}:=[-k,k]^{n}$, $\mathbb{R}^{n}=\cp{k}{\infty}G_{k}$ 
    and $(\lambda+\mu)(G_{k})<\infty$, so 
    $\forall k$, $\exists U_{k}\supset E\cap G_{k}$ s.t. $(\lambda+\mu)(U_k\setminus (E\cap G_{k}))<\epsilon 2^{-k}$. Then 
    \begin{displaymath}
        U:=\cp{k}{\infty}U_{k}\Rightarrow (\lambda+\mu)(U\setminus E)<\epsilon
        \Rightarrow\lambda(U\setminus E)<\epsilon.
    \end{displaymath}
    So 
    \begin{displaymath}
        \lambda(E)=\inf\left\{\lambda(U):E\subset U\;,\;U\text{ open}\right\}.
    \end{displaymath}
\end{proof}
\begin{homeworkProblem}
    If $F\in NBV$, let $G(x):=|\mu_{F}|((-\infty,x])$. Prove that $|\mu_{F}|=\mu_{T_{F}}$ 
    by showing that $G=T_{F}$.
\end{homeworkProblem}
\begin{proof}
    First, we show that $T_{F}\le G$. $\forall$ partition $-\infty=x_{0}<x_{1}<\cdots<x_{n}=x$, 
    \begin{displaymath}
        \begin{array}{rl}
           G(x)&=|\mu_{F}|((-\infty,x])\\
           &=|\mu_{F}|(\cp{i}{n}((x_{i-1},x_{i}]))\\
           &=\sm{i}{n}|\mu_{F}|((x_{i-1},x_{i}])\\
           &\ge\sm{i}{n}|F(x_i)-F(x_{i-1})|.\\ 
        \end{array}
    \end{displaymath}
    Get the supermum, it's clear that $G(x)\ge T_{F}(x)$. 
    
    Then for $E=(a,b]$, $|\mu_{F}(E)|=|F(b)-F(a)|$ and 
    $\mu_{TF}(E)=T_{F}(b)-T_{F}(a)$. By Lemma 3.26, 
    $\mu_{TF}(E)\ge|\mu_{F}(E)|$. So 
    $\mu_{TF}\ge|\mu_{F}|$ for intervals. 

    Mark $\A:=\{(a,b]:-\infty\le a<b\le\infty\}$, then 
    $\mathcal{C}:=\{E\in \mathcal{B}_{\mathbb{R}}||\mu_{F}(E)|\le\mu_{T_{F}}(E)\}$ is a monotone class contains $\A$, by Lemma 2.35, $\mathcal{C}=\M(\A)=\mathcal{B}(\mathbb{R})$. 
    So $\mu_{TF}(E)\ge|\mu_{F}(E)|$ for all Borel sets.

    $\forall$ Borel set $E$, by Exercise 21, 
    \begin{displaymath}
        \begin{array}{rl}
        |\mu_{F}|(E)&=\sup\{\sm{k}{\infty}|\mu_{F}(E_{k})|\}\\
        &\le\sup\{\sm{k}{\infty}\mu_{TF}(E_{k})\}\\
        &=\mu_{TF}(E).
        \end{array}
    \end{displaymath}
    So $G\le T_{F}$, i.e. $G=T_{F}$, and 
    $\M:=\left\{E:|\mu_{F}(E)=\mu_{TF}(E)\right\}$ is a $\sigma$-algebra 
    contains $(-\infty,x]$, which means 
    $\mathcal{B}_{\mathbb{R}}\subset\M$. 
    So $|\mu_{F}|=\mu_{TF}$.
\end{proof}
\begin{homeworkProblem}
    Construct an increasing function on $\mathbb{R}$ whose set of discontinuities is $\mathbb{Q}$.
\end{homeworkProblem}
\begin{proof}
    TBD
\end{proof}
\begin{homeworkProblem}
    Let $F(x)=x^{2}\sin(x^{-1})$ and $G(x)=x^2\sin(x^{-2})$ for $x\neq 0$, 
    and $F(0)=G(0)=0$.
    \begin{enumerate}
        \item $F$ and $G$ are differentiable.
        \item $F\in BV([-1,1])$ and $G\notin BV([-1,1])$.
    \end{enumerate}
\end{homeworkProblem}
\begin{proof}
    $(1)$ For $x\neq 0$, $F,G$ are both the composition of foundamental functions, i.e. differentiable. 
    Since 
    \begin{displaymath}
        \lim_{x\rightarrow 0}\frac{F(x)}{x}=\lim_{x\rightarrow 0}x\sin\frac{1}{x}=0,
        \lim_{x\rightarrow 0}\frac{G(x)}{x}=\lim_{x\rightarrow 0}x\sin\frac{1}{x^2}=0,
    \end{displaymath}
    $F,G$ are differentiable everywhere.

    $(2)$ For $x\neq 0$, 
    \begin{displaymath}
        F'(x)=(2x-1)\sin\frac{1}{x},
    \end{displaymath}
    which is bounded on $[-1,1]\setminus\{0\}$. So $F$ is Lipschitz continuous on $[-1,1]$, which means 
    $F\in BV([-1,1])$.

    Since $G\in D[-1,1]$, 
    \begin{displaymath}
        T_{G}(1)-T_{G}(-1)=\int_{-1}^{1}|G'(t)|\dif t
        =\int_{-1}^{1}\left|2x\sin\frac{1}{x^2}-\frac{2\cos\frac{1}{x^2}}{x}\right|\dif x=\infty,
    \end{displaymath}
    which means $G\notin BV([-1,1])$.
\end{proof}

\end{document}