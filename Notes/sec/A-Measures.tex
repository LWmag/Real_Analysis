\section{Introduction}
\begin{exm}
    The \textbf{length} of an interval 
    $[a,b]\subset\mathbb{R}$ is $b-a$.
\end{exm}
\begin{exm}
    The \textbf{area} of a rectangle 
    $[a_1,b_1]\times [a_2,b_{2}]\subset\mathbb{R}^{2}$ 
    is $(b_1-a_1)(b_2-a_2)$.
\end{exm}
\begin{exm}
    The \textbf{volume} of a cube 
    $[a_1,b_1]\times [a_2,b_2]\times [a_3,b_3]\subset\mathbb{R}^3$ 
    is $(b_1-a_1)(b_2-a_2)(b_3-a_3)$.
\end{exm}
\begin{rem}
    The length, area and volume have something in common. 
    They are all derived by the \textbf{size} of 
    a subset $\Sc\subset\mathbb{R}^{n}$. 
    Then, for $X=\mathbb{R}^n$, we can give the `volume' 
    of a subset $\Sc\subset X$ with some necessary properties.  
\end{rem}
\begin{ntn}
    For a set $\Sc$, we denote 
    \begin{displaymath}
        \mathcal{P}(\Sc):=\{\M:\M\subset\Sc\}.
    \end{displaymath}
\end{ntn}
\begin{defn}
    \label{Defn:volume}
    In $\mathbb{R}^{n}$, the \textit{volume} is a function 
    $\mu:\mathcal{D}\subset 
    \mathcal{P}(\mathbb{R}^{n})\rightarrow\mathbb{R}$ satisfies:
    \begin{itemize}
        \item For a sequence $\{E_{i}\}_1^{\infty}\subset\mathcal{D}$ 
        and $\forall i\neq j$, $E_i\cap E_{j}=\emptyset$, 
        $\mu(\cp{i}{\infty}E_{i})=\sm{i}{\infty}\mu(E_i)$.
        \item If $E\subset F$, then $\mu(E)\le\mu(F)$.
        \item If there exists an isometric transformation 
        $O:\mathcal{D}\rightarrow\mathcal{D}$ such that $O(E)=F$, 
        then $\mu(E)=\mu(F)$.
        \item $\mu([0,1]^{n})=1$.
    \end{itemize}
\end{defn}
\begin{thm}
    \label{Thm:NotMeasurableSet}
    $\mathcal{D}\subsetneq\mathcal{P}(\mathbb{R}^{n})$.
\end{thm}
\begin{proof}
    We show the case $n=1$. 
    On $[0,1]$, we choose the relation $\sim$ as follows:
    \begin{displaymath}
        x\sim y\Leftrightarrow x-y\in\mathbb{Q},
    \end{displaymath}
    then define $E:=[0,1]/\sim$, and $E_{t}:=\{x:x-t\in E\}$, 
    denote $\tilde{E}:=\cup_{t\in\mathbb{Q}\cap[-1,1]}E_{t}$, 
    it's clear that 
    \begin{displaymath}
        [0,1]\subset\tilde{E}\subset[-1,2],
    \end{displaymath}
    by the second property, $1\le\mu(\tilde{E})\le 3$.

    By the constuction of $E$, $\forall t_{1}\neq t_2$, 
    $E_{t_1}\cap E_{t_2}=\emptyset$. 
    Since $E_{t_1}$ can be transformed into $E_{t_2}$ by translation, 
    by the third property, $\mu(E_{t_1})=\mu(E_{t_3})$.

    Then, by the first property, we have 
    \begin{displaymath}
        1\le\aleph_{0}\mu(E)\le 3.
    \end{displaymath}
    It's absurd. So we can't define the volume of $E$.
\end{proof}
\begin{rem}
    By Theorem \ref{Thm:NotMeasurableSet}, 
    we should discuss the properties of $\mathcal{D}$. 
    It's the motivation of introducing 
    $\sigma$-algebras.
\end{rem}
\section{$\sigma-$algebras}
\begin{defn}
    \label{Defn:SigmaAlg}
    Given a set $X$, $\mathcal{A}\subset\mathcal{P}(X)$ 
    is an \textit{algebra} if 
    \begin{itemize}
        \item $\mathcal{A}\neq\emptyset$.
        \item If $E_1,E_2\in\mathcal{A}$, $E_1\cup E_2\in\mathcal{A}$.
        \item If $E\in\mathcal{A}$, $E^{c}\in\mathcal{A}$.
    \end{itemize}
    $\mathcal{A}$ is a \textit{$\sigma$-algebra} if 
    $\mathcal{A}$ is an algebra and 
    $\{E_{k}\}_1^{\infty}\subset\mathcal{A}$ yields 
    $\cp{k}{\infty}E_k\in\mathcal{A}$.
\end{defn}
\begin{rem}
    By Definition \ref{Defn:SigmaAlg}, an algebra is closed 
    under finite union and supplementation, 
    and an $\sigma$-algebra is closed under 
    countable union and supplementation.
\end{rem}
\begin{exc}
    If $\mathcal{A}\subset\mathcal{P}(X)$ is an algebra, 
    show $\emptyset,X\in\mathcal{A}$.
\end{exc}
\begin{proof}
    If $\mathcal{A}$ is an algebra, $\mathcal{A}\neq\emptyset$. 
    If $E\in\mathcal{A}$, we have $\emptyset=E\cap E^c\in\mathcal{A}$, 
    $X=E\cup E^c\in\mathcal{A}$.
\end{proof}
\begin{exc}
    If $\mathcal{A}$ is a $\sigma-$algebra and 
    $\{E_{k}\}_{1}^{\infty}\subset\mathcal{A}$, show that 
    $\cap_{k=1}^{\infty}E_{k}\in\mathcal{A}$.
\end{exc}
\begin{proof}
    If $\mathcal{A}$ is a $\sigma-$algebra and 
    $\{E_{k}\}_{1}^{\infty}\subset\mathcal{A}$, 
    $\{E_{k}^c\}_{1}^{\infty}\subset\mathcal{A}$,
    which implies that $\cup_{k=1}^{\infty}E_{k}^c\in\mathcal{A}$.
    We can derive that $\cap_{k=1}^{\infty}E_{k}=(\cup_{k=1}^{\infty}E_{k}^c)^c\in\mathcal{A}$.
\end{proof}
\begin{lem}
    \label{Lem:IntersecOfSigmaAlg}
    The intersection of any family of $\sigma$-algebras is a 
    $\sigma$-algebra.
\end{lem}
\begin{exc}
    Prove Lemma \ref{Lem:IntersecOfSigmaAlg}.
\end{exc}
\begin{proof}
    If $\mathcal{A}_{i}\subset\mathcal{P}(X)$, $i\in I$, 
    where $I$ is index set. We can notice that $\cap_{i \in I}\mathcal{A}_{i}\neq\emptyset$,
    since $\{\emptyset,X\}\subset\cap_{i \in I}\mathcal{A}_{i}$.
    
    If $E\in\cap_{i\in I}\mathcal{A}_{i}$ 
    $\Rightarrow$ $\forall i\in I$, 
    $E\in\mathcal{A}_{i}$. 
    By the third property of $\sigma$-algebras, $\forall i\in I$, 
    $E^c\in\mathcal{A}_{i}$, 
    which implies $E^c\in\cap_{i \in I}\mathcal{A}_{i}$.
    
    If $\{E_{k}\}_1^{\infty}\subset\cap_{i \in I}\mathcal{A}_{i}$ 
    $\Rightarrow$ 
    $\forall i\in I$, $\{E_{k}\}_1^{\infty}\in\mathcal{A}_{i}$. 
    By the second property of $\sigma$-algebras, $\forall i\in I$, 
    $\cp{k}{\infty}E_{k}\in\mathcal{A}_{i}$, 
    which implies $\cp{k}{\infty}E_k\in\cap_{i\in I}\mathcal{A}_{i}$.
\end{proof}
\begin{defn}
\label{Defn:GeneratedSigmaAlg}
For $\mathcal{E}\subset\mathcal{P}(X)$, $\mathcal{M}(\mathcal{E})$ 
is called the \textit{$\sigma$-algebra generated by $\mathcal{E}$} 
if $\mathcal{M}(\mathcal{E})$ is the intersection of all $\sigma$-algebras 
containing $\mathcal{E}$
\end{defn}
\begin{rem}
    $\mathcal{M}(\mathcal{E})$ is the smalled $\sigma$-algebra 
    containing $\mathcal{E}$.
\end{rem}
\begin{lem}
    \label{Lem:GeneratedSigmaAlgLem}
    If $\mathcal{E}\subset\mathcal{M}(\mathcal{F})$, 
    then $\M(\mathcal{E})\subset\M(\mathcal{F})$.
\end{lem}
\begin{exc}
    Prove Lemma \ref{Lem:GeneratedSigmaAlgLem}.
\end{exc}
\begin{proof}
    $\mathcal{M}(\mathcal{F})$ is a $\sigma$-algebra containing $\mathcal{E}$; it therefore contains $\mathcal{M}(\mathcal{E})$.
\end{proof}
\begin{defn}
    If $(X,\tau)$ is a topological space, then 
    the \textit{Borel $\sigma$-algebra} $\mathcal{B}_{X}$ is the 
    $\sigma$-algebra generated by the family of open sets 
    in $X$. 
    And the sets in $\mathcal{B}_{X}$ is called \textit{Borel sets}.
\end{defn}
\begin{exc}
    \label{Exer:BRGeneratedsets}
    $\mathcal{B}_{\mathbb{R}}$ is generated by each of the following:
    \begin{itemize}
        \item $\mathcal{E}_{1}:=\{(a,b):a<b\}$,
        \item $\mathcal{E}_{2}:=\{(a,b]:a<b\}$,
        \item $\mathcal{E}_{3}:=\{[a,b):a<b\}$,
        \item $\mathcal{E}_{4}:=\{[a,b]:a<b\}$.
    \end{itemize}
\end{exc}
\begin{proof}
    The elements of $\mathcal{E}_{j}$ for $j\neq3,4$ are open or closed,
    and the elements of $\mathcal{E}_{3}$ and $\mathcal{E}_{4}$ can be expressed
    by a countable intersection of open sets, for example, $(a,b]=\cap_{1}^{\infty}(a,b+n^{-1})$.
    All of these are Borel sets, so by Lemma \ref{Lem:GeneratedSigmaAlgLem},
    $\mathcal{M}(\mathcal{E}_{j})\subset\mathcal{B}_{\mathbb{R}}$ for all $j$.On the other hand,
    every open set in $\mathbb{R}$ is a countable union of open intervals,
    so by Lemma \ref{Lem:GeneratedSigmaAlgLem} again, $\mathcal{B}_{\mathbb{R}}\subset\mathcal{M}(\mathcal{E}_{1})$. 
    That $\mathcal{M}(\mathcal{E}_{j})\subset\mathcal{B}_{\mathbb{R}}$ for $j\geq2$ can be established
    by showing that all open intervals lie in $\mathcal{M}(\mathcal{E}_{j})$ and applying Lemma \ref{Lem:GeneratedSigmaAlgLem}.
    For example, $(a,b)=\cup_{1}^{\infty}(a,b-n^{-1}]\in\mathcal{M}(\mathcal{E}_{2})$.
\end{proof}
\begin{defn}
    \label{Defn:ProdSigmaAlg}
    Let $\{X_{\alpha}\}_{\alpha\in A}$ be an indexed collection 
    of nonempty sets, $X:=\prod_{\alpha\in A}X_{\alpha}$, 
    $\M_{\alpha}$ is a $\sigma$-algebra on $X_{\alpha}$, 
    and $\pi_{\alpha}:X\rightarrow X_{\alpha}$ is a projection, 
    the \textit{product $\sigma$-algebra} 
    \begin{displaymath}
        \otimes_{\alpha\in A}\M_{\alpha}:=\M\left(
            \left\{\pi_{\alpha}^{-1}(E_{\alpha}):
            E_{\alpha}\in\M_{\alpha},\alpha\in A\right\}
        \right).
    \end{displaymath}
\end{defn}
\begin{rem}
    The product $\sigma$-algebra is the 
    $\sigma$-algebra generated by a series of 
    $\sigma$-algebras.
\end{rem}
\begin{prop}
    \label{Prop:GeneraterOfProdAlg}
    $\otimes_{k=1}^{\infty}\M_{k}$ is generated by 
    $\{\prod_{k=1}^{\infty}E_{k}:E_{k}\subset\M_{k}\}$.
\end{prop}
\begin{proof}
    On one hand, 
    \begin{displaymath}
    \pi_{k}^{-1}(E_{k})=(X_{1},\ldots,X_{k-1},E_{k},X_{k+1},\ldots), 
    \end{displaymath}
    on the other hand, 
    \begin{displaymath}
        \prod_{k=1}^{\infty}E_{k}=\cap_{k=1}^{\infty}\pi_{k}^{-1}(E_k),
    \end{displaymath}
    then the 
    result follows from 
    Lemma \ref{Lem:GeneratedSigmaAlgLem}.
\end{proof}
\begin{exc}
    Show that $\mathcal{B}_{\mathbb{R}^n}
    =\otimes_{1}^{n}\mathcal{B}_{\mathbb{R}}$.
\end{exc}
\begin{proof}
    $\mathcal{B}_{\mathbb{R}^n}$ is generated by  $\{\prod_{k=1}^{n}(a_k,b_k):a_k<b_k \quad and \quad a_k,b_k\in\mathbb{R}\}$.
    By Proposition \ref{Prop:GeneraterOfProdAlg} and Exercise \ref{Exer:BRGeneratedsets},
    $\otimes_{1}^{n}\mathcal{B}_{\mathbb{R}}$ is generated by $\{\prod_{k=1}^{n}(a_k,b_k):a_k<b_k \quad and \quad a_k,b_k\in\mathbb{R}\}$ as well,
    so $\mathcal{B}_{\mathbb{R}^n}=\otimes_{1}^{n}\mathcal{B}_{\mathbb{R}}$.
\end{proof}
\section{Measures}
\begin{defn}[Measure]
    \label{Defn:Measure}
    A \textit{measure} on $(X,\M)$ is 
    a function $\mu:\M\rightarrow[0,\infty]$ such that 
    \begin{itemize}
        \item $\mu(\emptyset)=0$,
        \item If $\{E_{j}\}_{1}^{\infty}$ pairwise disjoint, 
        $\mu(\cp{j}{\infty}E_j)=\sm{j}{\infty}\mu(E_j)$.
    \end{itemize}
\end{defn}
\begin{rem}
    Definition \ref{Defn:Measure} is a 
    generalization of Definition \ref{Defn:volume}.
\end{rem}
\begin{rem}
    $\M$ is a $\sigma$-algebra, so $\mu(\cp{j}{\infty}E_{j})$ 
    is well-defined. 
    It means that $\M$ \textbf{must} be a $\sigma$-algebra.
\end{rem}
\begin{defn}
    \label{Defn:MeasureWithFinite}
    $(X,\M)$ is a \textit{measurable space}, the sets in $\M$ are 
    \textit{measurable sets}, and $(X,\M,\mu)$ is a 
    \textit{measurable space}.
    \begin{itemize}
        \item If $\mu(X)<\infty$, $\mu$ is \textit{finite}.
        \item If $X=\cp{j}{\infty}E_{j}$, $E_{j}\in\M$ and 
        $\forall j$, $\mu(E_j)<\infty$, $\mu$ is 
        \textit{$\sigma$-finite}.
        \item If $\forall E\in\M$, $\mu(E)=\infty$, 
        $\exists F\subset E$ s.t. 
        $0<\mu(F)<\infty$, then $\mu$ is \textit{semi-finite}.
    \end{itemize}
\end{defn}
\begin{rem}
    If $\mu$ is volume in $\mathbb{R}^{n}$, then $\mu(X)=\infty$ 
    means $X$ is unbounded. 
    So, it's essential to show a measure is 
    finite or not.
\end{rem}
\begin{exm}
    \label{Exm:CountingAndDirac}
    Given a function $f:X\rightarrow [0,\infty)$, 
    for $E\in\mathcal{P}(X)$, 
    \begin{displaymath}
        \mu_{f}(E):=\sum_{x\in E}f(x)
    \end{displaymath}
    is a measure on $X$. 
    If $f\equiv 1$, $\mu$ is called \textit{counting measure}. 
    If $f(x)=\left\{\begin{array}{rl}
        1,x=x_0,\\
        0,x\neq x_0,\\
    \end{array}\right.$
    then $\mu$ is called \textit{Dirac delta measure} on $x_{0}$, 
    marked as $\delta_{x_0}$.
\end{exm}
\begin{exc}
    Exclaim the meanings of 
    counting measure and Dirac delta measure 
    by some examples in real world.
\end{exc}
\begin{proof}
    Counting measure is used to describe the number of elements in a set. 
    Nevertheless, Dirac delta measure is a generalized function that is concentrated at a single point.
    For instance, if there are 5 apples and 3 oranges, the counting measure of the set “fruits in the basket” would be 8.
    For the Dirac delta measure,Think about a ruler. The Dirac delta measure at a specific point on the ruler would 
    represent the mass or density concentrated at that point. 
\end{proof}
\begin{thm}
    \label{Thm:PropertiesOfMeasure}
    Given a measurable space $(X,\M,\mu)$, 
    \begin{enumerate}[(a)]
        \item \textit{Monotonicity:} If $E,F\in\M$ and $E\subset F$, 
        then $\mu(E)\le\mu(F)$. 
        \item \textit{Subadditivity:} 
        If $\{E_{j}\}_{1}^{\infty}\subset\M$, 
        then $\mu\left(\cp{j}{\infty}E_j\right)\le\sm{j}{\infty}\mu(E_j)$.
        \item \textit{Continuity from below:} 
        If $\{E_j\}_{1}^{\infty}\subset\M$, 
        $E_1\subset E_2\subset\ldots$, then 
        $\mu(\cp{i}{\infty}E_{i})=\lim_{n\rightarrow\infty}\mu(E_n)$.
        \item \textit{Continuity from above:}
        If $\{E_j\}_{1}^{\infty}\subset\M$, 
        $E_1\supset E_2\supset\ldots$ and $\mu(E_{1})<\infty$, 
        then 
        $\mu(\cap_{i=1}^{\infty}E_{i})=\lim_{n\rightarrow\infty}\mu(E_n)$.
    \end{enumerate} 
\end{thm}
\begin{proof}
    (a) Since $F=E\cup (F\setminus E)$, 
    by Definition \ref{Defn:Measure}, 
    \begin{displaymath}
        \mu(F)=\mu(E)+\mu(F\setminus E)\ge\mu(E).
    \end{displaymath}
    (b) Mark 
    \begin{displaymath}
        F_{1}:=E_{1},\quad F_{k}:=E_{k}\setminus\cp{i}{k-1}E_{i},
    \end{displaymath}
    then $\{F_{k}\}$ are pairwise disjoint sets, and 
    $\cp{i}{\infty}E_{i}=\cp{i}{\infty}F_{i}$.  
    Then by Definition \ref{Defn:Measure}, 
    \begin{equation}
        \label{Equ:SubAddi1}
        \mu\left(\cp{i}{\infty}E_i\right)
        =\mu\left(\cp{i}{\infty}F_{i}\right)=\sm{i}{\infty}\mu(F_i).
    \end{equation}
    By the definition of $F_{k}$, 
    $\forall k\ge 1$, $F_{k}\subset E_{k}$, so 
    \begin{equation}
        \label{Equ:SubAddi2}
        \sm{i}{\infty}\mu(F_i)\le\sm{i}{\infty}(E_i).
    \end{equation}
    By \eqref{Equ:SubAddi1} and \eqref{Equ:SubAddi2}, 
    (b) is true. 

    (c) We set $E_0:=\emptyset$, since $\{E_{k}\}_{1}^{\infty}$ 
    increasing, $F_{k}:=E_{k}\setminus\left(\cp{i}{k-1}E_i\right)
    =E_{k}\setminus E_{k-1}$. 
    Then by Definition \ref{Defn:Measure}:
    \begin{displaymath}
        \begin{array}{rl}
        \mu(\cp{j}{\infty}E_j)=&\mu(\cp{j}{\infty}F_j)\\
        =&\sm{j}{\infty}\mu(F_j)\\
        =&\lim_{n\rightarrow\infty}\sm{j}{n}\mu(E_j\setminus E_{j-1})\\
        =&\lim_{n\rightarrow\infty}\mu(E_n).\\
        \end{array}
    \end{displaymath}

    (d) Mark $G_{j}:=E_{1}\setminus E_{j}$, 
    by Definition \ref{Defn:Measure} and $(c)$, 
    \begin{equation}
        \label{Equ:ContinuousFromAbove}
        \begin{array}{rl}
            \mu(E_1)&=\mu(\cap_{i=1}^{\infty}E_{i})
            +\lim_{j\rightarrow\infty}\mu(F_j)\\
            &=\mu(\cap_{i=1}^{\infty}E_{i})+\lim_{j\rightarrow\infty}
            (\mu(E_1)-\mu(E_{j})).
        \end{array}
    \end{equation} 
    Since $\mu(E_1)<\infty$, \eqref{Equ:ContinuousFromAbove} 
    means $\lim_{j\rightarrow\infty}\mu(E_{j})
    =\mu(\cap_{i=1}^{\infty}E_{i})$. 
\end{proof}
\begin{rem}
    The item (b) omit the condition 
    for $\{E_{k}\}$ disjoint. 
\end{rem}
\begin{rem}
    In this chapter, 
    unless otherwise specified, 
    we consider the measurable space 
    $(X,\M,\mu)$.
\end{rem}
\begin{exc}
    Give an example to show that 
    if $\mu(E_1)=\infty$, 
    (d) in Theorem \ref{Thm:PropertiesOfMeasure} may fail. 
\end{exc}
\begin{proof}
    For instance, consider the Lebesgue measure on the set of real numbers. 
    Let's assume $E_n=[n,+\infty)$,
    a sequence of intervals starting from $n$.
    These sets are decreasing, as 
    $E_1\supset E_2\supset\ldots$ and $\mu(E_n)=\infty$ for all $n$.
    Therefore, $\lim_{n\rightarrow\infty}\mu(E_n)=\infty$.
    However, $\cap_{i=1}^{\infty}E_{i}=\emptyset$, 
    $\mu(\cap_{i=1}^{\infty}E_{i})=0\neq\infty$.
    In this case, the theorem does not hold.
\end{proof}
\begin{defn}
    \label{Defn:NullSet}
    If $E\in\M$ satisfies $\mu(E)=0$, $E$ 
    is said to be a \textit{null set.}
\end{defn}
\begin{defn}
    \label{Defn:Trueae}
    If $P(x)$ is a statement which is true 
    for all $x$ outside of a null set $E$, 
    we say $P(x)$ is \textit{true a.e.}
\end{defn}
\begin{exc}
    If $\mu(E)=0$ and $F\subset E$, 
    can we conclude that $\mu(F)=0$? 
\end{exc}
\begin{proof}
    By Monotonicity in Theorem \ref{Thm:PropertiesOfMeasure},
    we can conclude that $\mu(F)=0$,
    since $0\leq\mu(F)\leq\mu(E)=0$.
    However, it may not be true that $F\in\M$. 
\end{proof}
\begin{defn}
    \label{Defn:CompleteMeas}
    If all subsets of any null sets are 
    in $\M$, we say $\mu$ is \textit{complete}.
\end{defn}
\begin{exm}
    The volume in Definition \ref{Defn:volume} 
    is a complete measure.
\end{exm}
\begin{thm}
    \label{Thm:CompletationForMeas}
    Let $\mathcal{N}:=\{N\in\M,\mu(N)=0\}$, 
    $\overline{\M}:=\{E\cup F:E\in\M,\; 
    F\subset N\text{ for some }N\in\mathcal{N}\}$, 
    then $\overline{\M}$ is a $\sigma$-algebra, 
    and there is a unique 
    extension $\bar{\mu}$ of $\mu$, 
    which is complete on $\overline{\M}$.
\end{thm}
\begin{exc}
    Prove Theorem \ref{Thm:CompletationForMeas} 
    (Hint: $\bar{\mu}(E\cup F):=\mu(E)$).
\end{exc}
\begin{proof}
    Since $\mathcal{N}$ and $\M$ are closed under countable unions,
    so is $\overline{\M}$.
    If $E\cup F\in\overline{\M}$ where $E\in\M$ and $F\subset N\in\mathcal{N}$,
    we can assume that $E\cap N=\emptyset$ 
    (otherwise, replace $F$ and $N$ by $F\setminus E$ and $N\setminus E$).
    Then $E\cup F=(E\cup N)\cap(N^c\cup F)$,
    so $(E\cup F)^c=(E\cup N)^c\cup(N\setminus F)$.
    But $(E\cup F)^c\in\M$ and $N\setminus F\subset N$, so that $(E\cup F)^c\in\overline{\M}$.
    Thus $\overline{\M}$ is a $\sigma$-algebra.

    If $E\cup F\in\overline{\M}$ as above, we set $\bar{\mu}(E\cup F):=\mu(E)$.
    This well defined, since if $E_1\cup F_1=E_2\cup F_2$ where $F_j\subset N_j$,
    then $E_1\subset E_2\cup N_2$ and so $\mu{E_1}\leq\mu{E_2}+\mu{N_2}=\mu{E_2}$,
    and likewise $\mu{E_2}\leq\mu{E_1}$.

    Firstly, we prove $\bar{\mu}$ is a measure. $\bar{\mu}(\emptyset)=\mu(\emptyset)=0$.
    If $\{E_j\cup F_j\}_{1}^{\infty}$ pairwise disjoint, 
    $\bar{\mu}(\cup_{j=1}^{\infty}(E_j\cup F_j))=\mu(\cup_{j=1}^{\infty}E_j)=\sum_{j=1}^{\infty}\mu{E_j}=\sum_{j=1}^{\infty}\bar{\mu}(E_j\cup F_j)$.
    Thus $\bar{\mu}$ is a measure.

    Secondly, we prove $\bar{\mu}$ is complete. If $\bar{\mu}(E)=0$, $E\in\mathcal{N}$.
    Any $F\subset E$, $\bar{\mu}(F)=\bar{\mu}(\emptyset\cup F)=\mu(\emptyset)=0$, 
    so $\bar{\mu}$ is complete.

    Finally, we prove $\bar{\mu}$ is the only complete measure on $\overline{\M}$ that extends $\mu$.
    If there exists another complete measure $\mu^*$ is a extension of $\mu$ on $\overline{\M}$.
    By monotonicity of $\mu^*$, $\bar{\mu}(E\cup F)=\mu(E)=\mu^*(E)\leq\mu^*(E\cup F)$.
    By completeness and subadditivity of $\mu^*$, 
    $\mu^*(E\cup F)\leq\mu^*(E)+\mu^*(F)=\mu^*(E)=\mu(E)=\bar{\mu}(E\cup F)$.
    We derive that for all $E\cup F\in\overline{\M}$, $\mu^*(E\cup F)=\bar{\mu}(E\cup F)$, 
    so $\mu^*=\bar{\mu}$.
\end{proof}
\section{Outer Measures}
\begin{rem}
    By Theorem \ref{Thm:NotMeasurableSet}, there exists 
    some sets $X\subset\mathbb{R}^{n}$ such that we can't define the 
    volume of $X$, but, on which condition can we define the volume of $X$?
    In this section, we define a map 
    $\mu^{*}:\mathcal{P}(X)\rightarrow [0,\infty]$, derive the 
    $\mu^{*}$-measurable sets $\M\subset \mathcal{P}(X)$, and 
    deduce a measure on $\M$. 
\end{rem}
\subsection{Definition}
\begin{defn}
    Given a set $X$, $\mu^{*}:\mathcal{P}(X)\rightarrow[0,\infty]$ 
    is an \textit{outer measure} if 
    \begin{itemize}
        \item $\mu^{*}(\emptyset)=0$,
        \item $\mu^{*}(A)\le\mu^{*}(B)$ if $A\subset B$, 
        \item $\mu^{*}(\cp{j}{\infty}A_{j})\le\sm{j}{\infty}\mu^{*}(A_{j})$.
    \end{itemize}
\end{defn}
\begin{rem}
    The outer measure only need 
    the subadditivity, rather than countable additivity. 
    So, we can derive an outer measure from the volume of unit cubes.
\end{rem}
\begin{prop}
    \label{Prop:OuterMeasureFromFunction}
    Given $\mathcal{E}\subset\mathcal{P}(X)$ and 
    $\rho:\mathcal{E}\rightarrow[0,\infty]$ such that 
    $\emptyset\in\mathcal{E}$, $x\in\mathcal{E}$ 
    and $\rho(\emptyset)=0$, for $A\subset X$, 
    \begin{equation}
        \label{Equ:DerivedOuterMeas}
        \mu^{*}(A):=\inf_{E_j\in\mathcal{E},A\subset\cp{j}{\infty}E_j}
        \sm{j}{\infty}\rho(E_j),
    \end{equation}
    then $\mu^{*}$ is an outer measure.
\end{prop}
\begin{rem}
    If $X=\mathbb{R}^{n}$, $\mathcal{E}$ marks all the 
    elementary cubes $\prod_{i=1}^{n}(a_i,b_i]$, 
    and 
    \begin{displaymath}
        \rho\left(\prod_{i=1}^{n}(a_i,b_i]\right)
        :=\prod_{i=1}^{n}(b_i-a_i),
    \end{displaymath}
    then 
    \eqref{Equ:DerivedOuterMeas} is the Lebesgue outer measure 
    on $\mathbb{R}^{n}$.
\end{rem}
\begin{exc}
    Prove Proposition \ref{Prop:OuterMeasureFromFunction}.
\end{exc}
\begin{proof}
    We only need to verify the three conditions of outer measure.

    Firstly, $\mu^{*}(\emptyset)=\inf_{E_j\in\mathcal{E},\emptyset\subset\cp{j}{\infty}E_j}
    \sm{j}{\infty}\rho(E_j)=0$ since each $E_j=\emptyset$.

    Secondly, if  $A\subset B$, $F_j\in\mathcal{E},A\subset B\subset\cp{j}{\infty}F_j$,
    we can derive that $\mu^{*}(A)=\inf_{E_j\in\mathcal{E},A\subset\cp{j}{\infty}E_j}
    \sm{j}{\infty}\rho(E_j)\leq \sm{j}{\infty}\rho(F_j)$, which means 
    $\mu^{*}(A)\leq\mu^{*}(B)$.

    Finally, if $\{A_j\}_{j=1}^{\infty}\in\mathcal{P}(X)$,
    for each $j$, $\mu^{*}(A_j)=\inf_{E_i^j\in\mathcal{E},A_j\subset\cp{i}{\infty}E_i^j}
    \sm{j}{\infty}\rho(E_i^j)$. If $E_i^j\in\mathcal{E},A_j\subset\cp{i}{\infty}E_i^j$,
    then $\cup_{j=1}^{\infty}A_j\subset\cup_{j=1}^{\infty}\cup_{i=1}^{\infty}E_i^j$,
    which means $\mu^{*}(\cup_{j=1}^{\infty}A_j)\leq\sum_{j=1}^{\infty}\mu^{*}(A_j)$.
\end{proof}
\subsection{From outer measure to measure}
In this section, 
we should construct a measurable space 
$(X,\M,\mu)$ from the outer measure $\mu^{*}$. 
\begin{defn}
    \label{Defn:MuStarMeasurable}
    Given an outward measure $\mu^{*}$ on $\mathcal{P}(X)$, 
    a set $A\subset X$ is called \textit{$\mu^{*}$-measurable} 
    if 
    \begin{equation}
        \label{Equ:MeasurableCondition}
        \forall E\subset X,\quad \mu^{*}(E)=\mu^{*}(E\cap A)
        +\mu^{*}(E\cap A^c).
    \end{equation}
\end{defn}
\begin{thm}[Caratheodory]
    \label{Thm:CaratheodoryThm}
    Give an outer measure $\mu^{*}$ on $\mathcal{P}(X)$, 
    $\M:=\{\Sc\subset X:\Sc\text{ is }\mu^*-\text{measurable}\}$, 
    then $\M$ is a $\sigma$-algebra and 
    $\mu^{*}|_{\M}$ is a complete measure. 
\end{thm}
\begin{proof}
    We divide this proof into four steps.
    
    First, by equation \eqref{Equ:MeasurableCondition}, it's straightforward 
    that $\M$ is closed under complementation. 

    Second, we show that $\M$ is closed under finite union. 
    If $A,B\in\M$, by \eqref{Equ:MeasurableCondition}, 
    $\forall E\subset X$ we have:
    \begin{equation*}
        \label{Equ:ABMeasurable}
        \begin{array}{rl}
            \mu^{*}(E)&=\mu^{*}(E\cap A)+\mu^{*}(E\cap A^c),\\
            \mu^{*}(E)&=\mu^{*}(E\cap B)+\mu^{*}(E\cap B^c).\\
        \end{array}
    \end{equation*}
    Then:
    \begin{equation}
        \label{Equ:ExpressMuStarE}
        \begin{array}{rl}
        &\mu^{*}(E)=\mu^{*}(E\cap A)+\mu^{*}(E\cap A^c)\\
        =&\mu^{*}(E\cap A\cap B)+\mu^{*}(E\cap A\cap B^c)\\
        +&\mu^{*}(E\cap A^c\cap B)+\mu^{*}(E\cap A^c\cap B^c).
        \end{array}
    \end{equation}
    As $A\cup B=(A\cap B)\cup(A\cap B^c)\cup(A^c\cap B)$, 
    from the subadditivity, 
    \begin{equation}
        \label{Equ:ExpressionsForEcapAcupB}
        \begin{array}{rl}
        &\mu^{*}(E\cap A\cap B)+\mu^{*}(E\cap A\cap B^c)\\
        +&\mu^{*}(E\cap A^c\cap B)\ge\mu^{*}(E\cap(A\cup B)).\\
        \end{array}
    \end{equation}
    From \eqref{Equ:ExpressMuStarE} and 
    \eqref{Equ:ExpressionsForEcapAcupB}, 
    \begin{displaymath}
        \mu^{*}(E)\ge\mu^{*}(E\cap (A\cup B))+\mu^{*}(E\cap (A\cup B)^c).
    \end{displaymath}
    And from the subadditivity, we derive $A\cup B\in\M$, 
    which means $\M$ is an algebra. If $A\cap B=\emptyset$, 
    \begin{displaymath}
        \mu^{*}(A\cup B)=\mu^{*}(A\cup B\cap B)+
        \mu^{*}(A\cup B\cap B^c)=\mu^{*}(A)+\mu^{*}(B).
    \end{displaymath}

    Third, we show that $\M$ is closed under countable union. 
    Set $B:=\cp{i}{\infty}A_{i}$, since $\mathcal{M}$ is an algebra, 
    WLOG, we assume $\{A_{j}\}$ disjoint. Mark $B_{n}:=\cp{j}{n}A_j$, 
    by \eqref{Equ:MeasurableCondition}, 
    \begin{displaymath}
        \begin{array}{rl}
        \mu^{*}(E\cap B_n)&=\mu^{*}(E\cap B_{n}\cap A_{n})+
        \mu^{*}(E\cap B_n\cap A_n^c)\\
        &=\mu^{*}(E\cap A_n)+\mu^{*}(E\cap B_{n-1}).
        \end{array}
    \end{displaymath}
    So $\mu^{*}(E\cap B_n)=\sm{j}{n}\mu^{*}(E\cap A_j)$. Since $\M$ 
    be an algebra, $B_{n}$ is $\mu^{*}$-countable, then:
    \begin{displaymath}
        \begin{array}{rl}
            \mu^{*}(E)&=\mu^{*}(E\cap B_n)+\mu^{*}(E\cap B_{n}^{c})\\
            &\ge\sm{j}{n}\mu^{*}(E\cap A_j)+\mu^{*}(E\cap B^c).\\
        \end{array}
    \end{displaymath}
    Set $n\rightarrow\infty$, it yields 
    \begin{displaymath}
        \mu^{*}(E)\ge\sm{j}{\infty}\mu^{*}(E\cap A_j)+\mu^{*}(E\cap B^c)
        \ge\mu^{*}(E\cap B)+\mu^{*}(E\cap B^c),
    \end{displaymath}
    so $B\in\M$. It means $\M$ is a $\sigma$-algebra. 
    The properties of $\mu^{*}$ are remained as exercise.
\end{proof}
\begin{exc}
    Complete the proof of Theorem \ref{Thm:CaratheodoryThm}, 
    you should show $\mu^{*}$ is countable additive 
    and complete on $\M$.
\end{exc}
\begin{proof}
    If we take $E=B$, $\mu^{*}(B)\ge\sm{j}{\infty}\mu^{*}(B\cap A_j)+\mu^{*}(B\cap B^c)
    \ge\mu^{*}(B\cap B)+\mu^{*}(B\cap B^c)\ge\mu^{*}(B)$, by $B\cap B^c=\emptyset$ and 
    $\mu^{*}(\emptyset)=0$, we can get $\mu^{*}(B)=\sm{j}{\infty}\mu^{*}(A_j)$, so 
    $\mu^{*}$ is countably additive on $\M$.

    If $\mu^{*}(A)=0$, for any $E\subset X$, we have $\mu^{*}(E)\leq\mu^{*}(E\cap A)+
    \mu^{*}(E\cap A^c)=\mu^{*}(E\cap A^c)\leq \mu^{*}(E)$ so that $A\in\M$, Therefore
    $\mu^{*}$ is complete on $\M$.
\end{proof}
\begin{rem}
    By Theorem \ref{Thm:CaratheodoryThm}, we can derive the conditions 
    of \textit{measurable set} by outer measure, and 
    define the \textit{measure} of each measurable sets.
\end{rem}
\subsection{Premeasure and its extension}
In this section, we define a \textit{premeasure} on an \textbf{algebra} 
(not necessary $\sigma$-algebra), 
then extend it to $\M:=\M(\mathcal{A})$.
\begin{defn}[premeasure]
    \label{Defn:Premeasure}
    $\mathcal{A}\subset\mathcal{P}(X)$ is an algebra, 
    $\mu_{0}:\mathcal{A}\rightarrow[0,\infty]$ is called 
    a \textit{premeasure} if:
    \begin{itemize}
        \item $\mu_{0}(\emptyset)=0$,
        \item If $\{A_j\}$ disjoint and 
        $\cp{j}{\infty}A_j\in\mathcal{A}$, then 
        \begin{displaymath}
            \mu_0(\cp{j}{\infty}A_j)=\sm{j}{\infty}\mu_{0}(A_j).
        \end{displaymath}
    \end{itemize}
\end{defn}
\begin{rem}
    If $\mathcal{A}$ isn't a $\sigma$-algebra, 
    $\cp{j}{\infty}A_{j}$ may not in $\mathcal{A}$. 
    The premeasure only keeps countable additivity 
    when $\cp{j}{\infty}A_j\in\mathcal{A}$.
\end{rem}
\begin{ntn}
    By Proposition \ref{Prop:OuterMeasureFromFunction}, 
    a premeasure $\mu_{0}$ can induce 
    an outer measure on $\mathcal{P}(X)$. 
    In the following part of this section, 
    we denote $\mu^{*}$ to represent the induced measure 
    of $\mu_{0}$, i.e. 
    \begin{displaymath}
        \mu^{*}(E):=\inf_{A_{j\in\mathcal{A}},E\subset\cp{j}{\infty}A_j}
        \sm{j}{\infty}\mu_{0}(A_j).
    \end{displaymath}
\end{ntn}
\begin{prop}
    \label{Prop:PremeasureAndOuterMeas}
    Given $\mu_{0}$ be a premeasure on $\mathcal{A}$, 
    $\mu^{*}$ is the outer measure induced by $\mu_{0}$, then 
    \begin{enumerate}
        \item $\forall A\in\mathcal{A}$, $\mu_{0}(A)=\mu^{*}(A)$.
        \item $\forall A\in\mathcal{A}$, $A$ is a $\mu^{*}$-measurable 
        set.
    \end{enumerate}
\end{prop}
\begin{exc}
    Prove Proposition \ref{Prop:PremeasureAndOuterMeas}.
\end{exc}
\begin{proof}
    (a)Suppose $E\in\mathcal{A}$. If $E\subset\cup_1^{\infty}A_j$ with $A_j\in\mathcal{A}$,
    let $B_n=E\cap(A_n\setminus\cup_1^{n-1}A_j)$. Then the $B_n$'s are disjoint members of $\mathcal{A}$
    whose union is $E$, so $\mu_{0}(E)=\sum_1^{\infty}\mu_{0}(B_j)\leq\sum_1^{\infty}\mu_{0}(A_j)$.
    It follows that $\mu_{0}(E)\leq\mu^*(E)$, and the reverse inequality is obvious since           
    $E\subset\cup_{1}^{\infty}A_j$ where $A_1=E$ and $A_j=\emptyset$ for $j>1$.

    (b)If $A\in\mathcal{A}$, $E\subset X$, and $\epsilon>0$, there is a sequence
    $\{B_j\}_{1}^{\infty}\subset\mathcal{A}$ with $E\subset\cup_{1}^{\infty}B_j$
    and $\sum_{1}^{\infty}\mu_{0}(B_j)\leq\mu^*(E)+\epsilon$. Since $\mu_0$ is additive
    on $\mathcal{A}$, 
    $\mu^*(E)+\epsilon\geq\sum_{1}^{\infty}\mu_{0}(B_j\cap A)+\sum_{1}^{\infty}\mu_{0}(B_j\cap A^c)
    \geq\mu^*(E\cap A)+\mu^*(E\cap A^c)$. Since $\epsilon$ is arbitrary, $A$ is $\mu^*$-measurable.
\end{proof}
\begin{thm}
    \label{Thm:PremeasureExtension}
    Suppose $\mathcal{A}$ is an algebra, $\mu_{0}$ is a premeasure on 
    $\mathcal{A}$, $\M:=\M(\mathcal{A})$, then:
    \begin{enumerate}
        \item There exists a measure $\mu:\M\rightarrow[0,\infty]$ 
        s.t. $\mu=\mu^{*}|_{\mathcal{M}}$ and $\mu|_{\mathcal{A}}
        =\mu_{0}.$
        \item If $\nu$ ia another measure on $\M$ such that 
        $\nu|_{\mathcal{A}}=\mu_{0}$, then $\nu(E)\le\mu(E)$, 
        and if $\mu(E)<\infty$ then $\nu(E)=\mu(E)$.
        \item If $\mu_{0}$ is $\sigma$-finite, then $\mu$ is unique.
    \end{enumerate}
\end{thm}
\begin{proof}
    $(1)$ By Proposition \ref{Prop:PremeasureAndOuterMeas}, 
    every sets in $\mathcal{A}$ are $\mu^{*}$-measurable. Then by 
    Theorem \ref{Thm:CaratheodoryThm}, the $\mu^{*}$-measurable 
    sets form a $\sigma$-algebra. 
    So $\forall M\in\M(\mathcal{A})$, $M$ is $\mu^{*}$-measurable. 
    By Theorem \ref{Thm:CaratheodoryThm}, 
    there exists the extended measure $\mu$ on $\M$. 

    $(2)$ If $E\in\M$, choose $E\subset\cp{j}{\infty}A_{j}$, 
    $A_{j}\in\mathcal{A}$, then:
    \begin{displaymath}
        \nu(E)\le\sm{j}{\infty}\nu(A_{j})
        =\sm{j}{\infty}\mu_{0}(A_j).
    \end{displaymath}
    Get the infimum for $E\subset\cp{j}{\infty}A_j$, it means 
    $\nu(E)\le\mu(E)$. 

    If $\mu(E)<\infty$, choose disjoint sets $\{A_{j}\}$ such that 
    \begin{itemize}
        \item $A:=\cp{j}{\infty}A_j\supset E$,
        \item $\mu(A)<\mu(E)+\epsilon$,
    \end{itemize}
    i.e. $\mu(A\setminus E)<\epsilon$. In additional:
    \begin{displaymath}
        \nu(A)=\lim_{n\rightarrow\infty}\nu(\cp{j}{n}A_j)
        =\lim_{n\rightarrow\infty}\mu(\cp{j}{n}A_j)=\mu(A),
    \end{displaymath}
    so 
    \begin{displaymath}
        \begin{array}{rl}
        &\mu(E)\le\mu(A)=\nu(A)=\nu(E)+\nu(A\setminus E)\\
        \le&\nu(E)+\mu(A\setminus E)<\nu(E)+\epsilon.
        \end{array}
    \end{displaymath}
    i.e. $\nu(E)=\mu(E)$.

    $(3)$ If $X=\cp{j}{\infty}A_{j}$ with $\mu_{0}(A_j)<\infty$, 
    and $\{A_j\}$ disjoint, 
    then:
    \begin{displaymath}
        \forall E\in\M,\quad \mu(E)=\sm{j}{\infty}\mu(E\cap A_j)
        =\sm{j}{\infty}\nu(E\cap A_j)=\nu(E).
    \end{displaymath}
\end{proof}
\section{Borel Measure}
\begin{rem}
    In this section, we consider a special case for 
    $\M:=\mathcal{B}_{\mathbb{R}}$, and induce the 
    \textit{Borel measure} on $\M$. Finally, we introduce the 
    \textit{Lebesgue measure} as a special case.
\end{rem}
\begin{defn}
    \label{Defn:BorelMeas}
    A measure $\mu$ is called a \textit{Borel measure} if 
    $\mu:\mathcal{B}_{\mathbb{R}}\rightarrow[0,\infty]$.
\end{defn}
\begin{defn}
    \label{Defn:DistributionFunc}
    If $\mu$ is a finite Borel measure, $F(x):=\mu((-\infty,x])$ 
    is the \textit{distribution function} of $\mu$.
\end{defn}
\begin{prop}
    \label{Prop:DistributionFuncProperties}
    If $F$ is a distribution function of a Borel measure $\mu$, then:
    \begin{enumerate}
        \item $F(x)$ is increasing on $\mathbb{R}$. 
        \item $F(x)$ is right continuous on $\mathbb{R}$.
        \item For $b>a$, $\mu(a,b]=F(b)-F(a)$.
    \end{enumerate}
\end{prop}
\begin{exc}
    Prove Proposition \ref{Prop:DistributionFuncProperties}.
\end{exc}
\begin{exc}
    Assume $X$ is a random variable on $\mathbb{R}$, 
    explain the relationship between the probability and 
    the distribution function of $X$.
\end{exc}
\begin{rem}
    Now, we induce a distribution function $F$ by a Borel measure 
    $\mu$. Can we induce a Borel measure $\mu$ by a 
    increasing and right continuous function $F$?
\end{rem}
\begin{ntn}
    $(a,b]$, $(a,\infty)$, $\emptyset$ are called 
    \textit{half-intervals} in $\mathbb{R}$.
\end{ntn}
\begin{prop}
    \label{Prop:FiniteHalfIntAlg}
    Mark $\mathcal{A}$:=\{finite disjoint union of \newline
    half-intervals in $\mathbb{R}$\}, then 
    $\mathcal{A}$ is an algebra, 
    and $\M(\mathcal{A})=\mathcal{B}_{\mathbb{R}}$.
\end{prop}
\begin{exc}
    Prove Proposition \ref{Prop:FiniteHalfIntAlg}.
\end{exc}
\begin{lem}
    \label{Lem:PremeasureOnAlg}
    Given a right-continuous and increasing function 
    $F:\mathbb{R}\rightarrow\mathbb{R}$, if 
    $\{(a_j,b_j]\}_{j=1}^{n}$ are disjoint half-intervals, 
    let $\mu_0(\cp{j}{n}(a_j,b_j]):=\sm{i}{n}(F(b_i)-F(a_i))$ 
    and $\mu_{0}(\emptyset)=0$, then $\mu_0$ is a premeasure on 
    $\mathcal{A}$.
\end{lem}
\begin{proof}
    We divide this proof into three steps.

    First, we show that $\mu_{0}$ is well-defined. 
    If $\{I_i\}_{i=1}^{n}$, $\{J_{j}\}_{j=1}^{m}$ are disjoint 
    half-intervals satisfies $\cp{i}{n}I_i=\cp{j}{m}J_j$, 
    then:
    \begin{displaymath}
        \sm{i}{n}\mu_{0}(I_{i})
        =\sum_{i,j}\mu_{0}(I_i\cap J_j)
        =\sm{j}{m}\mu_{0}(J_j),
    \end{displaymath}
    so $\mu_0(E)$ isn't related to the partition, i.e. 
    $\mu_{0}$ is well-defined on $\mathcal{A}$. 

    Second, we show $\mu_{0}$ is finitely additive. 
    Choose $I_{i}:=(a_i,b_i]$, and $\forall 1\le i<j\le n$, 
    $I_i\cap I_j=\emptyset$, then:
    \begin{displaymath}
        \mu_{0}(\cp{i}{n}I_{i})
        =\sm{i}{n}(F(b_i)-F(a_i))
        =\sm{i}{n}\mu_{0}(I_i).
    \end{displaymath}
    So $\mu_{0}$ is finitely additive on $\mathcal{A}$.

    Finally, we show $\mu_{0}$ is countably additive. 
    If $\{I_{j}\}_{j=1}^{\infty}$ are disjoint half-intervals, 
    it suffices to show 
    $\mu_{0}(\cp{j}{\infty}I_j)=\sm{j}{\infty}\mu_{0}(I_j)$. 
    Since $\cp{j}{\infty}I_{j}\in\mathcal{A}$, 
    $I:=\cp{j}{\infty}I_{j}$ is a finite union of 
    half-intervals. 
    So, WLOG, assume $\cp{j}{\infty}I_j=(a,b]$. 
    We consider the case for $-\infty<a<b<\infty$. If 
    $a=-\infty$ or $b=\infty$, the proof is remained as exercise. 
    Mark $I:=\cp{j}{\infty}I_j=(a,b]$, then:
    \begin{displaymath}
        \begin{array}{rl}
        \mu_{0}(I)&=\mu_{0}(\cp{j}{n}I_j)+\mu_{0}(I\setminus\cp{j}{n}I_j)\\
        &\ge\mu_{0}(\cp{j}{n}I_j)=\sm{j}{n}\mu_{0}(I_j).
        \end{array}
    \end{displaymath}
    Set $n\rightarrow\infty$, then 
    $\mu_{0}(I)\ge\sm{j}{\infty}\mu_{0}(I_j)$. 
    Since $F$ is right continuous, $\forall\epsilon>0$, 
    $\exists\delta>0$, $F(a+\delta)-F(a)<\epsilon$, 
    and each $I_j:=(a_j,b_j]$, $\exists \delta_j>0$ such that 
    $F(b_j+\delta_j)-F(b_j)<2^{-j}\epsilon$. 
    By the definition, $\cp{j}{\infty}(a_j,b_j+\delta_j)
    \supset[a+\delta_1,b]$. Since $[a+\delta_1,b]$ is compact, 
    there exists a finite subcover 
    $\{(a_{j_{k}},b_{j_k}+\delta_{j_k})\}_{k=1}^{N}$ 
    that covers $[a+\delta_1,b]$. 

    Now, sort this finite subcover such that $b_{j_k}+\delta_{j_k}
    \in(a_{j_{k+1}},b_{j_{k+1}}+\delta_{j_{k+1}})$, then:
    \begin{displaymath}
        \begin{array}{rl}
            \mu_{0}(I)&<F(b)-F(a+\delta)+\epsilon\\
            &\le F(b_{j_N}+\delta_{j_N})-F(a_{j_1})+\epsilon\\
            &\le\sm{j}{N}[F(b_{j_{k}}+\delta_{j_k})-F(a_{j_k})]+\epsilon\\
            &\le 2\epsilon+\sm{k}{N}(F(b_{j_k})-F(a_{j_k}))\\
            &\le\sm{j}{\infty}\mu(I_j)+2\epsilon.
        \end{array}
    \end{displaymath}
    So $\mu_{0}(I)\le\sm{j}{\infty}\mu_{0}(I_j)$. 
    The countable additivity is true 
    for $-\infty<a<b<\infty$. 
\end{proof}
\begin{exc}
    Complete the proof of Lemma \ref{Lem:PremeasureOnAlg}.
\end{exc}
\begin{rem}
    By Lemma \ref{Lem:PremeasureOnAlg}, we generalize 
    the length of intervals to the premeasure on $\mathcal{A}$.
\end{rem}
\begin{thm}
    \label{Thm:BorelMeasAndGenFunc}

    \begin{enumerate}
        \item If $F:\mathbb{R}\rightarrow\mathbb{R}$ is increasing and 
        right continuous, then there exists unique Borel measure 
        $\mu_{F}$ on $\mathbb{R}$ such that 
        $\mu_{F}((a,b])=F(b)-F(a)$.
        \item If $\mu$ is a Borel measure on $\mathbb{R}$ that is 
        finite on all bounded Borel sets, define 
        \begin{displaymath}
            F(x)=\left\{
                \begin{array}{rll}
                    &\mu((0,x]),&x>0\\
                    &0,&x=0\\
                    &-\mu([x,0)),&x<0\\
                \end{array}
            \right.,
        \end{displaymath}
        then $F$ is increasing and right continuous.
    \end{enumerate}
\end{thm}
\begin{proof}
    $(1)$ By Lemma \ref{Lem:PremeasureOnAlg}, $F$ gives a 
    $\sigma$-finite premeasure on $\mathcal{A}$. 
    Then by Theorem \ref{Thm:PremeasureExtension}, there exists a 
    unique measure $\mu_{F}$ on $\mathbb{R}$ such that 
    $\mu_{F}(a,b]=F(b)-F(a)$.

    $(2)$ A direct corollary of Proposition 
    \ref{Prop:DistributionFuncProperties}.
\end{proof}
\begin{rem}
    Theorem \ref{Thm:BorelMeasAndGenFunc} 
    omit the assumption that $\mu$ is finite. 
    The Borel measure $\mu_{F}$ is called 
    \textit{Lebesgue-Stieltjes measure.} 
\end{rem}
\begin{rem}
    \begin{displaymath}
        \begin{array}{rl}
            \mu^{*}(E)&:=\inf\left\{\sm{i}{\infty}\mu((a_i,b_i]):
            E\subset\cp{i}{\infty}(a_i,b_i]\right\}\\
            &=\inf\left\{\sm{i}{\infty}(F(b_i)-F(a_i))\right\}.\\
        \end{array}
    \end{displaymath}
    If $F(x)\equiv x$, $\mu^{*}$ is the \textit{Lebesgue 
    outer measure }on $\mathbb{R}$. 
\end{rem}
\begin{ntn}
    $\M_{\mu}$ is the collection of all the $\mu^{*}$-measurable 
    sets.
\end{ntn}
\begin{lem}
    \label{Lem:OuterMeasureByOpen}
    \begin{displaymath}
        \nu^{*}(E):=\inf\left\{\sm{i}{\infty}\mu((a_i,b_i)):
        E\subset\cp{i}{\infty}(a_i,b_i)\right\},
    \end{displaymath}
    then $\mu^{*}(E)=\nu^{*}(E)$.
\end{lem}
\begin{exc}
    Prove Lemma \ref{Lem:OuterMeasureByOpen}.
\end{exc}
\begin{thm}
    \label{Thm:ApproxMeasurableSet}
    If $E\in\M_{\mu}$, then:
    \begin{displaymath}
        \begin{array}{rl}
        \mu(E)&=\inf\{\mu(U):U\text{ is open, }U\supset E\}\\
        &=\sup\{\mu(K):K\text{ is compact, }K\subset E\}.
        \end{array}
    \end{displaymath}
\end{thm}
\begin{proof}
    The first equality follows from Lemma \ref{Lem:OuterMeasureByOpen}. 
    Now we need to show 
    \begin{displaymath}
        \mu(E)=\sup\{\mu(K):K\text{ is compact, }K\subset E\}.
    \end{displaymath}
    First, assume $E$ is bounded. If $E$ is closed, just choose $K=E$. 
    Otherwise, $\forall\epsilon>0$, $\exists$ open set 
    $U\supset\bar{E}\setminus E$ s.t. 
    \begin{displaymath}
        \mu(U)\le\mu(\bar{E}\setminus E)+\epsilon.
    \end{displaymath}
    Mark $K:=\bar{E}\setminus U$, then $K$ is compact and $K\subset E$, 
    and:
    \begin{displaymath}
        \begin{array}{rl}
            \mu(K)&=\mu(E)-\mu(E\cap U)\\
            &=\mu(E)-[\mu(U)-\mu(U\setminus E)]\\
            &\ge\mu(E)-\mu(U)+\mu(\bar{E}\setminus E)\\
            &\ge\mu(E)-\epsilon.
        \end{array}
    \end{displaymath}
    If $E$ is unbounded, let $E_j:=E\cap (j,j+1]$, then 
    $E_{j}$ is bounded. $\forall\epsilon>0$, $\exists$ a compact 
    set $K_j\subset E_j$ such that 
    $\mu(K_j)\ge\mu(E_j)-4^{-|j|}\epsilon$. 
    Let $H_{n}:=\cup_{j=-n}^{n}K_j$, then $H_n$ is compact and 
    $H_n\subset E$ with $\mu(H_n)\ge\mu(\cup_{j=-n}^{n}E_j)-\epsilon$. 
    Get supremum related to $n$, 
    it means:
    \begin{displaymath}
        \sup\{\mu(K):K\text{ is compact and }K\subset E\}
        \ge\mu(E)-\epsilon.
    \end{displaymath}
    What's more, $K\subset E$ means $\mu(K)\le\mu(E)$, so 
    \begin{displaymath}
        \mu(E)=\sup\{\mu(K):K\text{ is compact, }K\subset E\}.
    \end{displaymath}
\end{proof}
\begin{rem}
    Theorem \ref{Thm:ApproxMeasurableSet} shows we can 
    approximate a Borel measurable set by 
    open set or closed set.
\end{rem}
\begin{defn}
    A countable intersection of open sets is called a 
    \textit{$G_{\delta}$-set}.
\end{defn}
\begin{defn}
    A countable union of closed sets is called an 
    \textit{$F_{\sigma}$-set}.
\end{defn}
\begin{thm}
    \label{Thm:ApproxMeasSetByGdFs}
    If $E\subset\mathbb{R}$, TFAE:
    \begin{enumerate}
        \item $E\in\mathcal{M}_{\mu}$.
        \item $E=V\setminus N_1$, $V$ is a $G_{\delta}$ set and 
        $\mu(N_1)=0$.
        \item $E=H\cup N_2$, $H$ is a $F_{\sigma}$ set and 
        $\mu(N_2)=0$.
    \end{enumerate}
\end{thm}
\begin{proof}
    $(2)\Rightarrow(1)$ or $(3)\Rightarrow(1)$: Since $\mu$ 
    is complete on $\M_{\mu}$, $V,H,N_1,N_2$ all measurable, 
    so $E$ is measurable.

    $(1)\Rightarrow(3)$ or $(1)\Rightarrow(2)$: If $E\in\M_{\mu}$, 
    first assume $\mu(E)<\infty$, 
    by Theorem \ref{Thm:ApproxMeasurableSet}
    it means $\forall j\in\mathbb{N},\epsilon>0,$ 
    there exists open sets $U_{j}$, closed sets $K_{j}$ such that 
    $K_j\subset E\subset U_j$ and 
    \begin{displaymath}
        \mu(U_j)-2^{-j}\epsilon\le\mu(E)\le\mu(K_{j})+2^{-j}\epsilon.
    \end{displaymath}
    Choose $V:=\cap_{j=1}^{\infty}U_{j}$, $H:=\cup_{j=1}^{\infty}K_j$, 
    then $V$ is a $G_{\delta}$ set and 
    $H$ is a $F_{\sigma}$ set, and 
    $\mu(V)=\mu(H)=\mu(E)<\infty$. So 
    $\mu(V\setminus E)=\mu(E\setminus H)=0$.
\end{proof}
\begin{exc}
    When $\mu(E)=\infty$, prove Theorem \ref{Thm:ApproxMeasSetByGdFs}.
\end{exc}
\begin{exc}
    If $E\in\M_{\mu}$ and $\mu(E)<\infty$, then for every 
    $\epsilon>0$ there is a set $A$ that is a finite union of open 
    intervals such that $\mu(E\triangle A)<\epsilon$.
\end{exc}
\begin{defn}
    If $F(x)=x$, $\mu_{F}$ is called the \textit{Lebesgue measure }on 
    $\mathbb{R}$. The domain of $m$ is the class of Lebesgue 
    measurable sets $\mathcal{L}$.
\end{defn}
\begin{defn}
    The \textit{Cantor set }$\mathcal{C}\subset[0,1]$ is the 
    set ot all $x\in[0,1]$ with 
    \begin{displaymath}
        x=\sm{j}{\infty}a_j3^{-j},\quad a_j\in\{0,2\}.
    \end{displaymath}
\end{defn}
\begin{prop}
    \label{Prop:PropertiesOfCantorSet}
    Cantor set has some special properties:
    \begin{enumerate}
        \item $\mathcal{C}$ is compact, nowhere dense, 
        totally disconnected and with no isolated points.
        \item $m(\mathcal{C})=0$.
        \item $\text{card}(\mathcal{C})=\aleph$.
    \end{enumerate}
\end{prop}
\begin{exc}
    Prove Proposition \ref{Prop:PropertiesOfCantorSet}.
\end{exc}
\begin{rem}
    Cantor set is a noticeable example 
    for the subset of $[0,1]$.
\end{rem}