\begin{defn}
    The \textit{Dirichlet function }on $[0,1]$ is 
    \begin{equation}
        \label{Equ:Diri_Func}
        D(x)=\left\{
            \begin{array}{rl}
                1&x\in \mathbb{Q},\\
                0&x\notin\mathbb{Q}.
            \end{array}
        \right.
    \end{equation}
\end{defn}
\begin{exc}
    Show \eqref{Equ:Diri_Func} isn't Riemann integrable on $[0,1]$.
\end{exc}
\begin{rem}
    $D(x)=0$ a.e. on $[0,1]$, so we expact $\int_{0}^{1}D(x)\dif x=0$, 
    but $D(x)$ isn't Riemann integrable. 
    In this chapter, we introduce \textit{Lebesgue integral} to 
    handle this problem.
\end{rem}
\section{Measurable Functions}
\begin{defn}
    \label{Def:Preimage}
    Given a function $f:X\rightarrow Y$, $E\subset Y$, 
    \begin{displaymath}
        f^{-1}(E):=\{x\in X:f(x)\in E\}
    \end{displaymath}
    is the \textit{preimage} of $E$ related to $f$.
\end{defn}
\begin{prop}
    \label{Prop:PreimageProp}
    \begin{enumerate}
        \item $f^{-1}(\cup_{\lambda}E_{\lambda})
        =\cup_{\lambda}f^{-1}(E_{\lambda})$. 
        \item $f^{-1}(\cap_{\lambda}E_{\lambda})
        =\cap_{\lambda}f^{-1}(E_{\lambda})$.
        \item $f^{-1}(E^c)=(f^{-1}(E))^{c}$.
        \item If $\mathcal{N}$ is a $\sigma$-algebra on $Y$, 
        then $\{f^{-1}(E):E\in\mathcal{N}\}$ is a 
        $\sigma$-algebra on $X$.
    \end{enumerate}
\end{prop}
\begin{exc}
    Prove Proposition \ref{Prop:PreimageProp}.
\end{exc}
\begin{defn}
    \label{Def:MeasurableFunc}
    If $(X,\M)$, $(Y,\mathcal{N})$ are measurable spaces, 
    $f:X\rightarrow Y$ is called \textit{$(\M,\mathcal{N})$-measurable} 
    if $\forall E\in\mathcal{N}$, $f^{-1}(E)\in\M$.
\end{defn}
\begin{prop}
    \label{Prop:DescribeMeasFunc}
    If $\mathcal{N}=\M(\mathcal{E})$, then $f:X\rightarrow Y$ is 
    measurable iff $\forall E\in\mathcal{E}$, $f^{-1}(E)\in\M$.
\end{prop}
\begin{proof}
    $"\Rightarrow"$: Follows directly by Definition 
    \ref{Def:MeasurableFunc}.
    
    $"\Leftarrow"$: By Proposition \ref{Prop:PreimageProp}, 
    $\mathcal{A}:=\{E\subset Y:f^{-1}(E)\in\M\}$ is a $\sigma$-algebra, 
    and by the condition, this $\sigma$-algebra contains $\mathcal{E}$, 
    so $\mathcal{A}\supset\mathcal{M}(\mathcal{E})=\mathcal{N}$, 
    i.e. $f$ is a measurable function.  
\end{proof}
\begin{coro}
    \label{Cor:ContinuousMeas}
    If $(X,\tau_1)$, $(Y,\tau_2)$ are topological spaces and 
    $f:X\rightarrow Y$ is continuous, then 
    $f$ is $(\mathcal{B}_{X},\mathcal{B}_{Y})$-measurable.
\end{coro}
\begin{exc}
    Prove Corollary \ref{Cor:ContinuousMeas}.
\end{exc}
\begin{defn}
    \label{Defn:Mmeasurable}
    Given $(X,\M)$ be a measurable space, $f:X\rightarrow\mathbb{R}$ 
    (or $\mathbb{C}$) 
    is called \textit{$\M-$measurable} if $f$ is 
    $(\M,\mathcal{B}_{\mathbb{R}})$ (or $(\M,\mathcal{B}_{\mathbb{C}})$) 
    measurable.
\end{defn}
\begin{defn}
    
\end{defn}
\section{Integration}
\section{Convergence}
\section{Tonelli-Fubini Theorem}