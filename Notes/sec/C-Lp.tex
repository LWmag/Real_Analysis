\section{Definition}
\begin{rem}
    In Definition \ref{Defn:L1space}, we mark the 
    $L^{1}$ norm. Now, for $p\neq 1$, how to give the 
    corresponding $L^{p}$ norm?
\end{rem}
\begin{defn}
    Given a measurable space $(X,\M,\mu)$, and a 
    measurable function $f$ on $X$, and $1\le p<\infty$. 
    Define 
    \begin{equation}
        \label{Equ:LpNorm}
        \norm{f}_{p}:=\left[\int|f|^{p}\right]^{\frac{1}{p}}, 
    \end{equation}
    then $\norm{\cdot}_{p}$ is called the 
    \textit{$L^{p}$-norm} of $f$.
\end{defn}
\begin{exc}
    Show that if $0<p<1$, then \eqref{Equ:LpNorm} fails to be  
    a norm.
\end{exc}
\begin{defn}[$L^{p}$-space]
    \label{Defn:LpSpace}
    Given a measurable space $(X,\M,\mu)$, the $L^{p}$ space 
    is:
    \begin{displaymath}
        L^{p}(X,\M,\mu):=\{f:X\rightarrow\mathbb{C}
        \text{ measurable, }\norm{f}_{p}<\infty\}.
    \end{displaymath}
    $f=g$ in $L^{p}(X,\M,\mu)$ iff 
    $\norm{f-g}_{p}=0$.
\end{defn}
\begin{ntn}
    If $(X,\mathcal{M},\mu)$ is understood, 
    then $L^{p}(X,\mathcal{M},\mu)$ can be abbreviated as $L^{p}$. 
\end{ntn}
\begin{rem}
    It suffices to show that $\norm{\cdot}_{p}$ 
    is a norm.
\end{rem}
\begin{lem}
    \label{Lem:LogInequality}
    If $a,b\ge 0$, $\lambda\in(0,1)$, then 
    \begin{displaymath}
        a^{\lambda}b^{1-\lambda}\le\lambda a+(1-\lambda)b.
    \end{displaymath}
\end{lem}
\begin{exc}
    Prove Lemma \ref{Lem:LogInequality}.
\end{exc}
\begin{thm}[Holder's Inequality]
    \label{Thm:Holder}
    If $1<p<\infty$, $\frac{1}{p}+\frac{1}{q}=1$, 
    $f,g$ be measurable functions on $X$, then 
    \begin{equation}
        \label{Equ:HolderInequality}
        \norm{fg}_{1}\le\norm{f}_{p}\norm{g}_{q}.
    \end{equation}
    ``$=$'' holds if and only if $\alpha|f|^{p}=\beta|g|^{q}$ 
    for some $\alpha,\beta\in\mathbb{R}\setminus\{0\}$.
\end{thm}
\begin{rem}
    By Holder's Inequality, if $f\in L^{p}$ and $g\in L^{q}$, 
    then $fg\in L^{1}$.
\end{rem}
\begin{proof}
    First, if $\norm{f}_{p}=0$ or $\norm{g}_{q}=0$, 
    then $\text{LHS}=\text{RHS}=0$. If $\norm{f}_{p}=\infty$; 
    $\norm{g}_{q}>0$ 
    or $\norm{g}_{q}=\infty$; $\norm{f}_{p}>0$, then 
    $\text{LHS}=\text{RHS}=\infty$.

    Then, we consider the case $0<\norm{f}_{p}<\infty$, 
    $0<\norm{g}_{q}<\infty$. First, we assume 
    $\norm{f}_{p}=\norm{g}_{q}=1$. On Lemma 
    \ref{Lem:LogInequality}, take 
    $a=|f|^{p}$, $b=|g|^{q}$, $\lambda=\frac{1}{p}$, then 
    $1-\lambda=\frac{1}{q}$. It means 
    \begin{displaymath}
        \begin{array}{rl}
            a^{\lambda}b^{1-\lambda}\le\lambda a+(1-\lambda)b&
            \Rightarrow |fg|\le\frac{1}{p}|f|^{p}+\frac{1}{q}|f|^{q}\\
            &\Rightarrow\int|fg|\le\frac{1}{p}+\frac{1}{q}=1\\
            &\Rightarrow\norm{fg}_{1}\le 1.
        \end{array}
    \end{displaymath}
    ``$=$'' holds means $|f|^{p}=|g|^{q}$ a.e.. 
    If the assumption doesn't hold, choose 
    $\tilde{f}:=\frac{f}{\norm{f}_{p}}$, 
    $\tilde{g}:=\frac{g}{\norm{g}_{q}}$, 
    then $\norm{\tilde{f}}_{p}=\norm{\tilde{g}}_{q}=1$, i.e. 
    \begin{displaymath}
        \left\|{\frac{fg}{\norm{f}_p\norm{g}_{q}}}_{1}\right\|
        \le 1
        \Rightarrow\norm{fg}_{1}\le\norm{f}_{p}\norm{g}_{q}.
    \end{displaymath}
    ``$=$'' holds mean $\alpha|f|^{p}=\beta|g|^{q}$.
\end{proof}
\begin{ntn}
    $p'$ is called the \textit{conjugate exponent} of $p$ if 
    $\frac{1}{p}+\frac{1}{p'}=1$. 
\end{ntn}
\begin{thm}[Minkowski's Inequality]
    If $1\le p<\infty$, $f,g\in L^{p}$, then 
    $\norm{f+g}_{p}\le\norm{f}_{p}+\norm{g}_{p}$.
\end{thm}
\begin{proof}
    If $p=1$, since $|f+g|\le|f|+|g|$, it shows that 
    \begin{displaymath}
        \int|f+g|\le\int|f|+\int|g|,
    \end{displaymath}
    so $\norm{f+g}_{1}\le\norm{f}_{1}+\norm{g}_{1}$.

    If $f+g=0$ a.e., then LHS$=0$, RHS$\ge 0$, so the inequality 
    holds. 

    For other cases, $|f+g|^{p}\le(|f|+|g|)|f+g|^{p-1}$, then 
    \begin{displaymath}
        \begin{array}{rl}
            \int|f+g|^{p}&\le \int|f||f+g|^{p-1}+|g||f+g|^{p-1}\\
            &\le\norm{f}_{p}\norm{|f+g|^{p-1}}_{p'}
            +\norm{g}_{p}\norm{|f+g|^{p-1}}_{p'}\\
            &=(\norm{f}_{p}+\norm{g}_{p})
            \left[\int|f+g|^{(p-1)\frac{p}{p-1}}\right]
            ^{\frac{p-1}{p}}\\
            &=(\norm{f}_{p}+\norm{g}_{p})
            \left(\int|f+g|^{p}\right)^{1-\frac{1}{p}}.
        \end{array}
    \end{displaymath}
    So $\norm{f+g}_{p}\le\norm{f}_{p}+\norm{g}_{p}$.
\end{proof}
\begin{defn}[Normed vector space]
    \label{Defn:NVS}
    \textit{Normed vector space} is a vector space 
    over the real or complex numbers on which norm is defined.
\end{defn}
\begin{exc}
    \label{Exc:LpisNVS}
    Show that for $1\le p<\infty$, $L^{p}$ is a normed vector space, 
    and $L^{p}$ is an inner space iff $p=2$.
\end{exc}
\begin{defn}
    For $p=\infty$, define 
    \begin{displaymath}
        \norm{f}_{\infty}:=\inf\{a\ge 0:|f(x)|\le a\text{ a.e.}\},
    \end{displaymath}
    $\norm{f}_{\infty}$ is called the \textit{essential supremum} of 
    $|f|$, marked as $\text{esssup}_{x\in X}|f(x)|$. 
    the $L^{\infty}(X,\M,\mu)$ space is 
    \begin{displaymath}
        L^{\infty}(X,\M,\mu):=\{f:X\rightarrow\mathbb{C},
        \text{ measurable, }\norm{f}_{\infty}<\infty\},    
    \end{displaymath}
    and we se $f=g$ in $L^{\infty}$ iff $f=g$ a.e..
\end{defn}
\begin{rem}
    $(1,\infty)$ is a pair of conjugate exponents.
\end{rem}
\begin{exc}
    Show that $\norm{\cdot}_{\infty}$ is a norm on $L^{\infty}$.
\end{exc}
\begin{exc}
    If $f,g$ are measurable functions, then 
    $\norm{fg}_{1}\le\norm{f}_{1}\norm{g}_{\infty}$. 
    When will the ``$=$'' satisfies?
\end{exc}
\section{Basic Properties}
\begin{lem}
    \label{Lem:NVScomplete}
    A normed vector space $V$ is complete iff every absolutely 
    convergent series in $V$ converges. 
\end{lem}
\begin{exc}
    Prove Lemma \ref{Lem:NVScomplete}.
\end{exc}
\begin{thm}
    For $1\le p<\infty$, $L^{p}$ is a complete normed vector space, 
    i.e., a Banach space.
\end{thm}
\begin{proof}
    By Exercise \ref{Exc:LpisNVS} and 
    Lemma \ref{Lem:NVScomplete}, it suffices to show 
    every absolutely convergent series in $L^{p}$ 
    is convergent in $L^{p}$. 

    Suppose $\{f_{k}\}\subset L^{p}$ and 
    $\sm{k}{\infty}\norm{f_{k}}_{p}=B<\infty$, let 
    $G_{n}=\sm{k}{n}|f_{k}|$, $G=\sm{k}{\infty}|f_{k}|$, then 
    by Minkowski's inequality, 
    $\forall n\in\mathbb{N}$, 
    $\norm{G_{n}}_{p}\le\sm{k}{n}\norm{f_{k}}_{p}<B$. 
    Since $\{G_{n}\}$ is increasing, by MCT, 
    \begin{displaymath}
        \int G^{p}=\lim_{n\rightarrow\infty}\int G_{n}^{p}
        \le B^{p}.
    \end{displaymath}
    So $G\in L^{p}$, which yields $G(x)<\infty$ a.e., and 
    $\{f_{n}\}$ is absolutely convergent a.e.. 
    It means $\sm{k}{\infty}f_{k}(x)$ converges a.e. to $F(x)$. 
    Since $|F|\le G$ and $G\in L^{p}$, $F\in L^{p}$ as well. 

    Now we should show that $\sm{k}{\infty}f_{k}$ 
    converges to $F$ in $L^{p}$. 
    By triangular inequality:
    \begin{displaymath}
        |F-\sm{k}{n}f_{k}|^{p}
        \le(|F|+\sm{k}{\infty}|f_{k}|)^{p}
        \le(2G)^{p}\in L^{1}.
    \end{displaymath}
    Then by DCT:
    \begin{displaymath}
        \lim_{n\rightarrow\infty}\norm{F-\sm{k}{n}f_{k}}_{p}^{p}
        =\int\lim_{n\rightarrow\infty}|F-\sm{k}{n}f_{k}|^{p}
        =0.
    \end{displaymath}
    So $\{f_{k}\}$ converges to $F$ in $L^{p}$.
\end{proof}
\begin{exc}
    Show that $L^{\infty}$ is complete.
\end{exc}
\begin{defn}[Separable]
    A topological space is called 
    \textit{separable} if it contains a 
    dense and countable subset.
\end{defn}
\begin{thm}
    If $1\le p<\infty$, then $L^{p}$ is separable.
\end{thm}
\begin{exc}
    Show that $L^{\infty}$ isn't separable.
\end{exc}
\begin{prop}
    \label{Prop:InterpolationIneq1}
    If $0<p<q<r\le\infty$, then $L^{q}\subset L^{p}+L^{r}$, i.e. 
    $\forall f\in L^{q}$, $\exists g\in L^{p}$, $h\in L^{r}$ 
    such that $f=g+h$.
\end{prop}
\begin{proof}
    Take $E:=\{x:|f(x)|>1\}$, set $g=f\chi_{E}$, $h=f(1-\chi_{E})$. 
    Since $p<q$, 
    \begin{displaymath}
        |g|^{p}=|f|^{p}\chi_{E}\le |f|^{q}\chi_{E}\in L^{1},
    \end{displaymath}
    so $g\in L^{p}$. 

    If $r<\infty$, $|h|^r=|f|^{r}(1-\chi_{E})\le|f|^{q}(1-\chi_{E})
    \in L^{1}$, so $h\in L^{r}$. 
    If $r=\infty$, since $\text{supp}h\subset X\setminus E$, 
    then $|h|\le 1$, i.e. $h\in L^{\infty}$.
\end{proof}
\begin{prop}
    \label{Prop:InterpolationIneq2}
    If $0<p<q<r\le\infty$, then $L^{p}\cap L^{r}\subset L^{q}$, 
    and 
    \begin{displaymath}
        \norm{f}_{q}\le\norm{f}_{p}^{\lambda}\norm{f}_{r}^{1-\lambda},
        \quad
        \lambda=\frac{\frac{1}{q}-\frac{1}{r}}{\frac{1}{p}-\frac{1}{r}}.
    \end{displaymath}
\end{prop}
\begin{proof}
    Since 
    \begin{displaymath}
        \frac{1}{\frac{p}{\lambda q}}
        +\frac{1}{\frac{r}{(1-\lambda)q}}=1,
    \end{displaymath}
    by Holder's inequality, 
    \begin{displaymath}
        \begin{array}{rl}
            \int |f|^{q}&=\int|f|^{\lambda q}|f|^{(1-\lambda)q}\\
            &\le\norm{|f|^{\lambda q}}_{\frac{p}{\lambda q}}
            \norm{|f|^{(1-\lambda)q}}_{\frac{r}{(1-\lambda)q}}\\
            &=\left(\int|f|^{p}\right)^{\frac{\lambda q}{p}}
            \left(\int|f|^{r}\right)^{\frac{(1-\lambda)q}{r}}.
        \end{array}
    \end{displaymath}
    So $\norm{f}_{q}\le\norm{f}_{p}^{\lambda}\norm{f}_{r}^{1-\lambda}$.
\end{proof}
\begin{prop}
    \label{Prop:LpContainLq}
    If $\mu(X)<\infty$, $0<p<q\le\infty$, then 
    $L^{p}(\mu)\supset L^{q}(\mu)$ and 
    $\norm{f}_{p}\le\norm{f}_{q}\mu(X)^{\frac{1}{p}-\frac{1}{q}}$.
\end{prop}
\begin{proof}
    If $q=\infty$, then 
    \begin{displaymath}
        \norm{f}_{p}^{p}=\int|f|^{p}\le\norm{f}_{\infty}^{p}\int 1
        =\norm{f}_{\infty}^{p}\mu(X).
    \end{displaymath}
    If $q<\infty$, the conjugate exponent of $\frac{q}{p}$ is 
    $\left(\frac{q}{p}\right)'=\frac{q}{q-p}$, then by Holder's 
    inequality, 
    \begin{displaymath}
        \begin{array}{rl}
            norm{f}_{p}^{p}&=\int|f|^{p}\\
            &\le\norm{|f|^{p}}_{\frac{q}{p}}\norm{1}_{\frac{q}{q-p}}\\
            &=\left(\int|f|^{q}\right)^{\frac{p}{q}}
            \left(\mu(X)\right)^{1-\frac{p}{q}}.\\
        \end{array}
    \end{displaymath}
    So 
    \begin{displaymath}
        \norm{f}_{p}\le\norm{f}_{q}
        \left(\mu(X)\right)^{\frac{1}{p}-\frac{1}{q}}.
    \end{displaymath}
\end{proof}
\begin{exc}
    Suppose $0<p_0<\infty$, find example of functions $f$ 
    on $(0,\infty)$ with Lebesgue measure, such that $f\in L^{p}$ 
    iff $p=p_{0}$.
\end{exc}
\section{$L^{2}$ Space}
\begin{defn}
    \label{Defn:InnerProdOnL2}
    If $f,g\in L^{2}$, then the \textit{inner product} of $f$ and $g$ 
    is 
    \begin{equation}
        \label{Equ:InnerProdL2}
        \left<f,g\right>:=\int_{X}f\bar{g}\dif\mu.
    \end{equation}
\end{defn}
\begin{exc}
    Show that \eqref{Equ:InnerProdL2} is an inner product.
\end{exc}
\begin{thm}
    $L^{2}$ space equipped with inner product \eqref{Equ:InnerProdL2} 
    is a Hilbert space. 
\end{thm}
\begin{defn}
    \label{Defn:OrthonormalSet}
    Given $\{e_{i}\}_{1}^{\infty}\subset L^{2}(X)$, if 
    \begin{displaymath}
        \innerProd{e_i,e_{j}}=\delta_{ij},
    \end{displaymath}
    then $\{e_{i}\}$ is called an \textit{orthonormal system}.
\end{defn}
\begin{exm}
    The \textit{Legendre polynomials} form an orthonormal 
    system of $L^{2}([-1,1])$.
\end{exm}
\begin{defn}[Fourier extension]
    If $\{e_{i}\}_{1}^{\infty}$ is an orthonormal 
    system of $L^{2}(X)$, then for $f\in L^{2}(X)$, the 
    \textit{Fourier extension} of $f$ is 
    \begin{displaymath}
        f\sim\sm{i}{\infty}\innerProd{f,e_i}e_i
    \end{displaymath}
\end{defn}
\begin{thm}[Bessel's Inequality]
    Mark $a_{i}:=\innerProd{f,e_i}$, then 
    \begin{displaymath}
        \sm{i}{\infty}|a_{i}|^{2}\le\norm{f}_{2}^{2}.
    \end{displaymath}
\end{thm}
\begin{proof}
    \textit{please complete this proof.}
\end{proof}
\begin{rem}
    So, the $L^{2}$ approximation of $f\in L^{2}(X)$ is 
    \begin{displaymath}
        f\sim\sm{i}{\infty}\innerProd{f,e_i}e_i.
    \end{displaymath}
\end{rem}
\begin{defn}
    If an orthonormal system $\{e_{i}\}$ satisfies 
    $\forall f\in L^{2}(X)$, 
    \begin{displaymath}
        f=\sm{i}{\infty}\innerProd{f,e_i}e_{i},
    \end{displaymath}    
    then $\{e_{i}\}$ is a \textit{complete orthonormal system}.
\end{defn}
\begin{thm}[Parseval's Equality]
    \label{Thm:Parseval}
    If $\{e_{i}\}$ is a complete orthonormal system, 
    then 
    \begin{displaymath}
        \norm{f}_{2}^{2}=\sm{i}{\infty}|\innerProd{f,e_{i}}|^{2}.
    \end{displaymath}
\end{thm}
\begin{proof}
    \textit{please complete this proof.}
\end{proof}
