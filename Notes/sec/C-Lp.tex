\section{Definition}
\begin{rem}
    In Definition \ref{Defn:L1space}, we mark the 
    $L^{1}$ norm. Now, for $p\neq 1$, how to give the 
    corresponding $L^{p}$ norm?
\end{rem}
\begin{defn}
    Given a measurable space $(X,\M,\mu)$, and a 
    measurable function $f$ on $X$, and $1\le p<\infty$. 
    Define 
    \begin{equation}
        \label{Equ:LpNorm}
        \norm{f}_{p}:=\left[\int|f|^{p}\right]^{\frac{1}{p}}, 
    \end{equation}
    then $\norm{\cdot}_{p}$ is called the 
    \textit{$L^{p}$-norm} of $f$.
\end{defn}
\begin{exc}
    Show that if $0<p<1$, then \eqref{Equ:LpNorm} fails to be  
    a norm.
\end{exc}
\begin{defn}[$L^{p}$-space]
    \label{Defn:LpSpace}
    Given a measurable space $(X,\M,\mu)$, the $L^{p}$ space 
    is:
    \begin{displaymath}
        L^{p}(X,\M,\mu):=\{f:X\rightarrow\mathbb{C}
        \text{ measurable, }\norm{f}_{p}<\infty\}.
    \end{displaymath}
\end{defn}
\begin{rem}
    It suffices to show that $\norm{\cdot}_{p}$ 
    is a norm.
\end{rem}
\begin{lem}
    \label{Lem:LogInequality}
    If $a,b\ge 0$, $\lambda\in(0,1)$, then 
    \begin{displaymath}
        a^{\lambda}b^{1-\lambda}\le\lambda a+(1-\lambda)b.
    \end{displaymath}
\end{lem}
\begin{exc}
    Prove Lemma \ref{Lem:LogInequality}.
\end{exc}
\begin{thm}[Holder's Inequality]
    \label{Thm:Holder}
    If $1<p<\infty$, $\frac{1}{p}+\frac{1}{q}=1$, 
    $f,g$ be measurable functions on $X$, then 
    \begin{equation}
        \label{Equ:HolderInequality}
        \norm{fg}_{1}\le\norm{f}_{p}\norm{g}_{q}.
    \end{equation}
    ``$=$'' holds if and only if $\alpha|f|^{p}=\beta|g|^{q}$ 
    for some $\alpha,\beta\in\mathbb{R}\setminus\{0\}$.
\end{thm}
\begin{rem}
    By Holder's Inequality, if $f\in L^{p}$ and $g\in L^{q}$, 
    then $fg\in L^{1}$.
\end{rem}
\begin{proof}
    First, if $\norm{f}_{p}=0$ or $\norm{g}_{q}=0$, 
    then $\text{LHS}=\text{RHS}=0$. If $\norm{f}_{p}=\infty$; 
    $\norm{g}_{q}>0$ 
    or $\norm{g}_{q}=\infty$; $\norm{f}_{p}>0$, then 
    $\text{LHS}=\text{RHS}=\infty$.

    Then, we consider the case $0<\norm{f}_{p}<\infty$, 
    $0<\norm{g}_{q}<\infty$. First, we assume 
    $\norm{f}_{p}=\norm{g}_{q}=1$. On Lemma 
    \ref{Lem:LogInequality}, take 
    $a=|f|^{p}$, $b=|g|^{q}$, $\lambda=\frac{1}{p}$, then 
    $1-\lambda=\frac{1}{q}$. It means 
    \begin{displaymath}
        \begin{array}{rl}
            a^{\lambda}b^{1-\lambda}\le\lambda a+(1-\lambda)b&
            \Rightarrow |fg|\le\frac{1}{p}|f|^{p}+\frac{1}{q}|f|^{q}\\
            &\Rightarrow\int|fg|\le\frac{1}{p}+\frac{1}{q}=1\\
            &\Rightarrow\norm{fg}_{1}\le 1.
        \end{array}
    \end{displaymath}
    ``$=$'' holds means $|f|^{p}=|g|^{q}$ a.e.. 
    If the assumption doesn't hold, choose 
    $\tilde{f}:=\frac{f}{\norm{f}_{p}}$, 
    $\tilde{g}:=\frac{g}{\norm{g}_{q}}$, 
    then $\norm{\tilde{f}}_{p}=\norm{\tilde{g}}_{q}=1$, i.e. 
    \begin{displaymath}
        \left\|{\frac{fg}{\norm{f}_p\norm{g}_{q}}}_{1}\right\|
        \le 1
        \Rightarrow\norm{fg}_{1}\le\norm{f}_{p}\norm{g}_{q}.
    \end{displaymath}
    ``$=$'' holds mean $\alpha|f|^{p}=\beta|g|^{q}$.
\end{proof}
\begin{ntn}
    $p'$ is called the \textit{conjugate exponent} of $p$ if 
    $\frac{1}{p}+\frac{1}{p'}=1$. 
\end{ntn}
\begin{thm}[Minkowski's Inequality]
    If $1\le p<\infty$, $f,g\in L^{p}$, then 
    $\norm{f+g}_{p}\le\norm{f}_{p}+\norm{g}_{p}$.
\end{thm}
\begin{proof}
    If $p=1$, since $|f+g|\le|f|+|g|$, it shows that 
    \begin{displaymath}
        \int|f+g|\le\int|f|+\int|g|,
    \end{displaymath}
    so $\norm{f+g}_{1}\le\norm{f}_{1}+\norm{g}_{1}$.

    If $f+g=0$ a.e., then LHS$=0$, RHS$\ge 0$, so the inequality 
    holds. 

    For other cases, $|f+g|^{p}\le(|f|+|g|)|f+g|^{p-1}$, then 
    \begin{displaymath}
        \begin{array}{rl}
            \int|f+g|^{p}&\le \int|f||f+g|^{p-1}+|g||f+g|^{p-1}\\
            &\le\norm{f}_{p}\norm{|f+g|^{p-1}}_{p'}
            +\norm{g}_{p}\norm{|f+g|^{p-1}}_{p'}\\
            &=(\norm{f}_{p}+\norm{g}_{p})
            \left[\int|f+g|^{(p-1)\frac{p}{p-1}}\right]
            ^{\frac{p-1}{p}}\\
            &=(\norm{f}_{p}+\norm{g}_{p})
            \left(\int|f+g|^{p}\right)^{1-\frac{1}{p}}.
        \end{array}
    \end{displaymath}
    So $\norm{f+g}_{p}\le\norm{f}_{p}+\norm{g}_{p}$.
\end{proof}
\section{Basic Properties}
\section{$L^{2}$ Space}
\section{Some Useful Ineqalities}
\section{(*)From $L^{p}$ to Sobolev Space}