\section{Definition}
\begin{rem}
    In Definition \ref{Defn:L1space}, we mark the 
    $L^{1}$ norm. Now, for $p\neq 1$, how to give the 
    corresponding $L^{p}$ norm?
\end{rem}
\begin{defn}
    Given a measurable space $(X,\M,\mu)$, and a 
    measurable function $f$ on $X$, and $1\le p<\infty$. 
    Define 
    \begin{equation}
        \label{Equ:LpNorm}
        \norm{f}_{p}:=\left[\int|f|^{p}\right]^{\frac{1}{p}}, 
    \end{equation}
    then $\norm{\cdot}_{p}$ is called the 
    \textit{$L^{p}$-norm} of $f$.
\end{defn}
\begin{exc}
    Show that if $0<p<1$, then \eqref{Equ:LpNorm} fails to be  
    a norm.
\end{exc}
\begin{proof}
    To prove that the $L_p$ norm is not a norm when 
    $0 < p < 1$, we can consider the $p$-th root inequality.
    First, we consider the absolute value function 
    
    $$
    |x| = \begin{cases} x, & \text{if } x \geq 0 \\ -x, 
    & \text{if } x < 0 \end{cases}
    $$

    Then, we utilize the Holder's inequality 
    $\int |fg| \leq (\int |f|^p)^{\frac{1}{p}} 
    (\int |g|^q)^{\frac{1}{q}}$, where $\frac{1}{p} + 
    \frac{1}{q} = 1$, and $p, q > 1$.
    Now, let's consider a simple sequence of functions 
    $f_n(x) = \frac{1}{n^{\frac{1}{p}}} \chi_{[0, n]}(x)$, 
    where $\chi_{[0, n]}(x)$ is the indicator function.
    We have $\|f_n\|_p = \left(\int |f_n|^p\right)^{\frac{1}{p}}
    = \left(\int_0^n \frac{1}{n} dx\right)^{\frac{1}{p}} 
    = \left(\frac{n}{n}\right)^{\frac{1}{p}} = 1$.
    Now, let's compute the limit of the $L_p$ norm of $f_n$:

    $$
    \lim_{n \to \infty} \|f_n\|_p = \lim_{n \to \infty} 1 = 1
    $$

    However, we have $\|f\|_p = 0$, where $f(x) 
    = \lim_{n \to \infty} f_n(x) = 0$. This violates 
    the definition of a norm, which states that 
    $\|f\|_p = 0 \Rightarrow f = 0$.
    Therefore, when $0 < p < 1$, the $L_p$ norm does not 
    satisfy the definition of a norm.
\end{proof}
\begin{defn}[$L^{p}$-space]
    \label{Defn:LpSpace}
    Given a measurable space $(X,\M,\mu)$, the $L^{p}$ space 
    is:
    \begin{displaymath}
        L^{p}(X,\M,\mu):=\{f:X\rightarrow\mathbb{C}
        \text{ measurable, }\norm{f}_{p}<\infty\}.
    \end{displaymath}
    $f=g$ in $L^{p}(X,\M,\mu)$ iff 
    $\norm{f-g}_{p}=0$.
\end{defn}
\begin{ntn}
    If $(X,\mathcal{M},\mu)$ is understood, 
    then $L^{p}(X,\mathcal{M},\mu)$ can be abbreviated as $L^{p}$. 
\end{ntn}
\begin{rem}
    It suffices to show that $\norm{\cdot}_{p}$ 
    is a norm.
\end{rem}
\begin{lem}
    \label{Lem:LogInequality}
    If $a,b\ge 0$, $\lambda\in(0,1)$, then 
    \begin{displaymath}
        a^{\lambda}b^{1-\lambda}\le\lambda a+(1-\lambda)b.
    \end{displaymath}
\end{lem}
\begin{exc}
    Prove Lemma \ref{Lem:LogInequality}.
\end{exc}
\begin{proof}
    To prove the inequality $a^{\lambda}b^{1-\lambda} 
    \leq \lambda a + (1-\lambda)b$, we can utilize the 
    definition of a convex function.

    First, let's consider the function $f(x) = \log(x)$, 
    which is convex. According to Jensen's inequality, 
    for a convex function $f(x)$, we have:

    $$
    \lambda f(a) + (1-\lambda)f(b) \geq 
    f(\lambda a + (1-\lambda)b)
    $$
    Substituting $f(x) = \log(x)$, we obtain:

    $$
    \lambda \log(a) + (1-\lambda)\log(b) \geq 
    \log(\lambda a + (1-\lambda)b)
    $$
    Next, exponentiating both sides of the inequality, we get:

    $$
    e^{\lambda \log(a) + (1-\lambda)\log(b)} \geq 
    e^{\log(\lambda a + (1-\lambda)b)}
    $$
    Simplifying, we arrive at:

    $$
    a^{\lambda}b^{1-\lambda} \leq \lambda a + (1-\lambda)b
    $$
    Therefore, by the properties of convex functions, 
    we have proved the inequality $a^{\lambda}b^{1-\lambda} 
    \leq \lambda a + (1-\lambda)b$.
\end{proof}
\begin{thm}[Holder's Inequality]
    \label{Thm:Holder}
    If $1<p<\infty$, $\frac{1}{p}+\frac{1}{q}=1$, 
    $f,g$ be measurable functions on $X$, then 
    \begin{equation}
        \label{Equ:HolderInequality}
        \norm{fg}_{1}\le\norm{f}_{p}\norm{g}_{q}.
    \end{equation}
    ``$=$'' holds if and only if $\alpha|f|^{p}=\beta|g|^{q}$ 
    for some $\alpha,\beta\in\mathbb{R}\setminus\{0\}$.
\end{thm}
\begin{rem}
    By Holder's Inequality, if $f\in L^{p}$ and $g\in L^{q}$, 
    then $fg\in L^{1}$.
\end{rem}
\begin{proof}
    First, if $\norm{f}_{p}=0$ or $\norm{g}_{q}=0$, 
    then $\text{LHS}=\text{RHS}=0$. If $\norm{f}_{p}=\infty$; 
    $\norm{g}_{q}>0$ 
    or $\norm{g}_{q}=\infty$; $\norm{f}_{p}>0$, then 
    $\text{LHS}=\text{RHS}=\infty$.

    Then, we consider the case $0<\norm{f}_{p}<\infty$, 
    $0<\norm{g}_{q}<\infty$. First, we assume 
    $\norm{f}_{p}=\norm{g}_{q}=1$. On Lemma 
    \ref{Lem:LogInequality}, take 
    $a=|f|^{p}$, $b=|g|^{q}$, $\lambda=\frac{1}{p}$, then 
    $1-\lambda=\frac{1}{q}$. It means 
    \begin{displaymath}
        \begin{array}{rl}
            a^{\lambda}b^{1-\lambda}\le\lambda a+(1-\lambda)b&
            \Rightarrow |fg|\le\frac{1}{p}|f|^{p}+\frac{1}{q}|f|^{q}\\
            &\Rightarrow\int|fg|\le\frac{1}{p}+\frac{1}{q}=1\\
            &\Rightarrow\norm{fg}_{1}\le 1.
        \end{array}
    \end{displaymath}
    ``$=$'' holds means $|f|^{p}=|g|^{q}$ a.e.. 
    If the assumption doesn't hold, choose 
    $\tilde{f}:=\frac{f}{\norm{f}_{p}}$, 
    $\tilde{g}:=\frac{g}{\norm{g}_{q}}$, 
    then $\norm{\tilde{f}}_{p}=\norm{\tilde{g}}_{q}=1$, i.e. 
    \begin{displaymath}
        \left\|{\frac{fg}{\norm{f}_p\norm{g}_{q}}}_{1}\right\|
        \le 1
        \Rightarrow\norm{fg}_{1}\le\norm{f}_{p}\norm{g}_{q}.
    \end{displaymath}
    ``$=$'' holds mean $\alpha|f|^{p}=\beta|g|^{q}$.
\end{proof}
\begin{ntn}
    $p'$ is called the \textit{conjugate exponent} of $p$ if 
    $\frac{1}{p}+\frac{1}{p'}=1$. 
\end{ntn}
\begin{thm}[Minkowski's Inequality]
    If $1\le p<\infty$, $f,g\in L^{p}$, then 
    $\norm{f+g}_{p}\le\norm{f}_{p}+\norm{g}_{p}$.
\end{thm}
\begin{proof}
    If $p=1$, since $|f+g|\le|f|+|g|$, it shows that 
    \begin{displaymath}
        \int|f+g|\le\int|f|+\int|g|,
    \end{displaymath}
    so $\norm{f+g}_{1}\le\norm{f}_{1}+\norm{g}_{1}$.

    If $f+g=0$ a.e., then LHS$=0$, RHS$\ge 0$, so the inequality 
    holds. 

    For other cases, $|f+g|^{p}\le(|f|+|g|)|f+g|^{p-1}$, then 
    \begin{displaymath}
        \begin{array}{rl}
            \int|f+g|^{p}&\le \int|f||f+g|^{p-1}+|g||f+g|^{p-1}\\
            &\le\norm{f}_{p}\norm{|f+g|^{p-1}}_{p'}
            +\norm{g}_{p}\norm{|f+g|^{p-1}}_{p'}\\
            &=(\norm{f}_{p}+\norm{g}_{p})
            \left[\int|f+g|^{(p-1)\frac{p}{p-1}}\right]
            ^{\frac{p-1}{p}}\\
            &=(\norm{f}_{p}+\norm{g}_{p})
            \left(\int|f+g|^{p}\right)^{1-\frac{1}{p}}.
        \end{array}
    \end{displaymath}
    So $\norm{f+g}_{p}\le\norm{f}_{p}+\norm{g}_{p}$.
\end{proof}
\begin{defn}[Normed vector space]
    \label{Defn:NVS}
    \textit{Normed vector space} is a vector space 
    over the real or complex numbers on which norm is defined.
\end{defn}
\begin{exc}
    \label{Exc:LpisNVS}
    Show that for $1\le p<\infty$, $L^{p}$ is a normed vector space, 
    and $L^{p}$ is an inner space iff $p=2$.
\end{exc}
\begin{proof}
    To demonstrate that for $1\leq p<\infty$, $L^{p}$ 
    is a normed vector space, we need to verify the 
    following properties:

    1. Closure under addition: For any \(f,g \in L^{p}\),
    their sum \(f+g\) also belongs to \(L^{p}\).
    
    2. Scalar multiplication: For any \(f \in L^{p}\) 
    and any scalar \(\alpha\), the scalar multiple \(\alpha f\) is in \(L^{p}\).
    
    3. Triangle inequality: \(\|f+g\|_{p} \leq \|f\|_{p} 
    + \|g\|_{p}\).
    
    4. Non-negativity: \(\|f\|_{p} \geq 0\) 
    and \(\|f\|_{p} = 0\) if and only if \(f = 0\).
    
    5. Homogeneity: \(\|\alpha f\|_{p} = |\alpha| 
    \|f\|_{p}\).

    The $L^{p}$ norm is defined as:

    $$
    \|f\|_{p} = \left( \int |f(x)|^{p} dx \right)^{\frac{1}{p}}
    $$

    For the first four properties, we can utilize Minkowski's 
    inequality. For the last property, it follows from the 
    properties of the absolute value and the integral.

    Now, to show that $L^{p}$ is an inner space if and only 
    if $p=2$, we need to consider the inner product. 
    The inner product on $L^{p}$ is defined as:

    $$
    \langle f,g \rangle = \int f(x)g(x) dx
    $$

    For $L^{p}$ to be an inner space, the inner product 
    needs to satisfy the properties of an inner product, 
    such as linearity and positive definiteness.

    When $p=2$, the inner product defined above satisfies 
    these properties. However, for $p \neq 2$, the 
    Cauchy-Schwarz inequality does not hold for all 
    functions in $L^{p}$, thus violating the positive 
    definiteness condition.

    Therefore, $L^{p}$ is an inner space if and only if $p=2$.
\end{proof}
\begin{defn}
    For $p=\infty$, define 
    \begin{displaymath}
        \norm{f}_{\infty}:=\inf\{a\ge 0:|f(x)|\le a\text{ a.e.}\},
    \end{displaymath}
    $\norm{f}_{\infty}$ is called the \textit{essential supremum} of 
    $|f|$, marked as $\text{esssup}_{x\in X}|f(x)|$. 
    the $L^{\infty}(X,\M,\mu)$ space is 
    \begin{displaymath}
        L^{\infty}(X,\M,\mu):=\{f:X\rightarrow\mathbb{C},
        \text{ measurable, }\norm{f}_{\infty}<\infty\},    
    \end{displaymath}
    and we se $f=g$ in $L^{\infty}$ iff $f=g$ a.e..
\end{defn}
\begin{rem}
    $(1,\infty)$ is a pair of conjugate exponents.
\end{rem}
\begin{exc}
    Show that $\norm{\cdot}_{\infty}$ is a norm on $L^{\infty}$.
\end{exc}
\begin{proof}
    To show that $\|\cdot\|_{\infty}$ is a norm on 
    $L^{\infty}$, we need to verify the three properties 
    of a norm:

    1. Non-negativity: For any $f \in L^{\infty}$, 
    $\|f\|_{\infty} \geq 0$, and $\|f\|_{\infty} = 0$ if 
    and only if $f = 0$ almost everywhere.

    2. Homogeneity: For any $f \in L^{\infty}$ and any 
    scalar $\alpha$, $\|\alpha f\|_{\infty} = |\alpha| 
    \|\alpha\|_{\infty}$.

    3. Triangle inequality: For any $f, g \in L^{\infty}$, 
    $\|f+g\|_{\infty} \leq \|f\|_{\infty} + \|g\|_{\infty}$.

    Let's prove each of these properties:

    1. Non-negativity: 
    - $\|f\|_{\infty} = \inf\{M : |f(x)| \leq M 
    \text{ a.e.}\}$. It's clear that $\|f\|_{\infty} \geq 0$.
    - If $\|f\|_{\infty} = 0$, then $|f(x)| \leq 0$ a.e., 
    which implies $f(x) = 0$ a.e.

    2. Homogeneity: 
    - $\|\alpha f\|_{\infty} = \inf\{M : |\alpha f(x)| 
    \leq M \text{ a.e.}\} = |\alpha|\inf\{M : |f(x)| 
    \leq M \text{ a.e.}\} = |\alpha|\|f\|_{\infty}$.

    3. Triangle inequality:
    - $\|f+g\|_{\infty} = \inf\{M : |f(x) + g(x)| 
    \leq M \text{ a.e.}\} \leq \inf\{M : |f(x)| + |g(x)| 
    \leq M \text{ a.e.}\} \leq \inf\{M_1, M_2 : |f(x)| 
    \leq M_1 \text{ a.e.}, |g(x)| \leq M_2 \text{ a.e.}\} = 
    \inf\{M_1 : |f(x)| \leq M_1 \text{ a.e.}\} + 
    \inf\{M_2 : |g(x)| \leq M_2 \text{ a.e.}\} = 
    \|f\|_{\infty} + \|g\|_{\infty}$.

    Therefore, $\|\cdot\|_{\infty}$ satisfies all the 
    properties of a norm, and thus it is a norm on 
    $L^{\infty}$.
\end{proof}
\begin{exc}
    If $f,g$ are measurable functions, then 
    $\norm{fg}_{1}\le\norm{f}_{1}\norm{g}_{\infty}$. 
    When will the ``$=$'' satisfies?
\end{exc}
\begin{proof}
    If $f$ and $g$ are measurable functions, then 
    $\|fg\|_{1} \leq \|f\|_{1}\|g\|_{\infty}$. 
    The equality $\|fg\|_{1} = \|f\|_{1}\|g\|_{\infty}$ 
    holds if and only if one of the functions $f$ or $g$ 
    is almost everywhere equal to zero, or if 
    $|f(x)| = c|g(x)|$ for almost all $x$ and some constant 
    $c$.
\end{proof}
\section{Basic Properties}
\begin{lem}
    \label{Lem:NVScomplete}
    A normed vector space $V$ is complete iff every absolutely 
    convergent series in $V$ converges. 
\end{lem}
\begin{exc}
    Prove Lemma \ref{Lem:NVScomplete}.
\end{exc}
\begin{proof}
    For the necessity,assume $V$ is a complete normed vector space. 
    We want to show that every absolutely convergent series 
    in $V$ converges.

    Let $\sum_{n=1}^{\infty} x_n$ be an absolutely convergent 
    series in $V$. This means that $\sum_{n=1}^{\infty} 
    \|x_n\|$ is convergent.

    Consider the sequence of partial sums 
    $S_k = \sum_{n=1}^{k} x_n$. We need to show that this 
    sequence is Cauchy and thus converges.

    Take $\varepsilon > 0$. Since $\sum_{n=1}^{\infty} 
    \|x_n\|$ converges, there exists $N$ such that for 
    all $m > n > N$, we have $\left\|\sum_{k=n+1}^{m} 
    x_k\right\| < \varepsilon$.

    This shows that the sequence of partial sums is Cauchy. 
    Since $V$ is complete, the sequence of partial sums 
    converges to some limit $L$, i.e., $\lim_{k \to \infty} 
    S_k = L$. Therefore, the absolutely convergent series 
    $\sum_{n=1}^{\infty} x_n$ converges in $V$.


    For the sufficiency, conversely, assume that every 
    absolutely convergent series in $V$ converges. We want 
    to show that $V$ is complete.

    Let $\{x_n\}$ be a Cauchy sequence in $V$. Since $V$ is a 
    normed vector space, a Cauchy sequence is a sequence for 
    which, given any $\varepsilon > 0$, there exists $N$ such 
    that for all $m,n > N$, $\|x_m - x_n\| < \varepsilon$.

    Consider the series $\sum_{n=1}^{\infty} (x_{n+1} - x_n)$. 
    This series is absolutely convergent, as 
    $\sum_{n=1}^{\infty} \|x_{n+1} - x_n\|$ is convergent due 
    to the Cauchy property of $\{x_n\}$.

    By our assumption, the series $\sum_{n=1}^{\infty} (x_{n+1} - x_n)$ 
    converges. Let the sum of this series be denoted as $L$. 
    We claim that $x_n$ converges to $L$.

    Given $\varepsilon > 0$, choose $N$ such that for all 
    $m,n > N$, $\|x_m - x_n\| < \varepsilon$. Then, for 
    $m > N$, we have $\left\|\sum_{n=1}^{m-1} (x_{n+1} - x_n) 
    - L\right\| < \varepsilon$. This implies that 
    $\|x_m - L\| < \varepsilon$.

    This shows that the Cauchy sequence $\{x_n\}$ converges 
    to $L$. Therefore, $V$ is complete.
\end{proof}
\begin{thm}
    For $1\le p<\infty$, $L^{p}$ is a complete normed vector space, 
    i.e., a Banach space.
\end{thm}
\begin{proof}
    By Exercise \ref{Exc:LpisNVS} and 
    Lemma \ref{Lem:NVScomplete}, it suffices to show 
    every absolutely convergent series in $L^{p}$ 
    is convergent in $L^{p}$. 

    Suppose $\{f_{k}\}\subset L^{p}$ and 
    $\sm{k}{\infty}\norm{f_{k}}_{p}=B<\infty$, let 
    $G_{n}=\sm{k}{n}|f_{k}|$, $G=\sm{k}{\infty}|f_{k}|$, then 
    by Minkowski's inequality, 
    $\forall n\in\mathbb{N}$, 
    $\norm{G_{n}}_{p}\le\sm{k}{n}\norm{f_{k}}_{p}<B$. 
    Since $\{G_{n}\}$ is increasing, by MCT, 
    \begin{displaymath}
        \int G^{p}=\lim_{n\rightarrow\infty}\int G_{n}^{p}
        \le B^{p}.
    \end{displaymath}
    So $G\in L^{p}$, which yields $G(x)<\infty$ a.e., and 
    $\{f_{n}\}$ is absolutely convergent a.e.. 
    It means $\sm{k}{\infty}f_{k}(x)$ converges a.e. to $F(x)$. 
    Since $|F|\le G$ and $G\in L^{p}$, $F\in L^{p}$ as well. 

    Now we should show that $\sm{k}{\infty}f_{k}$ 
    converges to $F$ in $L^{p}$. 
    By triangular inequality:
    \begin{displaymath}
        |F-\sm{k}{n}f_{k}|^{p}
        \le(|F|+\sm{k}{\infty}|f_{k}|)^{p}
        \le(2G)^{p}\in L^{1}.
    \end{displaymath}
    Then by DCT:
    \begin{displaymath}
        \lim_{n\rightarrow\infty}\norm{F-\sm{k}{n}f_{k}}_{p}^{p}
        =\int\lim_{n\rightarrow\infty}|F-\sm{k}{n}f_{k}|^{p}
        =0.
    \end{displaymath}
    So $\{f_{k}\}$ converges to $F$ in $L^{p}$.
\end{proof}
\begin{exc}
    Show that $L^{\infty}$ is complete.
\end{exc}
\begin{proof}
    Let's consider a Cauchy sequence $\{f_n\}$ in $L^\infty$. 
    This means that for any $\varepsilon > 0$, there exists 
    $N$ such that for all $m,n > N$, we have 
    $\|f_m - f_n\|_\infty < \varepsilon$.

    Since $L^\infty$ consists of essentially bounded functions, 
    for each $x$ in the domain, $\{f_n(x)\}$ is a Cauchy 
    sequence of real numbers. Therefore, for each $x$, 
    the sequence $\{f_n(x)\}$ converges to a limit, 
    say $f(x)$.

    We need to show that $f$ is essentially bounded and 
    that $\|f_n - f\|_\infty \rightarrow 0$ as 
    $n \rightarrow \infty$. 

    First, we show that $f$ is essentially bounded. 
    For each $n$, $\|f_n - f\|_\infty = \text{ess sup} 
    |f_n - f|$. As $n$ tends to infinity, 
    $\|f_n - f\|_\infty$ tends to zero, which means that 
    $f$ is essentially bounded.

    Now, we need to show that $\|f_n - f\|_\infty 
    \rightarrow 0$ as $n \rightarrow \infty$. Given 
    $\varepsilon > 0$, there exists $N$ such that for all 
    $n > N$, $\|f_n - f\|_\infty < \varepsilon$, 
    which demonstrates the convergence of $\{f_n\}$ to 
    $f$ in $L^\infty$.

    Therefore, we have shown that every Cauchy sequence in 
    $L^\infty$ converges to a limit within $L^\infty$, 
    establishing the completeness of $L^\infty$.
\end{proof}
\begin{defn}[Separable]
    A topological space is called 
    \textit{separable} if it contains a 
    dense and countable subset.
\end{defn}
\begin{thm}
    If $1\le p<\infty$, then $L^{p}$ is separable.
\end{thm}
\begin{exc}
    Show that $L^{\infty}$ isn't separable.
\end{exc}
\begin{proof}
    To prove that $L^{\infty}$ is not separable, 
    we can use the diagonal argument.

    Suppose $L^{\infty}$ is separable, meaning it has a 
    countable dense subset. This implies the existence of a 
    countable set $\{f_n\}_{n=1}^{\infty}$ such that for any 
    function $f$ in $L^{\infty}$ and any $\varepsilon > 0$, 
    there exists an $f_n$ satisfying $\|f - f_n\|_{\infty} < 
    \varepsilon$.

    Now, we construct a new function $f$ such that it has a 
    distance of at least $\varepsilon$ from all functions in 
    $\{f_n\}_{n=1}^{\infty}$ at least at one point. 
    This can be achieved using a diagonal argument.

    We define $f$ as follows:
    $$
    f(x) = \begin{cases} 
        1 & \text{if } f_n(x) \neq 1 \text{ for any } n \\
        0 & \text{if } f_n(x) = 1 \text{ for some } n
    \end{cases}
    $$

    This means that for any $f_n$, $f$ has a distance of at 
    least $\varepsilon$ from $f_n$ at least at one point. 
    Therefore, $f$ is different from every function in the 
    countable subset $\{f_n\}_{n=1}^{\infty}$, contradicting 
    the fact that $\{f_n\}_{n=1}^{\infty}$ is a dense subset 
    of $L^{\infty}$.
\end{proof}
\begin{prop}
    \label{Prop:InterpolationIneq1}
    If $0<p<q<r\le\infty$, then $L^{q}\subset L^{p}+L^{r}$, i.e. 
    $\forall f\in L^{q}$, $\exists g\in L^{p}$, $h\in L^{r}$ 
    such that $f=g+h$.
\end{prop}
\begin{proof}
    Take $E:=\{x:|f(x)|>1\}$, set $g=f\chi_{E}$, $h=f(1-\chi_{E})$. 
    Since $p<q$, 
    \begin{displaymath}
        |g|^{p}=|f|^{p}\chi_{E}\le |f|^{q}\chi_{E}\in L^{1},
    \end{displaymath}
    so $g\in L^{p}$. 

    If $r<\infty$, $|h|^r=|f|^{r}(1-\chi_{E})\le|f|^{q}(1-\chi_{E})
    \in L^{1}$, so $h\in L^{r}$. 
    If $r=\infty$, since $\text{supp}h\subset X\setminus E$, 
    then $|h|\le 1$, i.e. $h\in L^{\infty}$.
\end{proof}
\begin{prop}
    \label{Prop:InterpolationIneq2}
    If $0<p<q<r\le\infty$, then $L^{p}\cap L^{r}\subset L^{q}$, 
    and 
    \begin{displaymath}
        \norm{f}_{q}\le\norm{f}_{p}^{\lambda}\norm{f}_{r}^{1-\lambda},
        \quad
        \lambda=\frac{\frac{1}{q}-\frac{1}{r}}{\frac{1}{p}-\frac{1}{r}}.
    \end{displaymath}
\end{prop}
\begin{proof}
    Since 
    \begin{displaymath}
        \frac{1}{\frac{p}{\lambda q}}
        +\frac{1}{\frac{r}{(1-\lambda)q}}=1,
    \end{displaymath}
    by Holder's inequality, 
    \begin{displaymath}
        \begin{array}{rl}
            \int |f|^{q}&=\int|f|^{\lambda q}|f|^{(1-\lambda)q}\\
            &\le\norm{|f|^{\lambda q}}_{\frac{p}{\lambda q}}
            \norm{|f|^{(1-\lambda)q}}_{\frac{r}{(1-\lambda)q}}\\
            &=\left(\int|f|^{p}\right)^{\frac{\lambda q}{p}}
            \left(\int|f|^{r}\right)^{\frac{(1-\lambda)q}{r}}.
        \end{array}
    \end{displaymath}
    So $\norm{f}_{q}\le\norm{f}_{p}^{\lambda}\norm{f}_{r}^{1-\lambda}$.
\end{proof}
\begin{prop}
    \label{Prop:LpContainLq}
    If $\mu(X)<\infty$, $0<p<q\le\infty$, then 
    $L^{p}(\mu)\supset L^{q}(\mu)$ and 
    $\norm{f}_{p}\le\norm{f}_{q}\mu(X)^{\frac{1}{p}-\frac{1}{q}}$.
\end{prop}
\begin{proof}
    If $q=\infty$, then 
    \begin{displaymath}
        \norm{f}_{p}^{p}=\int|f|^{p}\le\norm{f}_{\infty}^{p}\int 1
        =\norm{f}_{\infty}^{p}\mu(X).
    \end{displaymath}
    If $q<\infty$, the conjugate exponent of $\frac{q}{p}$ is 
    $\left(\frac{q}{p}\right)'=\frac{q}{q-p}$, then by Holder's 
    inequality, 
    \begin{displaymath}
        \begin{array}{rl}
            norm{f}_{p}^{p}&=\int|f|^{p}\\
            &\le\norm{|f|^{p}}_{\frac{q}{p}}\norm{1}_{\frac{q}{q-p}}\\
            &=\left(\int|f|^{q}\right)^{\frac{p}{q}}
            \left(\mu(X)\right)^{1-\frac{p}{q}}.\\
        \end{array}
    \end{displaymath}
    So 
    \begin{displaymath}
        \norm{f}_{p}\le\norm{f}_{q}
        \left(\mu(X)\right)^{\frac{1}{p}-\frac{1}{q}}.
    \end{displaymath}
\end{proof}
\begin{exc}
    Suppose $0<p_0<\infty$, find example of functions $f$ 
    on $(0,\infty)$ with Lebesgue measure, such that $f\in L^{p}$ 
    iff $p=p_{0}$.
\end{exc}
\begin{proof}
    By simple calculus, we can see:
    \begin{itemize}
        \item $\int_{e}^{\infty}\frac{\dif x}{\left(x(\log x)^{2}\right)^{\frac{p}{p_{1}}}}$ 
        converges $\Leftrightarrow$ $p\ge p_{0}$,
        \item $\int_{0}^{\frac{1}{e}}\frac{\dif x}{\left(x(\log x)^{2}\right)^{\frac{p}{p_{1}}}}$ 
        converges $\Leftrightarrow$ $p\le p_{0}$.
    \end{itemize}
    So the solutions:
    \begin{itemize}
        \item $f:=\frac{\chi_{x\ge e}+\chi_{x\le\frac{1}{e}}}
        {(x(\log x)^{2})^{\frac{1}{p_0}}}$.
    \end{itemize}
\end{proof}
\section{$L^{2}$ Space}
\begin{defn}
    \label{Defn:InnerProdOnL2}
    If $f,g\in L^{2}$, then the \textit{inner product} of $f$ and $g$ 
    is 
    \begin{equation}
        \label{Equ:InnerProdL2}
        \left<f,g\right>:=\int_{X}f\bar{g}\dif\mu.
    \end{equation}
\end{defn}
\begin{exc}
    Show that \eqref{Equ:InnerProdL2} is an inner product.
\end{exc}
\begin{proof}
    To show that $\left<f,g\right>:=\int_{X}f\bar{g}\,d\mu$ 
    is an inner product, we need to verify the four 
    properties of an inner product:

1. Conjugate Symmetry: $\left<f,g\right> = 
\overline{\left<g,f\right>}$ for all $f,g \in L^2(X)$.

2. Linearity in the First Argument: $\left<af + bg, 
h\right> = a\left<f,h\right> + b\left<g,h\right>$ for all 
$f,g,h \in L^2(X)$ and scalars $a,b$.

3. Conjugate Linearity in the Second Argument: 
$\left<f, ag + bh\right> = \overline{a}\left<f,g\right> + 
\overline{b}\left<f,h\right>$ for all $f,g,h \in L^2(X)$ 
and scalars $a,b$.

4. Positive Definiteness: $\left<f,f\right> \geq 0$, 
and $\left<f,f\right> = 0$ if and only if $f = 0$ almost 
everywhere.

Let's verify each of these properties:

1. Conjugate Symmetry:
   \begin{align*}
   \left<f,g\right> &= \int_{X}f\bar{g}\,d\mu \\
   \overline{\left<g,f\right>} &= 
   \overline{\int_{X}g\bar{f}\,d\mu} = 
   \int_{X}\bar{g}f\,d\mu
   \end{align*}
   Since $\int_{X}f\bar{g}\,d\mu = \int_{X}\bar{g}f\,d\mu$, 
   we have $\left<f,g\right> = \overline{\left<g,f\right>}$, 
   satisfying conjugate symmetry.

2. Linearity in the First Argument:
   \begin{align*}
   \left<af + bg, h\right> &= \int_{X}(af + bg)\bar{h}\,
   d\mu \\
   &= a\int_{X}f\bar{h}\,d\mu + b\int_{X}g\bar{h}\,d\mu \\
   &= a\left<f,h\right> + b\left<g,h\right>
   \end{align*}
   Therefore, the inner product is linear in the first 
   argument.

3. Conjugate Linearity in the Second Argument:
   \begin{align*}
   \left<f, ag + bh\right> &= \int_{X}f\overline{ag + bh}\,
   d\mu \\
   &= \int_{X}f(\bar{a}g + \bar{b}h)\,d\mu \\
   &= \bar{a}\int_{X}f\bar{g}\,d\mu + \bar{b}\int_{X}f\bar{h}\,
   d\mu \\
   &= \overline{a}\left<f,g\right> + \overline{b}\left<f,h\right>
   \end{align*}
   Hence, the inner product is conjugate linear in the 
   second argument.

4. Positive Definiteness:
   For any $f \in L^2(X)$, we have $\left<f,f\right> = 
   \int_{X}f\bar{f}\,d\mu = \int_{X}|f|^2\,d\mu \geq 0$. 
   Additionally, $\int_{X}|f|^2\,d\mu = 0$ if and only if 
   $f = 0$ almost everywhere.
\end{proof}
\begin{thm}
    $L^{2}$ space equipped with inner product \eqref{Equ:InnerProdL2} 
    is a Hilbert space. 
\end{thm}
\begin{defn}
    \label{Defn:OrthonormalSet}
    Given $\{e_{i}\}_{1}^{\infty}\subset L^{2}(X)$, if 
    \begin{displaymath}
        \innerProd{e_i,e_{j}}=\delta_{ij},
    \end{displaymath}
    then $\{e_{i}\}$ is called an \textit{orthonormal system}.
\end{defn}
\begin{exm}
    The \textit{Legendre polynomials} form an orthonormal 
    system of $L^{2}([-1,1])$.
\end{exm}
\begin{defn}[Fourier extension]
    If $\{e_{i}\}_{1}^{\infty}$ is an orthonormal 
    system of $L^{2}(X)$, then for $f\in L^{2}(X)$, the 
    \textit{Fourier extension} of $f$ is 
    \begin{displaymath}
        f\sim\sm{i}{\infty}\innerProd{f,e_i}e_i
    \end{displaymath}
\end{defn}
\begin{thm}[Bessel's Inequality]
    Mark $a_{i}:=\innerProd{f,e_i}$, then 
    \begin{displaymath}
        \sm{i}{\infty}|a_{i}|^{2}\le\norm{f}_{2}^{2}.
    \end{displaymath}
\end{thm}
\begin{proof}
    In the $L^2$ space, Bessel's inequality states: for any 
    complete orthonormal system $\{e_i\}$ in a Hilbert space, 
    and for any $f$ in that Hilbert space:

$$
\sum_{i=1}^{\infty} |a_{i}|^{2} \leq \|f\|_{2}^{2}
$$

where $a_{i}:=\left<f,e_i\right>$ represents the inner 
product of $f$ with $e_i$.

We will use the completeness of the orthonormal system to 
prove this inequality. By completeness, any $f$ can be 
expressed as:

$$
f = \sum_{i=1}^{\infty} \left<f,e_i\right>e_i
$$

Now consider the $L^2$ norm of $f$:

$$
\|f\|_{2}^{2} = \left<f, f\right> = \left<\sum_{i=1}^{\infty} 
\left<f,e_i\right>e_i, \sum_{j=1}^{\infty} \left<f,e_j\right>e_j\right>
$$

Expanding this using linearity and properties of inner products, we get:

$$
\|f\|_{2}^{2} = \sum_{i=1}^{\infty} \sum_{j=1}^{\infty} 
\left<f,e_i\right>\overline{\left<f,e_j\right>}\left<e_i,e_j\right>
$$

Since $\{e_i\}$ is orthonormal, when $i\neq j$, 
$\left<e_i,e_j\right>=0$. Therefore, the above expression 
simplifies to:

$$
\|f\|_{2}^{2} = \sum_{i=1}^{\infty} |\left<f,e_i\right>|^2
$$
\end{proof}
\begin{rem}
    So, the $L^{2}$ approximation of $f\in L^{2}(X)$ is 
    \begin{displaymath}
        f\sim\sm{i}{\infty}\innerProd{f,e_i}e_i.
    \end{displaymath}
\end{rem}
\begin{defn}
    If an orthonormal system $\{e_{i}\}$ satisfies 
    $\forall f\in L^{2}(X)$, 
    \begin{displaymath}
        f=\sm{i}{\infty}\innerProd{f,e_i}e_{i},
    \end{displaymath}    
    then $\{e_{i}\}$ is a \textit{complete orthonormal system}.
\end{defn}
\begin{thm}[Parseval's Equality]
    \label{Thm:Parseval}
    If $\{e_{i}\}$ is a complete orthonormal system, 
    then 
    \begin{displaymath}
        \norm{f}_{2}^{2}=\sm{i}{\infty}|\innerProd{f,e_{i}}|^{2}.
    \end{displaymath}
\end{thm}
\begin{proof}
    Parseval's equality states that for any $f$ in a Hilbert 
    space with a complete orthonormal system $\{e_i\}$:

$$
\|f\|_{2}^{2} = \sum_{i=1}^{\infty} |a_{i}|^{2}
$$

where $a_{i}:=\left<f,e_i\right>$ represents the inner 
product of $f$ with $e_i$.

To prove Parseval's equality, we start with the expression 
for the $L^2$ norm of $f$:

$$
\|f\|_{2}^{2} = \left<f, f\right> = \left<\sum_{i=1}^{\infty} 
\left<f,e_i\right>e_i, \sum_{j=1}^{\infty} \left<f,e_j\right>e_j\right>
$$

Expanding this using linearity and properties of inner products, 
we get:

$$
\|f\|_{2}^{2} = \sum_{i=1}^{\infty} \sum_{j=1}^{\infty} 
\left<f,e_i\right>\overline{\left<f,e_j\right>}\left<e_i,e_j\right>
$$

Since $\{e_i\}$ is orthonormal, when $i\neq j$, 
$\left<e_i,e_j\right>=0$. Therefore, the above expression 
simplifies to:

$$
\|f\|_{2}^{2} = \sum_{i=1}^{\infty} |\left<f,e_i\right>|^2
$$
\end{proof}
